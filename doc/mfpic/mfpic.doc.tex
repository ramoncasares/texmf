% mfpic.doc.tex (RMCG19940803)

\input doc
\rmcglayout
\font\logo=logo10 scaled 1200
\font\biglogo=logo10 scaled 2100
\def\MT{{\logo META}\-\TeX}
\def\MF{{\logo META}\-{\logo FONT}}

\def\macro #1 ={\par\hangindent20pt\noindent{\tt #1}\quad\ignorespaces}

%%% PRINCIPIO DEL TEXTO %%%

\title0 {{\tt graphbase.mf} en {\biglogo META}\TeX}

Pasos a seguir para usar "graphbase.mf" con {\MT}, o sea en un
fichero que ya importó {\MT} haciendo "\input metatex.tex".

\beginnumbers\parskip=\smallskipamount
\item Llamar al fichero "graphbase.tex" que incluye las macros
que adaptan mfpic a {\MT}.
\begincode
\input graphbase.tex
\endcode
\item Cada vez que se quieran utilizar las macros de "graphbase.mf"
hay que abrir el entorno en el que están definidas.
De este modo quedan protegidas y basta cerrar el entorno para que no
interfieran con otras macros.
\begincode
\mfpicopenenv
\endcode
\item Los caracteres que utilicen las macros de "graphbase.mf"
deben ser declarados utilizando un "beginchar" propio.
\begincode
\mfpicbeginchar(wd,ht,dp);
\endcode
\item Para finalizar la definición de un caracter también debe
utilizarse una orden propia.
\begincode
\mfpicendchar;
\endcode
\item Por último, para cerrar el entorno, se escribirá.
\begincode
\mfpiccloseenv
\endcode
\endnumbers

%\vglue1pc

En la definición de un carácter pueden utilizarse las siguientes
macros de {\MF}:

\macro pointd(a,ptwd) = Dibuja un punto de diámetro "ptwd" en "a".

\macro line(a,b) = Dibuja una línea de "a" a "b".

\macro dottedline(a,b,dlen,slen) = Dibuja una línea discontínua
desde "a" hasta "b".  Las partes dibujadas tienen una longitud
"dlen" ("4pt") y las silenciosas "slen" ("4pt").  Se obtiene un
efecto de línea de puntos con "dlen" = "1pt" y "slen" = "2pt".

\macro arrow(tl,hd,hlen) = Dibuja una flecha de "tl" a "hd" con una
punta (en el extremo "hd") de longitud "hlen" ("3pt").  Para dibujar
la punta se utilizan dos variables numéricas, "hdwdr" (por defecto
vale "1") que es la razón del ancho al largo de la punta, menor
valor más puntiaguda, y "hdten" (por defecto vale "1") que es el
valor de la tensión de las curvas Bezier que dibujan la punta, si
vale "infinity" entonces son rectas.

\macro dottedarrow(tl,hd,dlen,slen,hlen) = Idem, discontínua.

\macro axes(hlen) = Dibuja los ejes, en $h=0$ y $w=0$, con punta de
flecha de longitud "hlen" ("5pt").

\macro xmarks(len)(p,q,...) = Dibuja marcas de longitud "len"
("4pt") en los puntos "p,q,..." del eje x ($h=0$).

\macro ymarks(len)(p,q,...) = Dibuja marcas de longitud "len"
("4pt") en los puntos "p,q,..." del eje y ($w=0$).

\macro rect(ll,ur) = Dibuja un rectángulo cuya esquina inferior
izquierda está "ll" y cuya esquina superior derecha está en "ur".

\macro dottedrect(ll,ur,dlen,slen) = Idem, discontínuo.

\macro block(ll,ur) = Idem, sólido.

\macro rectshade(sp,ll,ur) = Idem, relleno de puntos, separados "sp"
("1pt", ratio de crecimiento $6/5$).  No dibuja los límites.  Si
"sp" vale "0", o menos, entonces el relleno es sólido.

\macro ellipse(center,radx,rady,angle) = Dibuja una elipse con
centro en "center" con radios "radx" y "rady" e inclinada "angle"
grados.

\macro ellshade(sp,center,radx,rady,angle) = Idem, relleno.

\macro circle(center,rad) = Dibuja un círculo centrado en "center" y
de radio "rad".

\macro circshade(sp,center,rad) = Idem, relleno.

\macro arc(from,to,sweep) = Dibuja un arco desde "from" hasta "to"
que representa un recorrido de "sweep" grados.

\macro arcarrow(hlen,from,to,sweep) = Idem, terminado en punta.

\macro arcshade(sp,from,to,sweep) = Idem, relleno.

\macro linedir(a,theta,len) = Dibuja una línea desde "a" en
dirección "theta" grados y de longitud "len".

\macro arrowdir(hlen,a,theta,len) = Idem, con punta.

\macro arcth(center,frtheta,totheta,rad) = Dibuja un arco centrado
en "center" de radio "rad" y que recorre desde el ángulo "frtheta"
hasta el ángulo "totheta".

\macro arctharrow(hlen,center,frtheta,totheta,rad) = Idem, con
punta.

\macro wedgeshade(sp,center,frtheta,totheta,rad) = Idem, relleno.

\macro curve(smooth,cyclic)(a,b,...) = Dibuja una curva si "smooth"
vale "true" o un polígono si "smooth" vale "false", cerrada si
"cyclic" vale "true" o abierta si "cyclic" vale "false", que pasa
por los puntos "a,b,...".

\macro curvedarrow(smooth,hlen)(a,b,...) = Idem, pero abierto y con
punta.

\macro cycleshade(sp,smooth)(a,b,...) = Idem, pero cerrado y
relleno.

\macro function(smooth,xmin,xmax,st)(fx) = Dibuja una función si
"smooth" vele "true", o una aproximación a tramos de una función si
"smooth" vale "false", descrita de la forma $f(x) = "fx"$ donde "fx"
es una expresión {\MF} cuya única incógnita es "x".  Se dibuja la
función desde $x = "xmin"$ hasta $x = "xmax"$ tomando pasos de
tamaño "st" e interpolando á la Bezier para dibujar el resto.

\macro parafcn(smooth,tmin,tmax,st)(ftx,fty) = Idem, pero la función
se expresa en función de un parámetro $t$ así:  $(x(t),y(t)) =
("ftx","fty")$ donde "ftx" y "fty" son expresiones {\MF} en "t".

\macro shadefcn(sp,xmin,xmax)(fcni)(fcnii) = Rellena la región
comprendida entre las rectas $x = "xmin"$, $x = "xmax"$ y las
funciones $f(x) = "fcni"$ y $f(x) = "fcnii"$ que no deben cortarse
en el intervalo.  Tanto "fcni" como "fcnii" son expresiones {\MF} en
"x".

\bye
