% verbatim.tex (RMCG20090909)

\input doc
\cm12fonts \rmcglayout \files \colors

\title0 Verbatim

Ampliaciones de la macro \verb|\verb"texto"| que aparece
definida en [382].

Quería diseñar una macro que me permitiera escribir verbatim tanto en el
papel como en un fichero, con la siguiente sintaxis
\beginpoints
\item \verb|\verb"texto"| en el papel
\item \verb|\verb\write\file"texto"| en el fichero, o
\item \verb|\verb\immediate\write\file"texto"| en el fichero,
inmediatamente.
\endpoints

El problema es insalvable, ya que el carácter limitador, \verb|"| en los
ejemplos, también puede ser el escape "\", no hay modo de decidir si con
\verb|\verb\write\file"texto"| se quiere escribir "write" en el papel,
 o "texto" en el fichero "\file".

Se requiere, por lo tanto, una nueva macro para escribir verbatim en un
fichero, a la que denominaremos "\fverb", y que se encuentra definida en
"explain.tex", así:

\begingroup\def\codeoptions{\parindent=0pt}
\begincode
\def\fverb#1\write{\ifx#1\immediate \let\next=\iverb
 \else \let\next=\wverb \fi \next}
\def\iverb#1{\begingroup\uncatcodeall\obeyspaces\obeylines\d@iverb#1}
\def\d@iverb#1#2{\def\next##1#2{\immediate\write#1{##1}\endgroup}\next}
\def\wverb#1{\begingroup\uncatcodeall\obeyspaces\obeylines\d@wverb#1}
\def\d@wverb#1#2{\def\next##1#2{\write#1{##1}\endgroup}\next}
\endcode
\endgroup

Las macros "\iverb" y "\wverb" son también útiles, por ejemplo
para formatear el índice, haciendo:
\verb!\wverb\auxf"\iverb\tocf|material en toc|"!.

Como prueba, véase que haciendo:

\begincode
\newwrite\borrar
\immediate\openout\borrar=auxiliar.out
\fverb\write\borrar\ es $ es \
\fverb\immediate\write\borrar"¿Adelanta,  o no?"
\fverb\immediate\write\borrar\&Entre escapes&\
\closeout\borrar
\endcode

Entonces en el fichero "auxiliar.out" tenemos:

\listing{auxiliar.out} % OJO, auxiliar.out debe estar cerrado ahora


\newcount\auxa\newcount\auxb\newcount\auxc
\catcode`\^=12
\def\hex{{\catcode`\^=12 {\tt ^^}}\afterassignment\hexx\auxa=`}
\catcode`\^=7
\def\hexx{\auxb=\auxa
 \divide\auxa by16 \auxc=\auxa
 \ifnum\auxa>9 \advance\auxa by87 {\tt\char\auxa}\else
  {\tt\number\auxa}\fi
 \multiply\auxc by16 \advance\auxb by-\auxc
 \ifnum\auxb>9 \advance\auxb by87 {\tt\char\auxb}\else
  {\tt\number\auxb}\fi}

Algunas versiones de \TeX\ son reluctantes a escribir en ficheros caracteres
no ascii imprimibles, en estos casos "¿" resulta \hex¿.

\bye

