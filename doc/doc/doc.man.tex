% doc.man.tex (RMCG19950306)

\input doc
\rmcglayout \files \colors
\parindent=20pt

\font\sf=cmss10 scaled1200

\font\xiilogo=logo10 scaled1200
\def\METAFONT{\hbox{\xiilogo META}\-\hbox{\xiilogo FONT}}
\def\METATeX{\hbox{\xiilogo META}\-\TeX}


\def\comando{\par\hang\noindent\begingroup\setupverbatim\doxverbatim}
\def\doxverbatim{\def\next ##1:{##1\endgroup\quad}\next}


\title0 DOC

\beginnarrow\beginnarrow\begincenter
Este documento explica el uso y las posibilidades que proporciona
"doc.tex".  Como primera aproximación, puede decirse que sirve
para facilitar la elaboración de documentos.
\endnarrow\endnarrow\endcenter


\title1 Estructura

El fichero "doc.tex" usa el formato "spplain.fmt" y
llama a los ficheros "explain.tex", "fonts.tex" e "index.tex",
de manera que puede hacer uso de todas los comandos (o macros) de:
"plain", "spplain", "explain", "fonts" e "index", además
de las definidas en el propio fichero "doc.tex".

Si el documento a procesar necesita, además de los comandos
proporcionadas por "doc", de otros comandos, entonces deben
importarse haciendo
 "\input filename".


\title1 Uso

Para que un documento sea procesado por "doc" deben ejecutarse
los siguientes pasos:
\beginpoints
\item Incluir en el fichero que precise de las definiciones de "doc"
la orden "\input doc".
\item Ejecutar {\TeX} con el formato castellano.  Vale
"tex &spplain filename".
\endpoints

Se puede utilizar en el fichero de entrada,
"filename" en el ejemplo anterior,
la codificación ISO-8859-1, o la UTF-8, incluso mezcladas.
Esto se consigue a cambio de perder dos caracteres en ISO-8859-1
que, de todos modos, no se usan en castellano:
\beginpoints
\item El "0xC2", que es la \^A (A circumfleja). % Â
\item El "0xC3", que es la \~A (A con tilde). % Ã
\endpoints

No es que se pueda utilizar cualquier caracter Unicode
usando la codificación UTF-8,
sino cualquiera de los caracteres del subconjunto Latin-1,
esto es, cualquiera de los 256 primeros códigos de Unicode.

Otro caracter ISO-8859-1 perdido es el "0xE2",
que es la \^a (a circumfleja), % â
en este caso para poder pillar algunos códigos UTF-8 de tres octetos,
entre los que está el €.
Por cierto, el ISO-8859-1 utilizado no es puro,
ya que en la posición "0xA4" está el carácter €.
Es decir, hago para el € la misma substitución que ISO-8859-15.

\title1 Comandos

\title2 Tipos de documento

Existen cuatro tipos de documento que se pueden utilizar llamando a los
siguientes comandos:

\comando\plainlayout:  Que usa la disposición de "plain", o sea, sin
cabeceras y con el número de hoja (con romanos en minúscula) centrado
en el pie, sobre una hoja A4.  Es el usado por defecto.

\comando\rmcglayout:  Que no usa pie y dispone una cabecera con los
siguientes datos: mi página web, la fecha en formato ISO,
el nombre del fichero y el número de página (con romanos en mayúscula).
Hoja tamaño A4.

\comando\letterlayout:  Que no usa pie y conforma la cabecera con las
siguientes informaciones: mi dirección, la dirección de destino
definida como "\address" para la que valen "\\" como separadores de
líneas y la fecha definida como "\date". Hoja tamaño A4.

\comando\booklayout:  Que usa una hoja tamaño A5, no usa cabecera,
y en el pie solo la página en el lado exterior.


\title2 Tipografía

No es muy variada. Mi idea es que la font Computer Modern de 
\person{Knuth} sea como mi letra escrita, es decir, una de las
maneras de reconocer mis escritos. Así que, aunque yo no sea el
único que la usa, me propongo escribir siempre con ella.

Aún así, hay dos variantes de esta font en el formato Type1:
la versión mantenida por AMS, que conserva el nombre, y
la conocida como Latin Modern, que añade los caracteres
acentuados, y otros. Además, por razones históricas, mantengo
también las familias originales de PostScript.

La elección de uno de estos tres tipos, y su tamaño base, se hace con
los siguientes comandos:

\comando\cm10fonts: Que selecciona las fonts de AMS a 10pt.
\comando\cm12fonts: Que selecciona las fonts de AMS a 12pt.
\comando\lm10fonts: Que selecciona las fonts Latin Modern a 10pt.
\comando\lm12fonts: Que selecciona las fonts Latin Modern a 12pt.
\comando\ps10fonts: Que selecciona las fonts de PostScript a 10pt.
\comando\ps12fonts: Que selecciona las fonts de PostScript a 12pt.


\title2 Idiomas

Al elegir el idioma se eligen las reglas de separación de palabras,
y las palabras de encabezamiento, como ``page'' o ``página''.
Sólo soporto tres idiomas:
\comando\language0: Que selecciona el inglés.
\comando\language1: Que no selecciona ningún idioma, y por consiguiente
no parte las palabras.
\comando\language2: Que selecciona el castellano (español).


\title2 Seccionamiento

Existen seis niveles de sección, desde 0 para el título del
documento, hasta 5.  La cabecera del nivel 3 se marca
"\title3 Título \par".  El {\tt\char32\string\par} puede ser
sustituido por una línea en blanco. Además el nivel 9 puede
usarse para las secciones especiales, como el índice.

Existen dos opciones de numeración.  O bien no se numera, o bien se
numera.  La numeración, que es la opción por defecto, es tipo
decimal, es decir, un título del nivel 3 se compondrá de tres
números decimales separados por puntos, por ejemplo 3.4.2 si se
trata de la segunda subsubsección de la cuarta subsección de la
sección tercera.  Se usan los siguientes comandos:

\comando\numbers:  Que activa la numeración.  Es el valor
por defecto.

\comando\nonumbers:  Que desactiva la numeración.


\title2 Documento activo

Se puede recomponer el orden del documento usando dos comandos:  el
comando {\TeX} "\input filename.ext" y el comando "\tofile".

\comando\tofile{filename.ext}<marca de fin>:  Que crea un fichero con el
nombre dado, en el ejemplo "filename.ext", con lo que sigue hasta
que se encuentre una línea idéntica, en el caso del ejemplo
mostrado, a \hbox{"<marca de fin>"}.

Además, tras hacer "\newwrite\file \openout\file=filename", puede
pasarse información verbatim al fichero "filename.tex" usando el
comando "\fverb" antes de cerrarlo con "\closeout\file".  Las
escrituras no "\immediate" se efectúan en la rutina de salida, por
lo que si se hace un "\immediate\closeout\file", pueden quedar
algunas de estas escrituras en el fichero "log".  La sintaxis de
"\fverb", donde "\immediate" es opcional, es:

\comando\fverb[\immediate]\write\file<char><texto><char>:  El carácter
"<char>" debe ser el mismo al comienzo y al final.

Por último, la orden "\filed" está pensada para tomar secciones de
otros ficheros, sin más que dar el nombre del fichero y el título de
la sección. En detalle se explica ahora.

\comando\filed filename \anyCS Title of Section: Que, comenzando desde
el principio del fichero "filename", ignora todo hasta que se encuentra
una línea idéntica a "\anyCS Title of Section". Desde ese punto
va tipografiando normalmente, como si lo leyera del fichero
actual, hasta que se encuentra un comando "\anyCS" o "\endanyCS",
momento en el que regresa al fichero actual.


\title2 Verbatim

El carácter comillas dobles (\verb|"|) sirve para marcar el inicio
y el final de una secuencia de caracteres que se quiere que figuren
tal y como son. Nótese que dentro de la zona entrecomillada se 
respetan los fines de línea y los espacios.

\comando"<texto verbatim>": Que reproduce verbatim, y en una font
de espaciado fijo, lo contenido entre las comillas.

\comando\verb<char><texto><char>:  El carácter "<char>" debe ser el
mismo al comienzo y al final. Si el carácter comillas dobles 
forma parte de la secuencia, entonces ha de utilizarse esta forma.

\comando\listing{filename.ext}:  Que escribe verbatim el contenido
del fichero nombrado, "filename.ext" en el ejemplo.

\comando\plisting{filename.ext}
FIRST-LINE
END-LINE:\\Que escribe verbating parte del contenido
del fichero "filename.ext",  concretamente desde
la primera línea que sea idéntica a la que sigue a la orden,
"FIRST-LINE" en el ejemplo,
hasta la siguiente línea que siga a esa y que sea idéntica
a la subsiguiente, "END-LINE" en el ejemplo.

\comando\file{filename.ext}:  Que escribe verbatim el contenido
del fichero nombrado, "filename.ext" en el ejemplo. A diferencia
de "\listing", encabeza el listado con el nombre del fichero, y
numera las líneas del fichero.

\comando\pfile{filename.ext}
FIRST-LINE
END-LINE:\\Que escribe verbatim una parte del fichero "filename.ext".
A diferencia de "\plisting", numera las líneas del fichero.

\comando\begincode:  Que escribe verbatim hasta que se encuentra
un "\endcode". Usa, como "\verbatimoptions" lo que esté definido
como "\codeoptions"; por defecto indenta 20pt.
Haciendo "\let\codeoptions\numbercodeoptions" se numeran las líneas.

\title2 Índices

Para que se efectúe automáticamente el índice son precisos los
siguientes comandos:

\comando\files:  Que debe ser incluido antes de que aparezca la
primera información que deba ser puesta en el índice y antes de que
aparezca la primera cita o referencia. Esto es suficiente
para que se añada un índice al documento "pdf", pero que no
aparece impreso.

\comando\input auxiliar.toc:  Que imprime en ese lugar la tabla de
contenidos.


\title2 Citas

Las referencias a otros documentos usan dos comandos:

\comando\bibitem{etiqueta}:  Después del comando "\bibitem", y
completando el párrafo, o sea, antes del próximo "\par" o línea en
blanco, debe darse noticia completa de la publicación a la que se
refiere.  La etiqueta debe ser descriptiva, como "PUSI4/2" o
"señalización E y M".

\comando\cite{etiqueta}:  El comando "\cite" se usará para referirse
al documeto.  Al imprimir, por ejemplo, "\cite{PUSI4/2}" aparecerá
[2], si tal documento fue aquel presentado como "\referencia"
segunda (en orden de aparición).


\title2 Referencias

Si se quiere referir en un punto de un documento a otro punto del
documento se utilizarán los siguientes comandos:

\comando\label{etiqueta}{texto}:  Que marca el lugar al que se quiere
referir en otro lugar.
\comando\labeled: Para marcar una sección debe usarse el prefijo
"\labeled" del comando "\title", por ejemplo:
 "\labeled\title3 Título \par". En este caso la etiqueta es
el "Título" de la sección. Si se quiere usar otra etiqueta, ésta se coloca
entre "\labeled" y "\title", así,
 "\labeled etiqueta\title3 Título \par" o, lo que es lo mismo,
 "\labeled{etiqueta}\title3 Título \par".

\comando\ref{etiqueta}: Que imprime, en el lugar en el que
aparece, el "texto" que era el segundo argumento de "\label".
Si lo que se produce es un fichero "pdf", entonces pulsando
el "texto" en el visor de ficheros "pdf" se va al lugar en el
que está el "\label".

\comando\refsc{etiqueta}:  Que imprime, en el lugar en el que
aparece, la sección en la que se encuentra la "etiqueta" marcada
con "\label". 
Pulsando la sección en el visor "pdf" se va a "\label".

\comando\refpg{etiqueta}:  Que imprime, en el lugar en el que
aparece, la página en la que se encuentra la "etiqueta" marcada con
"\label".
Pulsando la página en el visor "pdf" se va a "\label".

Por ejemplo, si quiero hacer referencia a la sección que trata sobre los
tipos de letra, la defino como "\labeled\title2 Tipos de letra", y entonces
escribiendo aquí "\refsc{Tipos de letra}" se imprime \refsc{Tipos de letra}
y escribiendo "\refpg{Tipos de letra}" se imprime \refpg{Tipos de letra}.

\comando\URL{texto}<dirección>: que hace una referencia externa.
El argumento primero, "texto" en el ejemplo, es opcional.
Si existe, entonces lo imprime en el lugar en el que aparece, y
si no existe, entonces imprime el segundo en letra de máquina.
En cualquiera de los casos, pulsando lo impreso en el visor "pdf"
se intentará abrir el URL que aparece en el segundo argumento,
"dirección" en el ejemplo.

Así, para referirme a mi página web escribo
"\URL<http://www.ramoncasares.com/>" y aparece
\URL<http://www.ramoncasares.com/>.
También podría haber escrito,
"véase \URL mi página web<http://www.ramoncasares.com/>",
y quedaría oculta la dirección, así:
véase \URL mi página web<http://www.ramoncasares.com/>.


\title2 Notas

Para poner una nota a pié de página, 
como ésta\footnote{Nota a pié de página},
se escribe "\footnote{Nota a pié de página}".


\title2 Comentarios

Se pueden escribir en el fichero fuente anotaciones que no serán
impresas.  Hay varias posibilidades:

\comando%:  Que ignora el signo "%" y lodo lo que se encuetra en la
misma línea y a la derecha de él.

\comando\comment<marca fin>:  Que ignora todo lo escrito hasta que se
encuentra una línea cuyo contenido coincide con "<marca fin>".

\title2 Páginas

La división en páginas puede hacer uso de los siguientes comandos:

\comando\newpage:  Fuerza el comienzo de una hoja.

\comando\goodpage:  Hace una división de página si queda menos de
un tercio de hoja sin escribir.

\title2 Líneas

En general, {\TeX} no respeta los finales de línea que figuran en el
fichero fuente, sino que toma el párrafo completo y lo divide en líneas
del mejor modo posible. Sin embargo, hay varias maneras de alterar
este comportamiento.

\comando\newline: Fuerza una línea nueva.

\comando\\: Consigue el mismo efecto que "\newline", aunque con más
generalidad, ya que éste también funciona en alineaciones.

\comando\beginlines: El entorno comprendido entre un "\beginlines" y
su correspondiente "\endlines" respeta los finales de línea, es decir,
añade un "\newline" al final de cada línea del fichero fuente.

\title2 Párrafos

El comando "\par" o una línea en blanco marca la división en párrafos.  Un
párrafo normal sólo tiene indentada la primera línea un grado por la
izquierda, o sea, "20pt".  Existen otros tipos de párrafos.  Algunos
aplican sólo al párrafo que se escribe a continuación, mientras que los que
tienen tanto signo de apertura, "begin", como de cierre, "end", tienen
efecto sobre todos los párrafos que median entre los signos.

\comando\beginnarrow:  Comienza una serie de párrafos indentados por la
derecha y por la izquierda un grado más.  Agrupa hasta el próximo
"\endnarrow".

\comando\begincenter:  Comienza una serie de párrafos centrados.  Agrupa
hasta el próximo "\endcenter".

\comando\point: Que es sinónimo de "\bulletedpar",
y que indenta todas las líneas un grado por la izquierda
y pone un punto a la izquierda de la primera.

\comando\subpoint: Que es sinónimo de "\subbulletedpar",
y que indenta todas las líneas dos grados por la izquierda y
pone un círculo delante de la primera línea.

\comando\subitem3*$\Delta$*: Que indenta todas las líneas "3" grados,
en el ejemplo, por la izquierda, y pone una $\Delta$ ("$\Delta$"),
en el ejemplo, delante de la primera línea.

\comando\*2*: Que  indenta todas las líneas un grado por la izquierda
y pone lo que está entre los asteriscos, un ``2'' en este caso,
a la izquierda de la primera línea.

\comando\indented2par:  Indenta todas las líneas del párrafo
"2" grados, en el ejemplo, por la izquierda.

\comando\unindentedpar:  Indenta todas las líneas un grado excepto la
primera, que se ajusta al margen.

\unindentedpar"\describe Title:"\quad Que es sinónimo de "\titledpar",
y que indenta todas las líneas, excepto la primera,
 un grado por la izquierda.
La primera la comienza, sin indentar, escribiendo en negrita "Title".

\title2 Listas

Los últimos tipos de párrafos deben utilizarse en listas.
Cada tipo tiene un comando de comienzo, tipo "\beginlist",
y un comando de fin, como "\endlist", dentro del cual
cada punto debe encabezarse con un comando "\item".
Pueden empotrarse unas listas en otras, es decir,
dentro de una lista puede ponerse otra.
Hay tres tipos.

\comando\beginpoints: Que termina con el comando "\endpoints",
y que encabeza cada "\item" con un punto ($\bullet$, $\circ$,
$\cdot$, según su nivel).

\comando\beginnumbers: Que termina con el comando "\endnumbers",
y que encabeza cada "\item" con un número ({\bf1}, 2,
iii, según su nivel).

\comando\begintitles: Que termina con el comando "\endtitles".
En este caso "\item" toma un argumento, que se toma como encabezado,
o sea, que "\item{Título}" comienza con su {\bf Título}.


\labeled\title2 Tipos de letra

Los tipos básicos son: {\rm normal} ("\rm"), {\bf negrita\/}
("\bf"), {\sl inclinada\/} ("sl"), {\it cursiva\/} ("\it"), {\sf
palo seco} ("\sf") y {\tt máquina} ("\tt").

Para enfatizar se recomienda el comando "\em" dentro de un grupo, ya
que incluye automáticamente la corrección itálica.  O sea,
"{\em esto está {\em doblemente} enfatizado}" resulta
 {\em esto está {\em doblemente} enfatizado}.


\goodpage
\title2 Figuras

Para poder dibujar figuras con {\METAFONT} debe utilizarse el comando
"\input metatex" que importa {\METATeX}.
Para facilitar su uso, además de las órdenes de {\METATeX},
pueden usarse las macros "\MTfigure" y "\MTendfigure".
"\MTfigure" equivale a "\MTbeginchar",
 pero hace "\input metatex" si no se ha hecho.
"\MTendfigure" tiene dos formas:  si se quiere que la figura tenga un
título, entonces toma la forma \verb=\MTendfigure"Título";=
y si no se quiere titular entonces toma la forma "\MTendfigure;".
En cualquiera de los casos, dibuja la figura centrada en el
lugar en donde aparece (hace un "\box\MTbox").

Por ejemplo, el código:

\begincode
\MTfigure(10cm,4cm,0cm);
\MT: draw (0,0)--(w,0)--(w,h)--(0,h)--cycle;
\MT: draw (0,0)--(w,h); draw (0,h)--(w,0);
\MTendfigure"Un sobre, o rectángulo aspado";
\endcode

produce:

\MTfigure(10cm,4cm,0cm);
\MT: draw (0,0)--(w,0)--(w,h)--(0,h)--cycle;
\MT: draw (0,0)--(w,h); draw (0,h)--(w,0);
\MTendfigure"Un sobre, o rectángulo aspado";


%\goodpage
\title2 Alineaciones

Para imprimir listas o pequeños cuadros debe utilizarse el comando
"\beginalign".  Desde ese comando hasta el comando complementario
"\endalign", pueden usarse las siguientes convenciones:  el carácter "&"
que señala la separación entre columnas, el comando "\\" que señala el fin
de la línea, y el comando "\rule" que dibuja una línea horizontal entre
líneas.

\bigskip
\line{\vbox{\hsize6truecm
Por ejemplo:\medskip
"\beginalign
\rule
 Uno& Primero& 1\\
 Dos& Segundo& 2\\
 Tres& Tercero& 3\\
 Cuatro& Cuarto& 4\\
\rule
\endalign"}\quad
\vbox{\hsize6truecm
resulta:
 \beginalign
\rule
 Uno& Primero& 1\\
 Dos& Segundo& 2\\
 Tres& Tercero& 3\\
 Cuatro& Cuarto& 4\\
\rule
\endalign}\hfil}


\title2 Marcas de revisión

Importando "marks.tex", o sea, haciendo "\input marks", se tiene
la posibilidad de incluir marcas de revisión en un documento.

Las partes añadidas deben ser marcadas entre "\new/" y "\wen/", y
las partes eliminadas deben ser marcadas entre "\del/" y "\led/".

Existen dos posibilidades de impresión.  Por defecto, o haciendo
uso de "\nomarks", las partes añadidas aparecen normalmente
impresas y las eliminadas no aparecen.  Haciendo "\marks" las
partes añadidas aparecen subrayadas y las eliminadas tachadas, y
además en el margen derecho aparecen indicaciones de que algo
ha sido modificado.

{\sl Nota}:  Cuando "\marks" está activo, el final de línea cuenta como un
espacio normal, o sea, no es ni un espacio subrayado como los otros, ni un
espacio tachado.  Una solución es terminar las líneas que caen dentro de
una zona modificada con un comentario "%".


\title2 Varios

\comando\fecha: Que escribe ``\fecha''.
\comando\todayiso: Que escribe ``\todayiso''.
\comando\todayISO: Que escribe ``\todayISO''.

\title9 Índice

\input auxiliar.toc


\bye
