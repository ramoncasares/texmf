% RCstyle.man.tex (RMCG20200328)

\input RCstyle
\darkcolors

\font\sltt=cmsltt10 at 12pt

\catcode`\"=13 \def"{\verb"}

\def\command{\par\begingroup\truecolor{Blue}\everypar{}\noindent
 \setupverbatim\doxverbatim}
\def\doxverbatim{\def\next ##1:{##1\endcolor\endgroup
 \hangindent20pt\quad}\next}

%\def\version{20191231}  %% uncomment to fix version

%\def\new/{\relax}
%\def\wen/{\relax}

%%%%%%%%%%%%%%%%%%%%%%%%%%%%%%%%%%%%%%%%%%%%%%%%%%%%%%

\title{RCstyle: The Manual}
\author{Ramón Casares}
\contact{{\sc eMail:} {\tt\URL papa@ramoncasares.com<mailto:papa@ramoncasares.com>}}
\subject{Document composition}
\keywords{TeX, document composition}

\maketitle

\beginnote
This is version {\tt\version}.\\
\copyright\ 2020--2022 Ramón Casares; licensed as {\tt cc-by}.\\
Any comments on it to
{\tt\URL papa@ramoncasares.com<mailto:papa@ramoncasares.com>}
are welcome.
\endnote

\beginabstract
This document explains how to write papers
in {\sltt plain} {\TeX}
with the help of the {\sltt RCstyle} macros.
\endabstract

%\let\tocline=\toclineX
\maketoc


\section{Introduction}

This manual assumes that the user can use
and is familiar with "plain" {\TeX}.
In fact, "RCStyle" adds some functionality
on top of "plain".
I will not explain here the "plain" {\TeX} commands,
since they are much better explained by \cite Knuth (1984).

File "RCstyle.tex" works with format "plain.fmt" and it calls
files "indexes.tex", "utf.tex" and "fonts.tex",
so it can use every macro in
"plain", "indexes", "utf" and "fonts"
in addition to  those defined in "RCstyle" itself.

The files to be processed by "RCstyle" should be
UTF-8 coded, though only the characters in the Latin-1 range,
which are codes from U+0000 to U+00FF,
will be served.

The text will be hyphenated using the rules for English,
in particular those defined for the "plain" format.

By default, "RCstyle" produces
 A4 ($210\,\hbox{mm}\times297\,\hbox{mm}$) documents
using the Computer Modern fonts at 12 points.


\section{Use}

In order to transform file "filename.tex"
into a "RCstyle"d document,
two steps are needed:
\beginpoints
\point To include the command "\input RCstyle"
       in the file "filename.tex".
\point Execute {\TeX} with its "plain" format.
       Just type "tex filename" in a CLI.
\endpoints

If everything goes well, you will get a "filename.dvi" file.
This file is not usually the final product,
but rather a PDF file, and there are three ways to produce it.

After doing "tex filename", you get a "filename.dvi" file
with "\special" commands that are understood by dviPDFm.
So typing "dvipdfm filename" will produce file
"filename.pdf" in the same directory.
So this is the first path:
\beginpoints
\item{1} "tex filename",
 to get file "filename.dvi" with dviPDFm "\special"s, and
\item{2} "dvipdfm filename",
 to get file "filename.pdf".
\endpoints
The resulting PDF file is the shortest of the three.

A shorter path is to type "pdftex filename",
which will result in file "filename.pdf" directly.
This PDF file is longer than the one produced by dviPDFm,
though their contents are the same,
because of the way PDF{\TeX} deals with fonts,
which is not as good as the one used by dviPDFm.
So this is the direct path:
\beginpoints
\item{1} "pdftex filename",
 to get file "filename.pdf".
\endpoints
Regarding its size,
the resulting PDF file is in the middle of the three.

A longer path that results in a PDF file with the same
contents, but also in a PostScript version, involves three steps:
\beginpoints
\item{1} "tex '\let\dvips\relax \input filename'",
 to get file "filename.dvi" with dvips "\special"s,
\item{2} "dvips -j -K -M -z filename",
 to get file "filename.ps", and
\item{3} "ps2pdf filename.ps",
 to get file "filename.pdf".
\endpoints
The resulting PDF file is the longest of the three.



\section{Characters}

The input file, "filename.tex" in the example above,
has to be a UTF-8 coded text file.
However, not every character in Unicode is available,
but only those in the Latin-1 subset,
which are those from U+0000 to U+00FF,
which are also those in the ISO-8859-1 set.
So, besides all ASCII characters,
all those having "c2" and "c3" as the
first byte of its UTF-8 coding are included.

A missing character is the European currency, €.

File "utf.tex" deals with the UTF-8 encoded characters,
so it is in there where the whole truth about characters
is said. And the whole truth is that
some Latin-1 characters are not available in
OT1 fonts, as the Computer Modern fonts, as the
cent, pound, currency, and yen, while others as
$^\circ$, $\pm$, $^2$, $^3$, $1/4$, $1/2$ and $3/4$
are only available in math mode or via "plain" macros.
French guillemots are not available in Computer Modern,
but they are replaced with their corresponding quotations.
In compensation, some characters in Unicode plane U+20XY
are available: quotations, left and right, single and double,
plus the German low opening ones;
hyphens, short, long and for number ranges; and
dagger, double dagger and ellipsis.

Other characters, mainly accented ones, but also
other that are composed,
as the polish {\L} (U+0141) and {\l} (U+0142),
are available as "plain" macros, "\L" and "\l".
These can be used freely in normal text,
but not directly in commands, as "\section" or "\cite".
For example, to include these two polish ones,
{\L} and {\l}, in titles, the following code
has to be written {\em before} the command "\input RCstyle":
\verb|
   \let\oldL=\L \let\oldl=\l
   \def\L/{\oldL}\def\l/{\oldl}
   \def\doextrasave{\def\L/{\string\L/}\def\l/{\string\l/}}
   \def\doextrapdf{\def\L/{\string^^95}\def\l/{\string^^9b}}|
\continuepar
Firstly, macros "\L" and "\l" have to be redefined
as delimited, "\L/" y "\l/", to avoid space problems.
Then, the "\doextrasave" saves then as delimited macros
in the auxiliar files, and the "\doextrapdf" translates them
to the PDFDocEncoding used in PDF outlines and doc info.
If they were not in PDFDocEncoding, then you
will have to resort to UTF16BE encoded strings,
either bypassing or redefining some critical macros in "indexes.tex",
but that will not be explained here.

Other useful characters for mathematics,
which are not directly provided by "RCstyle",
can be incorporated easily,
though they will not work in  commands, as "\section" or "\cite".
\command\input utf8greek: To use UTF-8 coded Greek characters
 in maths; please note that
 they are not ready for long texts in Greek.
\command\input utf8maths: To use a long list of
 UTF-8 coded mathematical characters,
 instead of its corresponding "plain" {\TeX} macros.


\section{Fonts}

By default, "RCstyle" uses
the Computer Modern fonts at 12 points,
separating 14.4 points the bases of the lines;
the main font is "cmr12".
Besides the "plain" font commands for
{\rm normal} ("\rm"),
{\bf bold\/} ("\bf"),
{\sl slanted\/} ("\sl"),
{\it italic\/} ("\it"),
{\tt typewriter} ("\tt"), and
{\em emphasis} ("\em"),
which "RCstyle" redefines to do automatic italic correction,
the commands for
\command\sc: {\sc small caps},
\command\sf: {\sf sans serif},
\command\sfit: {\sfit slanted sans serif},
\command\bb: blackboard bold, as in {\bb N}, and
\command\frak: fraktur, as in {\frak R}, can also be used.

Another font command that could be used in a document is "\bold".
\command\bold: Selects bold text and maths.
\command\roman: Restores normal fonts in text and maths.

To manage the fonts, "RCstyle" calls "fonts.tex",
which is based on
\URL PDFSEL<https://ctan.org/pkg/pdcmac> by Damian Cugley.
The data base in "fonts.tex" defines several font sets:
\command\cmfonts: Selects the Computer Modern fonts at 10 pt.
\command\xiifonts: Selects the Computer Modern fonts at 12 pt.
 This is the default fontset.
\command\xivfonts: Selects the Computer Modern fonts at 14 pt.
\command\xviifonts: Selects the Computer Modern fonts at 17 pt.
\command\xxifonts: Selects the Computer Modern fonts at 21 pt.
\continuepar
All the previous ones, except the last one,
have bold variants:
\command\cmbfonts: Selects the Computer Modern bold fonts at 10 pt.
\command\xiibfonts: Selects the Computer Modern bold fonts at 12 pt.
\command\xivbfonts: Selects the Computer Modern bold fonts at 14 pt.
\command\xviibfonts: Selects the Computer Modern bold fonts at 17 pt.
\continuepar
In addition, other fonts are available:
\command\lmxiifonts: Selects the Latin Modern fonts at 12 pt.
\command\psxiifonts: Selects the PostScript fonts at 12 pt.
\continuepar
Some more are also available,
but it is not recommended to use
any of the macros in this paragraph,
except "\lmxiifonts", if you need the EUR symbol, for example.


\section{Front matter}

The information about the document, as the title and the author,
should be provided at the beginning of the file.
These are the helping macros.
\command\title{Title}: Defines the title
 of the document, both in print and in the PDF outlines and properties.
\command\author{Author}: Defines the author
 of the document, both in print and in the PDF properties.
\command\contact{Contact}: Provides an author contact, an address
 or whatever should added in print after the author.
\command\subject{Subject}: Defines the subject
 of the document, both in print and in the PDF properties.
\command\keywords{keyword, other keyword}: Defines the keywords
 of the document, both in print and in the PDF properties.

These previous five commands can be given in any order,
but they only become effective when the following commands
are processed.
\command\infodoc: Writes the "Title", the "Author",
the "Subject" and the "keyword"s in the PDF properties dictionary.
\command\maketitle: Executes "\infodoc" and prints,
in this order, centered, and in different fonts,
the "Title", the "Author"
and, if it was already defined, the "Contact".
It also adds a level 0 destination, or root,
named "Title" with an empty {\sc secid} to the document
hierarchy.
\command\beginabstract: Starts the document abstract.
\command\endabstract: Finishes and prints the abstract and,
 if they were defined, it also prints the "keyword"s.

Should some other information about the document were needed,
this should be included in a non-numbered footnote.
\command\beginnote: Starts a non-numbered footnote.
 It can be used in the first page to give additional data.
\command\endnote: Finishes the non-numbered footnote.

The table of contents is also front matter,
but we will postpone its explanation till later,
see \refsc{Table of contents},
because we should explain sectioning and referencing before.


\section{Sectioning}

Four levels are implemented,
and the root level, or level 0, corresponds to the
document itself represented by its "\title{Title}",
as defined in \refsc{Front matter}.
The three other levels are the section (level 1),
 the subsection (level 2), and the clause (level 3),
which can be numbered or non-numbered,
so they are served by six macros.
\command\section{Section title}: Starts and prints
 numbered section "Section title".
 It also adds a level 1 destination named "Section title" with its {\sc secid},
 which is \S1 for the first, and so on.
 Its paragraphs are numbered, beginning with \P1.
\command\subsection{Subsection title}: Starts and prints
 numbered subsection "Subsection title".
 It also adds a level 2 destination named "Section title" with its {\sc secid},
 which is \S1.1, or similar.
 Its paragraphs are numbered, beginning with \P1.
\command\xsection{Section title}: Starts and prints
 non-numbered section "Section title".
 It also adds a level 1 destination named "Section title" and an empty {\sc secid}.
 Its paragraphs are non-numbered.
\command\xsubsection{Subsection title}: Starts and prints
 a non-numbered subsection  titled "Subsection title".
 It also adds a level 2 destination named "Subsection title"
 and an empty {\sc secid}.
 It continues paragraph numbering without altering it:
 If the previous paragraph was \P4,
  then its first one will be \P5 and so on,
 and if the previous paragraph was not numbered,
  then its paragraphs  will not be numbered.

The last two sectioning commands are not general purpose,
as the previous four.
You can see them used in \cite Casares (PT).
\command\clause{Theorem}: Starts and prints
 a level 3 numbered very short section,
 as a theorem or a definition,
 but no destination is added.
 The “claused” paragraph will be hanged.
 Its paragraphs will not be numbered.
\command\comment{Comment}:  Starts and prints
 a level 3 non-numbered  very short section,
 typically a comment to a "\clause",
 and no destination is added.
 The “commented” paragraph will be hanged.
 Its paragraphs will not be numbered.

Sometimes you will not want to start a numbered paragraph.
\command\continuepar: Starts a non-numbered paragraph.


\section{Referencing}

\subsection{Inside}

A destination can be defined anywhere in the document.
\command\label{label}{text}:  Adds a destination
 named "label" and content "text". The "label" has to
 be unique in the document. It also takes note
 of the current {\sc secid} and page.
 The next destination number is assigned to this label.

In addition, by default, every sectioning command adds
a destination where both the "label" and the "text"
are the title of the section or subsection.
But it is possible to change the "label".
\command\labeled{new label}: Causes the next sectioning command
 to use "new label" as its label, instead of using its title.
 This is needed to avoid repeating a label if, for example,
 two subsections are titled the same.

And a reference to any destination can be done in four ways.
In a PDF viewer, clicking on the “printed” reference
sets the focus on the referred to destination.
\command\ref{label}: Prints the "text" of the
 referred to destination "\label{label}{text}".
\command\refsc{label}: Prints the {\sc secid} of the
 referred to destination.
\command\refpg{label}: Prints the page number of the
 referred to destination.
\command\refx{variant}{label}: Prints "variant".

For example, if I want to refer to the section on fonts, then:
\beginpoints
\point writing "\ref{Fonts}" it prints \ref{Fonts},
\point writing "\refsc{Fonts}" it prints \refsc{Fonts},
\point writing "\refpg{Fonts}" it prints \refpg{Fonts}, and
\point writing "\refx{see fonts}{Fonts}" it prints
                \refx{see fonts}{Fonts}.
\endpoints


\subsection{Citing}

Citing is just a specialized form of inside referencing.
In this context, "\biblabel"s are used instead of "\label"s,
and "\cite"s instead of "\ref"s.
{\em See also} \refsc{Bibliography}.

\command\biblabel Knuth (1984): Starts a reference destination
with the information about the book, article, or document.
Macro "\biblabel" is delimited thus: "#1 (#2)",
meaning that the space and the pair of parenthesis are mandatory.
The real "label" of "\biblabel Name (MCDU)" is "NameMCDU".
Then, "\refpg{Knuth1984}" prints \refpg{Knuth1984}.

\command\cite[\1's \2 book] Knuth (1984): Refers to reference
destination \cite Knuth (1984), but it prints \cite[\1's \2 book] Knuth (1984).
The bracketed part is optional, and it explains how to format it,
being "\1" the first parameter
 (between the closing bracket and the opening parenthesis)
and "\2" the second parameter (inside the parenthesis).
The default format is "\1 (\2)".



\subsection{Outside}

References to sites and objects in the Internet use this command.

\command\URL text<address>: Refers to the Internet resource
"address", which should be a URL, and prints "text".
Argument "text" is optional, and if it is empty,
then "address" will be printed.
\continuepar
For example,
“"see \URL my web site<http://www.ramoncasares.com/>"”, prints
 “see \URL my web site<http://www.ramoncasares.com/>”, and
“"\URL<http://www.casar.es/>"” prints
 “\URL<http://www.casar.es/>”.


\section{Table of contents}

The table of contents is created automatically.
\command\maketoc:  Prints the table of contents. It also creates
 a level 1 destination named "Contents" and {\sc secid} "\lq",

The default presentation is the one used in this document,
but it is not difficult to define others.
However, some more explanations are needed.

Table of contents information is firstly recollected
in file "auxiliar.aux" with other data and then
it is translated to file "auxiliar.toc".
File "auxiliar.toc" is composed of lines formatted:
"\tocline{level}{text}{destno}{secid}{page}".
Then, the format of the table of contents depends
on the current definition of macro "\tocline"
when the "\input auxiliar.toc" is met.
It is just as simple as that.

Both, "\files" (executed by "RCstyle") and "\maketoc"
do an "\input auxiliar.toc",
but their definitions of "\tocline" are different,
because "\files" compose the PDF outlines,
which require the PDFDocEncoding coding, and "\maketoc"
prints the table of contents, which can use different fonts.

The "\tocline" definition in "\files"
should not be modified, though it can be altered
defining "\doextrapdf", as shown in \refsc{Characters}.
This is how "\files" processes a generic
"\tocline{level}{text}{destno}{secid}{page}":
\beginpoints
\point If "level" is greater than 9: the line is ignored, so it will not be
 in the PDF outlines.
\point If "secid" is "*": the line is also ignored.
\point If "secid" is either empty,
 or is "\lq" (`), or  "\rq" ('), or "\lbrack" ([), or "\lbrack" (]):
 "text" is added to the outline at level "level".
\point Else: "sedid text" is added to the outline at level "level".
\endpoints

The "\tocline" definition in "\maketoc" can be modified.
"RCstyle" provides some standard definitions.
The basic one, is the hierarchical "\toclineX". This is how\\
"\toclineX{level}{text}{destno}{secid}{page}" works:
\beginpoints
\point If "secid" is "*": "\csname text\endcsname" is executed,
 so if "text" were "bigskip", then "\bigskip" will be executed.
\point If "secid" is "\lq" (`) or "\rq" ('): the line is ignored,
 so it is not printed.
 This is why Contents is not printed in the table of contents.
\point If "secid" is "\lbrack" ([) or "\rbrack" (]):
 "text", indented "level" times from the left border,
 leaders and the page number, at the right in bold face, are printed.
\point Else: "secid text", indented "level" times from the left border,
 leaders and the page number, at the right in bold face, are printed.
\endpoints
Therefore, if you write "\let\tocline=\toclineX" just before the
"\maketoc" you will get a hierarchical table of contents
instead of the default one.

The default table of contents, "\toclineR",
centers level 0 using "\toclineC",
uses the hierarchical version "\toclineX" for levels 1 and numbered 2,
though one level down,
and ignores everything else.
Inspecting how "\toclineR" is defined in file "RCstyle.tex"
can give you some hints to implement other formats.


\section{Bibliography}

To format consistently the bibliographic entries,
{\em see also} \refsc{Citing},
there are some helping macros.
\command\book{Title}: Prints properly the \book{Title}.
\command\article{Title}: Prints properly the \article{Title}.
\command\periodical{Title}: Prints properly the \periodical{Title}.
\command\ISBN: Prints the {\sc isbn}. So,
 "\ISBN 0-201-13448-9" prints \ISBN{0-201-13448-9}.
\command\DOI{doi}: Prints the {\sc doi}. For example,
 "\DOI{10.6084/m9.figshare.7928240}" will print
  \DOI{10.6084/m9.figshare.7928240}.
\command\DOIx{text}{url}: Prints ‘{\sc doi}: "text"’
 but it points to "url". It could be needed if the
 URL contains some characters that are special for {\TeX}.


\section{Blocks}

Short quotations should be “quoted”,
but long quotations are better displayed.
\command\begincitation: Starts a long quotation.
\command\endcitation: Ends a long quotation.

Line long centering can use the
"plain" macro "\centerline",
but sometimes one line is not enough.
\command\begincenter: Starts centering text.
\command\endcenter: Stops centering text.


\section{Lists}

A list has to be delimited between a pair of
"\beginpoints" and "\endpoints" commands.
Inside, each point has to start with
\beginpoints
\point "\point", which puts a bullet in the left, or
 \item{1} "\item{1}", which puts its argument, "1" in the example,
 instead of the bullet, or
 \itemitem{1.1} "\itemitem{1.1}", which is the second level
 version, or
\point[2] "\point[2]" which works like "\item{2}", but
 when its argument is longer, then
\point[Titled point] "\point[Titled point]" writes its
 argument left aligned, and it hangs subsequent lines
 one level.
\endpoints

Then, in addition to "\item{1}" and "\itemitem{1.1}",
which are defined in "plain",
these are the commands used in lists.
\command\beginpoints: Starts a list.
\command\point: Starts a bulleted item.
\command\point[Title]: Starts a named item.
\command\endpoints: Finishes the list.



\section{Verbatim}

Verbatim text can be included with \verb!\verb|verbatim|!.

\command\verb|verbatim text|: Prints "verbatim text" verbatim.

The character just following "\verb", which is "|" in this example,
is used to delimit the text, so it can be any one not happening
inside the text, and it has to be repeated to end the text.

{\catcode`\"=12
After doing \verb|\catcode`\"=13 \def"{\verb"}|,
you print \verb|\verb| just typing \verb|"\verb"|.}\par



\section{Revision marks}

Added parts should be between a pair of "\new/" and "\wen/" marks.
For \new/example\wen/, this marks that the word example was added.
It is usually marked in red, and an asterisk is put in the margin.
Deleted parts should be between a pair of "\del/" and "\led/" marks.
Here, the following words \del/for example\led/ were deleted.
\command\new/: Marks the beginning of the added text.
\command\wen/: Marks the end of the added text.
\command\del/: Marks the beginning of the deleted text.
\command\led/: Marks the end of the deleted text.

\del/Whole paragraph deleted.\led/

Of course, you can ignore these commands,
redefining them:
"\def\new/{\relax}", "\def\wen/{\relax}",
"\def\del/{\relax}" and "\def\led/{\relax}".

When using them inside titles,
we must avoid their inclusion in the auxiliary files,
and then the following code
has to be written {\em before} the command "\input RCstyle":
\verb|
   \def\doextrasave{\def\new/{}\def\wen/{}\def\del{}\def\led/{}}|



\section{Other}

Other helping macros are listed below.

\command\todayiso: Prints \todayiso.
\command\QED: Prints \QED.
\command\footnote{label}{text}: Just as in "plain",
 but with PDF knowledge.
 For example, "\footnote*{This is a footnote.}" prints
 an asterisk\footnote*{This is a footnote.} and a footnote.
\command\latin{et al.}: Prints properly some
 non-English text, as \latin{et al}.

\command\newline: Terminates a line, but not a paragraph.
\command\\: Terminates a line, but not a paragraph.
\command\needspace{length}: Breaks the page if there is
 less than "length" of vertical space in the current page;
 otherwise it does nothing.

\command\person{Name}: Prints properly the person \person{Name}.
 It can be redefined to compose a persons index, for instance.
\command\definition{concept}: Prints properly
 the definition of \definition{concept}.
 It can be redefined to compose an index, for instance.


\xsection{References}

\biblabel Casares (PT):
Ramón Casares,
\article{Problem Theory},
\DOI{10.6084/m9.figshare.4956353}.

\biblabel Knuth (1984):
Donald E. Knuth,
\book{The {\TeX}book},
Addison-Wesley Publishing Company, Reading, MA,
\ISBN 0-201-13448-9.





\bye
