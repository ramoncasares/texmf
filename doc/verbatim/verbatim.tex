% verbatim.tex (RMCG20090909)

\input doc

\newwrite\borrar
\immediate\openout\borrar=auxiliar.out
\fverb\write\borrar\ es $ es \
\fverb\immediate\write\borrar"¿Adelanta,  o no?"
\fverb\immediate\write\borrar\&Entre escapes&\
\closeout\borrar
\vfill

\rmcglayout

\title0 Verbatim

\title1 Lo que no hice

He definido varias macros para mostrar texto verbatim.
La más básica es \verb|\verb"texto"|, que aparece definida en [382].
La uso tanto que he hecho a las dobles comillas, \verb|"|, carácter
activo, y así puedo usar, como sinónimo, la forma menos intrusiva
\verb|"texto"|. Tanto es así que, muchas veces, la uso simplemente
para escribir en letra de máquina.

En temporadas pensé en usar \verb|\"texto"|, para no tener que
definir más caracteres activos. No lo hice por dos razones:
porque es más largo, pasa de dos a tres caracteres, y
porque me impide escribir la `o con diéresis' que necesita \[G\"odel].

Otra posibilidad era usar, en vez de las comillas dobles, \verb|"|,
la comilla invertida, "`", como hace markdown, por ejemplo.
Esto también tiene dos contraindicaciones. La primera es que me
parece más claro entender que \verb|"texto"| es código, a interpretar
que lo es "`texto`", aunque supongo que es una cuestión de gustos.
El otro problema es que "plain" {\TeX} usa la comilla invertida
para hacer las ligaturas que permiten escribir, con caracteres
puramente {\sc ascii}, las aperturas de explamaciones e
interrogaciones propias del castellano.
O sea, escribiendo "?`qué?" se obtiene ?`qué?,
y escribiendo "!`qué!" se obtiene !`qué!.


\title1 Lo que sí hice

Ahora comentaré sobre las ampliaciones que sí he hecho de
la macro \verb|\verb"texto"|.

Quería diseñar una macro que me permitiera escribir verbatim tanto en el
papel como en un fichero, con la siguiente sintaxis
\beginpoints
\item \verb|\verb"texto"| en el papel
\item \verb|\verb\write\file"texto"| en el fichero, o
\item \verb|\verb\immediate\write\file"texto"| en el fichero,
inmediatamente.
\endpoints

El problema es insalvable, ya que el carácter limitador, \verb|"| en los
ejemplos, también puede ser el escape "\", no hay modo de decidir si con
\verb|\verb\write\file"texto"| se quiere escribir "write" en el papel,
 o "texto" en el fichero "\file".

Se requiere, por lo tanto, una nueva macro para escribir verbatim en un
fichero, a la que denominaremos "\fverb", y que se encuentra definida en
"explain.tex", así:

\begingroup\def\codeoptions{\parindent=0pt}
\begincode
\def\fverb#1\write{\ifx#1\immediate \let\next=\iverb
 \else \let\next=\wverb \fi \next}
\def\iverb#1{\begingroup\uncatcodeall\obeyspaces\obeylines\d@iverb#1}
\def\d@iverb#1#2{\def\next##1#2{\immediate\write#1{##1}\endgroup}\next}
\def\wverb#1{\begingroup\uncatcodeall\obeyspaces\obeylines\d@wverb#1}
\def\d@wverb#1#2{\def\next##1#2{\write#1{##1}\endgroup}\next}
\endcode
\endgroup

Las macros "\iverb" y "\wverb" son también útiles, por ejemplo
para formatear el índice, haciendo:
\verb!\wverb\auxf"\iverb\tocf|material en toc|"!.

Como prueba, véase que haciendo:

\begincode
\newwrite\borrar
\immediate\openout\borrar=auxiliar.out
\fverb\write\borrar\ es $ es \
\fverb\immediate\write\borrar"¿Adelanta,  o no?"
\fverb\immediate\write\borrar\&Entre escapes&\
\closeout\borrar
\endcode

Entonces en el fichero "auxiliar.out" tenemos:
\medskip
\listing{auxiliar.out} % OJO, auxiliar.out debe estar cerrado ahora

Algunas versiones de \TeX\ son reluctantes a escribir en ficheros
caracteres no ascii imprimibles, en estos casos "¿" resulta "^^bf",
o, si la codificación es {\sc utf-8}, incluso "^^c2^^bf".

\bye

