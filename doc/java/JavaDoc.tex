%JavaDoc.TEX (RMCG20000513)

\input doc
\rmcglayout \lm12fonts \files

\catcode`\|=13 \def|{\verb|}
\def\comando{\par\hangindent20pt\noindent
 \begingroup\setupverbatim\doxverbatim}
{\obeyspaces
\gdef\doxverbatim{\def\next##1: {##1\endgroup\quad\ignorespaces}\next}}
\def\modo#1{{\sc#1}}


\title0 Java

Este documento explica el uso y las posibilidades que proporciona
|java.tex|.  Como primera aproximación, puede decirse que sirve para
facilitar la documentación de los programas (ficheros) escritos en el
lenguaje Java.


\title1 Estructura

Cuando se quieren utilizar las herramientas facilitadas por |java.tex|
se escribe, hacia el comienzo del documento, el comando
 |\input java.tex|.
El fichero |java.tex| lee el fichero |doc.tex| que, a su vez, llama a
los ficheros |explain.tex|, |fonts.tex| e |index.tex|, de manera que
puede hacer uso de todos los comandos (o macros) de: |plain|, |spplain|,
|doc|, |explain|, |fonts| e |index|, además de las definidas en el
propio fichero |java|. En este documento sólo se explican las macros
definidas en |java.tex|.


\title1 Uso

Para que un documento sea procesado por |java.tex| deben ejecutarse los
siguientes pasos:

\bulletedpar Incluir en el fichero que precise de las definiciones de
|java.tex| la orden \hfill\break |\input java.tex|.

\bulletedpar Ejecutar {\TeX} con el formato plain en castellano.
Vale \hfill\break |TEX &SPPLAIN filename.tex|.


\title1 Código Java

El lenguaje Java impone cómo deben ser los contenidos de los ficheros
que contienen código. En términos generales, un fichero debe contener
una clase, o más, o una interface, habiendo de coincidir el nombre del
fichero con el de la clase principal o interface.

Por otra parte, el objetivo de este sistema de documentación es que el
mismo fichero que puede ser compilado, por |javac|, pueda ser tratado,
sin ninguna modificación, por \TeX. Además, como objetivo adicional, el
mismo fichero de código Java podrá ser procesado, también sin modificar,
por la herramienta de documentación propia de Java, |javadoc|.

Estos objetivos imponen algunas condiciones. Una es que {\TeX} sepa
cuando está leyendo un fichero Java. Para ello se utiliza el comando:

\comando\javainput"filename.java": hace que {\TeX} lea el fichero Java
indicado. Los ficheros Java se consideran títulos del nivel 2, porque se
supone que todos ellos aparecen dentro de la sección de ``Ficheros
Java'', que es un título de nivel 1. Véase, también, en los comandos
generales.

\title1 Modos

Dentro del fichero Java, {\TeX} se encontrará en uno de los dos modos
siguientes: \modo{código} o \modo{comentario}. Al comenzar el fichero se
encuentra en modo \modo{código}, que abandonará en cuanto se encuentre
la secuencia |/**| que marca el comienzo del modo \modo{comentario}.
Nótese que sólo atiende a los comentarios de documentación, tratando a
los comentarios ordinarios, que comienzan por |/*|, como código. Estando
en modo \modo{comentario}, retornará al modo \modo{código} en cuanto se
encuentre la secuencia |*/|.


\title2 Modo código

En modo código, |java.tex| tipografía {\it verbatim}, pero añadiendo a
la izquierda el número de la línea de código. Para ello cambia los
|\catcode|s de todos los caracteres a |12| (otros), excepto los del
carácter |/|, que pasa a |13| (activo), para poder reconocer la
secuencia |/**| de comienzo del modo \modo{comentario}.

\comando /: Que usa el |\catcode`\/=13|, de activo, para buscar los
comienzos (|/**|) de los comentarios de documentación.


\title2 Modo comentario

En modo comentario, |java.tex| tiene toda la potencia tipográfica de
{\TeX}, recoge información para realizar automáticamente los índices, y
acepta los comandos de documentación de Java. Revisemos todo esto por
apartados.

\title3 Caracteres especiales

El modo \modo{comentario} usa tres caracteres especiales. Las razones de
cambiar los |\catcode|s son muy diferentes. Para poder percatarse de la
finalización del modo \modo{comentario}, cambia el |\catcode| del
carácter |*| a |13| (activo). Para poder tratar los cometarios de
documentación de Java, cambia el |\catcode`\@=0|, esto es a escape. Y,
para facilitar la escritura {\it verbatim}, usa el carácter
{\tt\char34} como activo.

\comando *: Que usa el |\catcode`\*=13|, de activo, para buscar los
finales (|*/|) de los comentarios de documentación.

\comando @: Que usa el |\catcode`\@=0|, de escape, en modo
\modo{comentario} porque la documentación automática de Java, |javadoc|,
lo usa como escape.

\comando ": Que usa el |\catcode`\"=13|, de activo,
para facilitar la escritura {\it verbatim}. Véase, también,
en los comandos generales.

\title3 Comandos de indexación

Estos comandos señalan cómo titular las diversas partes del código y
toman su información para los índices y referencias.

\comando\class"Nombre": Indica que comienza la definición de una clase Java.
\comando\interface"Nombre": Indica que comienza una interface.
\comando\constructor"Nombre(Args)": Indica que comienza la definición
de un constructor de la clase.
\comando\variable"Nombre": Indica que comienza la definición de una
variable.
\comando\method"Nombre(Args)": Indica que comienza la definición de
un método.


\title3 Comandos de documentación Java

Cada comando de documentación de Java define un párrafo. Acepta todos
los comandos de documentación de Java:

\comando@author: Que sirve para señalar al autor del fichero
\comando@version: Que sirve para indicar la versión del fichero
\comando@return: Que sirve para explicar qué es lo que el método
devuelve como resultado.
\comando@param nombre: Que sirve para explicar el uso del parámetro
|nombre| del método en cuestión.
\comando@exception nombre: Que permite explicar las circunstancias
en las que el método devuelve la referida excepción.
\comando@deprecated: Que indica, cuando se usa, que no se recomienda el
uso del método.
\comando@see clase#nombre: que es como |\See|, véase a continuación,
pero define un párrafo y usa una sintaxis diferente.


\title1 Comandos generales

Estos son comandos que se pueden utilizar fuera de los ficheros Java, en
los que incluyen el comando |\input java.tex|. Son los siguientes:

\comando\javainput"filename.java": Que ya hemos visto, y que lee el
fichero Java de nombre |filename.java|.

\comando\javaindex: Que crea un índice alfabético con los elementos
recogidos por los comandos de indexación.

\comando\See"clase#nombre": Que permite referirse a los elementos
recogidos por los comandos de indexación. Debe especificarse la clase en
la que el elemento aparece definido, ya que de otra forma la referencia
podría ser ambigua. Para referirse a un fichero, en vez del nombre de la
clase, han de escribirse literalmente las cuatro letras |file| antes del
cuadrado y el nombre del fichero,
 por ejemplo, |\See"file#filename.ext"|.
También se utiliza dentro del fichero Java, en el modo \modo{comentario}.

\comando"texto": Que facilita la escritura {\it verbatim} del |texto|
escrito entre las comillas. Para ello, el carácter |"| cambia % "
al |\catcode| |13| (activo). También se utiliza dentro del fichero Java,
en el modo \modo{comentario}.

\comando\URI Java (http://www.sun.com/): Que hace un enlace a una
dirección de internet. En el caso de este ejemplo, aparecería escrito
``Java'' y pulsando sobre tal palabra en el documento |pdf| se iría a la
dirección |http| de Sun. Puede ser |http|, |mailto|, |ftp|, etc. También se
utiliza dentro del fichero Java, en el modo \modo{comentario}.


\title9 Índice

\def\toclinezero#1#2#3#4{\tocln{#1}{#2}{}{\bf#1}{#2}{#4}}
\def\toclineone#1#2#3#4{\def\3{#3}\ifx\empty\3
 \tocln{#1}{#2}{}{#1}{#2}{#4}\else\tocln{#3 #1}{#2}{#3\quad}{#1}{#2}{#4}\fi}
\def\toclinetwo#1#2#3#4{\tocln{#3 #1}{#2}{\quad #3\quad}{#1}{#2}{#4}}
\def\toclinethree#1#2#3#4{\tocln{#3 #1}{#2}{\qquad #3\quad}{#1}{#2}{#4}}
\ifauxf\input auxiliar.toc\fi

\bye
