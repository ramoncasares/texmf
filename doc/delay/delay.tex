% delay.tex (RMCG20090909)

\input doc
\cm12fonts \rmcglayout \files \colors
\let\codeoptions=\numbercodeoptions

\font\xiilogo=logo10 scaled1200
\def\METAFONT{\hbox{\xiilogo META}\-\hbox{\xiilogo FONT}}
\def\METATeX{\hbox{\xiilogo META}\-\TeX}

\title0 Delay

\title1 General

El fichero "delay.mf" contiene macros que facilitan el dibujo de diagramas
usando \METAFONT.  También está preparado para que pueda ser utilizado con
\METATeX.

\title1 Uso

Una vez importado, haciendo "input delay", ya se pueden utilizar sus
definiciones, aunque no se conseguirá imprimir ni un punto a no ser que se
defina previamente un "pen".   Tres "pen"s están predefinidos, por lo que
basta hacer "thinpen;", "medpen;" o "thickpen;" para que sean utilizados.

Cada objeto utiliza un "suffix".  Así que todos los puntos
interesantes de ese objeto tienen como base dicho "suffix".

Hay dos grandes tipos de figuras, las superficies y las líneas.  Las
superficies definen las siguientes cantidades:  su punto central
("z\s"), su extremo superior ("y\s\t"), su extremo
inferior ("y\s\b"), su extremo izquierdo ("x\s\l") y su
extremo derecho ("x\s\r").  Las líneas quedan definidas, por lo
menos, por los puntos de origen ("z\s\o") y de destino
("z\s\d"), y pueden tener definidos otros intermedios,
"z\s\m" si es uno, o "z\s\m1", "z\s\m2", etc., si son
más.

Al estar hechas todas las definiciones sobre sufijos, se pueden
calcular unos puntos en función de otros.  Así que basta
proporcionar dos dimensiones horizontales y dos verticales para
ubicar un objeto superficial, quedando definidas ocho dimensiones.
Por ejemplo, haciendo "rectangle(2)(20pt,30pt)" y si
"z[2] = (40pt,40pt)" tenemos definidas las dimensiones
horizontales "x[2]" y "width" y las verticales "y[2]"
y "height" y, a partir de éstas, se calculan "x[2].l",
"x[2].r", "y[2].b", "y[2].t".  Pero también pueden
definirse otras, quedando en cualquier caso las ocho definidas.

Dos objetos no pueden utilizar el mismo "suffix", puesto que si
alguna de las dimensiones citadas no coincide en ambas se obtiene
una ecuación ``off by x pt''.  Como es perfectamente posible que dos
objetos utilicen el mismo punto, basta hacer, por ejemplo:
"z\3.lbl = z\3", para que el centro de la etiqueta "3.lbl"
se ubique en el centro mismo del objeto "3".

Como al hacer un "draw" deben estar las ocho magnitudes
diferentes del objeto superficial definidas, "delay" retrasa
estas órdenes hasta el final, lo que permite sacar el máximo
provecho de las execelentes capacidades de {\METAFONT} para resolver
ecuaciones, lo que, a su vez, permite hacer una definición funcional
de las figuras.

Para que una figura sea fácilmente modificable y reutilizable, debe
utilizarse una definición funcional.  Así, por ejemplo, las
relaciones entre objetos deben hacerse si tales relaciones existen
funcionalmente, y no porque exista, únicamente, una relación
gráfica.  Esto es, si dos objetos caen a la misma altura, pero ello
se debe a que ambos se ubican encima de dos objetos de iguales
dimensiones que están a la misma altura, las relaciones deben ser
con respecto a aquellos objetos sobre los que descansan, ya que si
éstos se modifican, los dos de partida dejarán de estar a la misma
altura.

Por defecto, "font_size", ("designsize") vale
"128pt".  Nótese que cada una de las dimensiones de la font ha
de ser como máximo 16 veces tal valor, y que la precisión en el
fichero "tfm" es peor que $2^{-20}$ ese valor.  Con
"font_size 16pt#;" o menor, se consigue una precisión de
"1sp", que es la máxima con la que trabaja \TeX.

Si se usa con "metatex.tex", entonces puede utilizarse un valor
de "font_size" pequeño, porque las modificaciones que hace
{\METAFONT} a las dimensiones de los caracteres, son ignoradas por
{\TeX}, que asume las pedidas a {\METAFONT} merced a la macro
"\MTbeginchar".

Si se usa con "metatex.tex", entonces también se retrasa el
borrado bajo las etiquetas.


\title1 Órdenes

Las órdenes disponibles son:
\begincode
thinpen; % 0.4pt
medpen; % 0.8pt
thickpen; % 1.2pt
rectangle(suffix s)(expr width,height); % t,b,l,r
square(suffix s)(expr side); % t,b,l,r
ellipse(suffix s)(expr width,height); % t,b,l,r
circle(suffix s)(expr diameter); % t,b,l,r
oval(suffix s)(expr width,height,superness); % t,b,l,r
triangle(suffix s)(expr width,height); % t,b,l,r
lozenge(suffix s)(expr width,height); % t,b,l,r
dot(suffix s); % dot_diameter
line(suffix s); % o,d
dottedline(suffix s); % o,d; dot_diameter, dash_length
dashedline(suffix s); % o,d; dash_length
arrowhead(suffix s); % o,d; arrow_head_length, arrow_head_width
arrow(suffix s); % o,d; arrow_head_length, arrow_head_width
cillinder(suffix s)(expr width,height); % t,b,l,r
box(suffix s)(expr width,height); % t,b,l,r; join_radius
soft(suffix s); % o,m,d; join_radius
softt(suffix s); % o,m1,m2,d; join_radius
fork(suffix s); % o,m,d; dot_diameter, arrow_head_length, arrow_head_width
darrow(suffix s); % o,d; arrow_head_length, arrow_head_width
arroww(suffix s); % o,m,d; arrow_head_length, arrow_head_width; join_radius
arrowww(suffix s); % o,m1,m2,d; arrow_head_length, arrow_head_width; join_radius
\endcode

Los parámetros disponibles, con sus valores, son:
\begincode
dot_diameter := 2.4pt;
dash_length := 4pt;
arrow_head_length := 6pt;
arrow_head_width := 2.4pt;
join_radius := 5pt;
\endcode

\bye
