% WATERMARK.TeX (20021227)

\input doc
\rmcglayout

\newtoks\oldheadline \oldheadline=\headline
\headline={\watermark\the\oldheadline }
\def\watermark{%
 \pdfliteral{q 0.9 G 0.9 g 169 -32 m}% % ®
 \pdfliteral{82 -97 109 -111 110 -32 c}% Ramon
 \pdfliteral{67 -97 115 -97 114 -101 c}% Casare
 \pdfliteral{115 -32 50 -48 48 -51 c}% % s 2003
 \pdfliteral{b Q}%
 }


\title0 Watermark en {pdf\TeX}

Una marca de agua, o {\latin watermark} en inglés, es una marca
imperceptible pero que puede servir para mostrar el origen del
documento.

La marca de agua que aquí explico queda en el archivo "pdf" generado
por pdf{\TeX}. Usa las siguientes macros:
\begincode
\newtoks\oldheadline \oldheadline=\headline
\headline={\watermark\the\oldheadline }
\def\watermark{%
 \pdfliteral{q 1 G 1 g 169 -32 m}% % ®
 \pdfliteral{82 -97 109 -111 110 -32 c}% Ramon
 \pdfliteral{67 -97 115 -97 114 -101 c}% Casare
 \pdfliteral{115 -32 50 -48 48 -51 c}% % s 2003
 \pdfliteral{b Q}%
 }
\endcode

Para que, en este fichero explicativo, las marcas puedan verse, en vez
de utilizar el color blanco, se ha utilizado un gris claro, esto es, en
vez de poner "1 G 1 g" hemos puesto "0.9 G 0.9 g".

Tanto PostScript como Acrobat usan tintas opacas. Por esta razón los
valores de las componentes $y$ de las coordenadas $(x,y)$ son negativos.
De este modo, la marca se hace sobre la parte de la hoja todavía no
impresa, esto es, a la derecha ($x$ positiva) y abajo ($y$ negativa).
Conviene poner la marca en la cabecera, para que aparezca en cada hoja,
y dentro del área de impresión.

En general, es decir, siempre que no se use el comando
"\pdfcompresslevel=0", la marca de agua quedará en un "stream"
filtrado con "/Filter /FlateDecode" por lo que será imperceptible,
también, en el fichero "pdf" generado.

Para recuperar la marca es necesario invertir el filtrado. Una manera
consiste en utilizar la conversión de GhostScript a pdfwrite. Debe
usarse una resolución de 72 ppi, para que mantenga como unidad el punto
PostScript, y después efectuar una traslación. Por ejemplo, si aparece
"240.862 729.642 m" entonces es que la traslación aplicada es de
$(-71.862,-761.642)$, porque $240.862-71.862=169$ y
$729.642-761.642=-32$. Aplicando esta traslación a todas las coordenadas
de la marca se recomponen los valores originales.


\bye
