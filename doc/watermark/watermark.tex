% WATERMARK.TeX (20021227)

\input doc
\rmcglayout \lm12fonts

\newtoks\oldheadline \oldheadline=\headline
\headline={\watermark\the\oldheadline }
\def\watermark{\ifx\pdfliteral\undefined
  \special{pdf: bop q 0.9 G 0.9 g 169 32 m
    82 97 109 111 110 32 c
    67 97 115 97 114 101 c
    115 32 50 48 49 53 c b Q}%
 \else%
  \pdfliteral{q 0.9 G 0.9 g 169 32 m}% ©
  \pdfliteral{82 97 109 111 110 32 c}% Ramon
  \pdfliteral{67 97 115 97 114 101 c}% Casare
  \pdfliteral{115 32 50 48 49 53 c}% % s 2015
  \pdfliteral{b Q}%
 \fi}


\title0 Watermark en {pdf\TeX} y dvipdfm

Una marca de agua, o {\latin watermark} en inglés, es una marca
imperceptible pero que puede servir para mostrar el origen del
documento.

La marca de agua que aquí explico queda en el archivo "pdf" generado
por pdf{\TeX} o dvipdfm. Usa las siguientes macros:
\begincode
\newtoks\oldheadline \oldheadline=\headline
\headline={\watermark\the\oldheadline }
\def\watermark{\ifx\pdfliteral\undefined
  \special{pdf: bop q 0.9 G 0.9 g 169 32 m
    82 97 109 111 110 32 c
    67 97 115 97 114 101 c
    115 32 50 48 49 53 c b Q}%
 \else%
  \pdfliteral{q 0.9 G 0.9 g 169 32 m}% ©
  \pdfliteral{82 97 109 111 110 32 c}% Ramon
  \pdfliteral{67 97 115 97 114 101 c}% Casare
  \pdfliteral{115 32 50 48 49 53 c}% % s 2015
  \pdfliteral{b Q}%
 \fi}
\endcode

Para que, en este fichero explicativo, las marcas puedan verse, en vez
de utilizar el color blanco, se ha utilizado un gris claro, esto es, en
vez de poner "1 G 1 g" hemos puesto "0.9 G 0.9 g".

Tanto PostScript como Acrobat usan tintas opacas. Por esta razón
conviene poner la marca en la cabecera, para que la marca se haga
antes de escribir el resto del texto, y no lo borre.

En general, es decir, siempre que no se use el comando
"\pdfcompresslevel=0", la marca de agua quedará en un "stream"
filtrado con "/Filter /FlateDecode" por lo que será imperceptible,
también, en el fichero "pdf" generado.

Para recuperar la marca es necesario invertir el filtrado. Una manera
consiste en utilizar la conversión de GhostScript a pdfwrite. Debe
usarse una resolución de 72 ppi, para que mantenga como unidad el punto
PostScript, y después efectuar una traslación. Por ejemplo, si aparece
"240.862 729.642 m" entonces es que la traslación aplicada es de
$(-71.862,-697.642)$, porque $240.862-71.862=169$ y
$729.642-697.642=32$. Aplicando esta traslación a todas las coordenadas
de la marca se recomponen los valores originales.


\bye
