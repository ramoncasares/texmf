% MET@TEX.TEX (RMCG19950831)

\ifx\pdfoutput\undefined \csname newcount\endcsname\pdfoutput \fi
\ifcase\pdfoutput \def\pdfcode#1\pdfendcode{\relax}\else
                  \let\pdfcode=\relax \let\pdfendcode=\relax \fi

% twelvept
\catcode`\@=11
 \font\twentyr=cmr17 scaled1200
 \font\seventeenbx=cmbx12 scaled1440
 \font\fourteenbx=cmbx12 scaled1200
 \font\twelverm=cmr12 \font\ninerm=cmr9 \font\sixrm=cmr6
 \font\twelvei=cmmi12 \font\ninei=cmmi9 \font\sixi=cmmi6
 \font\twelvesy=cmsy10 scaled 1200
 \font\ninesy=cmsy9 \font\sixsy=cmsy6
 \font\twelveex=cmex10 scaled 1200
 \font\twelveit=cmti12
 \font\twelvesl=cmsl12
 \font\twelvebf=cmbx12 \font\ninebf=cmbx9 \font\sixbf=cmbx6
 \font\twelvecsc=cmcsc10 scaled 1200
 \font\twelvess=cmss12 \font\niness=cmss9
 \font\twelvett=cmtt12
 \def\twelvebig#1{{\hbox{$\left#1\vbox to10.2pt{}\right.\n@space$}}}
 \def\twelveBig#1{{\hbox{$\left#1\vbox to13.8pt{}\right.\n@space$}}}
 \def\twelvebigg#1{{\hbox{$\left#1\vbox to17.4pt{}\right.\n@space$}}}
 \def\twelveBigg#1{{\hbox{$\left#1\vbox to21pt{}\right.\n@space$}}}
 \def\twelvept{\def\rm{\fam0\twelverm}\def\oldstyle{\fam1\twelvei}%
  \textfont0=\twelverm \scriptfont0=\ninerm \scriptscriptfont0=\sixrm
  \textfont1=\twelvei \scriptfont1=\ninei \scriptscriptfont1=\sixi
  \textfont2=\twelvesy \scriptfont2=\ninesy \scriptscriptfont2=\sixsy
  \textfont3=\twelveex \scriptfont3=\twelveex \scriptscriptfont3=\twelveex
  \def\it{\fam\itfam\twelveit}\textfont\itfam=\twelveit
  \def\sl{\fam\slfam\twelvesl}\textfont\slfam=\twelvesl
  \def\bf{\fam\bffam\twelvebf}\textfont\bffam=\twelvebf
  \scriptfont\bffam=\ninebf \scriptscriptfont\bffam=\sixbf
  \def\tt{\fam\ttfam\twelvett}\textfont\ttfam=\twelvett
  \def\sf{\twelvess\ordinals\niness}\let\sc=\twelvecsc
  \setbox\strutbox=\hbox{\vrule height10.2pt depth4.2pt width\z@}%
  \let\big=\twelvebig \let\Big=\twelveBig
  \let\bigg=\twelvebigg \let\Bigg=\twelveBigg
  \let\fontzero=\twentyr
  \let\fontone=\seventeenbx
  \let\fonttwo=\fourteenbx
  \normalbaselineskip=14.4pt\normalbaselines\rm}
\catcode`\@=12
\twelvept

\input metatex

\hsize=15.92truecm \vsize=24.62truecm % DIN A4
%\hsize=6.5truein \vsize=9truein % letter

%\parskip=\smallskipamount

\font\logo=logo10 at 12pt %at 12pt
\font\biglogo=logo10 at 20.736truept
\font\bigrm=cmr10 at 20.736truept
\font\sf=cmss10
\font\smallsf=cmssq8

\def\MTeX{\leavevmode\hbox{\logo META}\-\TeX\spacefactor=1000 }
\def\MF{\leavevmode\hbox{\logo META}\-\hbox{\logo FONT}\spacefactor=1000 }
\def\MP{\leavevmode\hbox{\logo META}\-\hbox{\logo POST}\spacefactor=1000 }

\def\pass#1{\smallskip \hang\indent
 \llap{\hbox to\parindent{\bf #1)\hfil}}\ignorespaces}

\def\setupverbatim{\tt \def\par{\leavevmode\endgraf}\catcode`\`=\active
 \obeylines\uncatcodespecials\obeyspaces\verbatimoptions}
\def\uncatcodespecials{\def\do##1{\catcode`##1=12 }\dospecials}
{\obeyspaces\global\let =\ }
{\catcode`\`=\active \gdef`{\relax\lq}}

% For line numbering do:
\def\listverbatimoptions{\tentt \parskip=0pt
 \baselineskip=12pt plus 0.1pt minus 0.1pt
 \parindent=20pt \count255=0
 \everypar{\advance\count255 1 \llap{\sevenrm\the\count255\quad}}}
\def\lineverbatimoptions{}

\def\verb{\begingroup\let\verbatimoptions\lineverbatimoptions
 \setupverbatim\doverbatim}
\def\doverbatim#1{\def\next##1#1{##1\endgroup}\next}

\def\macro{\par\hang\noindent\begingroup
 \let\verbatimoptions\lineverbatimoptions\setupverbatim\doxverbatim}
\def\doxverbatim{\def\next##1={##1\endgroup\unskip\unskip\quad}\next}

\def\listing#1{\par\begingroup \let\verbatimoptions\listverbatimoptions
 \setupverbatim \input#1 \endgroup}

\def\beginlisting{\smallskip\hrule\smallskip}
\def\endlisting{\smallskip\hrule\smallskip}

\catcode`\|=13 \def|{\verb|}

% \tofile{<filename>}
%    writes verbatim to <filename> till the next . line

\newwrite\auxfile \def\mark{.}
\def\tofile#1{\immediate\openout\auxfile=#1 \begingroup
 \uncatcodespecials\obeyspaces\obeylines\copyline}
{\obeylines\gdef\copyline#1
 {\def\next{#1}\ifx\next\mark
   \def\next{\endgroup\immediate\closeout\auxfile}\else %
  \immediate\write\auxfile{\next}\let\next\copyline\fi\next}}

%%%%%

\pdfcode\begingroup
 \deactivate\dohigh
 \pdfinfo{
   /Author (Ramón Casares)
   /Title (METATeX)
   /Subject (METAFONT graphics in TeX docs)
   /Keywords (TeX METAFONT)}
\endgroup\pdfendcode

%%%%%%%%%%%%%%%%%%%%%%%%%%%%%%%%%%%%%%%%%%%%%%%%%%%%%%%%%%%%%%%%%%%%%%

\vglue1pc
\centerline{\biglogo META\bigrm\TeX}
\bigskip

\TeX\ and \MF\ define the page layout to the pixel.  This means that
nothing more is needed, not even some means to include figures in a
\TeX\ document.  To prove this is the aim of this paper.

To split the typesetting process in two parts, one to define and draw the
characters and the other to arrange the characters in paragraphs and
pages, is surely the best way to reduce the typesetting task
complexity provided it needs simplification.  But it makes
difficult, for example, to integrate labels with graphics in
figures, because while \TeX\ is best suited to typeset the labels,
\MF\ is the appropriate tool to draw the graphics.  And, of course,
labels should be located in accordance to graphics.  Therefore the
true successor of \TeX\ has to include in a single program both the
capabilities of \TeX\ and \MF.  Then the typesetting engine would
include a powerful graphic tool, a grid to typeset in if required,
and it could take into account the shapes of the characters to
determine, for example, kernings or italic corrections.  Even the
other way around is possible.  It could be seen as a graphic engine
with a powerful typesetting tool.  From this point of view the page
would be a graphic object that could contain paragraphs of different
shapes requested to the typesetting tool.

\MTeX, although does not fulfill the requirements of such a successor,
it can be used as an early announcer of its possibilities.  For the
moment, \MTeX\ takes advantage of \MF\ equation solving capabilities
that are very useful to locate objects, including the labels to be
typeset by \TeX\null.  The cost of such a nice feature is that two
\TeX\ passes are required.

During the first \TeX\ pass a \MF\ file is written.  As it is \TeX\
itself who writes the \MF\ file, any dimension controlled by \TeX\
can be used and incorporated to \MF\ calculations.  For example, the
labels sizes as they will be typeset by \TeX\null, are made known to
\MF.

In between the two \TeX\ passes, \MF\ draws the figures graphics
and writes in its log file the labels locations.  So it is \MF\
responsibility to locate the labels.  Note that, depending on the
\MF\ style of programming, this can be completely determined from
\TeX\null.  In other words, you can relate the label location to the
location and size of other \MF\ objects, or not.

So when \TeX\ executes its second pass, it has to take the graphics
from the new font and it has to read the location of labels from the
\MF\ log file, but by then everything is complete.

Note that because labels are just |\hbox|es typed by \TeX, every macro
currently defined for text automatically applies also to figures. For
example, if a macro |\person| is defined to write its argument in a
small caps font and save it to an index file, the same happens whenever
it is used inside a figure label.

\vfill\break

%\vglue1pc
\centerline{\biglogo META\bigrm\TeX\ Manual}
\bigskip

{\MTeX} allows to include in the source file, in addition to the
customary {\TeX} macros to control the text, other commands to
generate {\MF} figures.

The following steps are in order to use {\MTeX}:

\pass1 The {\MTeX} file, suppose it is |filename.ext|, needs a first
processing by the |TEX| program, with format |plain|, during which a
{\MF} file named |auxiliar.mf| is created.  This {\MF} file contains
information provided by {\TeX} concerning the size of the labels, so
the |MF| program can delete this area from the figure if it is
requested.  If the file |filename.dvi| were typeset now, it would be
the same as the final document but with blanks instead of figures.

\pass2 Then program |MF|, with format |plain|, should be run on file
|auxiliar.mf|.  As a result, information explaining where to typeset
the labels is written in the log file.  In addition to the log file,
|auxiliar.log|, and the tfm file, |auxiliar.tfm|, also the font in
the generic format |auxiliar.329gf| is created.  In my system I have
to process this gf file to get a pk file that my drivers can read,
so I execute the program |GFtoPK| on it, getting the packet formatted
font |auxiliar.329pk|.  \underbar{Note 1}:  The number |329|
referring to the resolution, can be different.  \underbar{Note 2}:
The tfm and pk files should be in directories where programs can
find them.  \underbar{Note 3}:  {\MTeX} sets {\MF} mode to
|localfont|, assuming that |localfont| is assigned the mode of the
local printer.

\pass3 Lastly, file |filename.ext| should be run again by the |TEX|
program.  During this second run both the font |auxiliar| containing
the figures and information explaining where to locate the labels
are available, so the document is complete.

The figures fill exactly the same area in the first than in the
second |TEX| program run, so references, indices and tables of
contents, that need also two passes to be right, can take advantage
of the two runs needed by {\MTeX}.

\bigskip

To use {\MTeX} macros they should be imported writing in the
source file:

\indent|\input metatex|

This has to be written after |\mag| has taken its final value.  When
|metatex.tex| is imported, {\MTeX} checks if file |auxiliar.mf|
exists.  If it does not exist, then everything is set for the first
pass, and for example, file |auxiliar.mf| is opened for writing.  If
it exists, then things are set for the second pass, and for example
file |auxiliar.log| is opened for reading.  This means that if file
|auxiliar.mf| is not deleted, then step {\bf 3}, the second |TEX|
program pass, is executed directly.  This saves time when only the
text in file |filename.ext|, but not the figures, were modified.

{\MTeX} user macros are:

\macro\MTbeginchar(wd,ht,dp);= stating that a figure sized as shown,
width |wd|, height |ht| and depth |dp|, will be created.  These
values should be known both by {\TeX} and by {\MF}, so for example
|12pt|, |6cm|, |\the\hsize| or |\the\dimen0|, always without |#|,
are all right.  During the {\it first pass\/} it writes in file
|auxiliar.mf| the {\MF} macro |beginchar| using as character code
the next value, and box |\MTbox| is made empty but sized
as specified by the arguments of this macro.  During the {\it second
pass\/} it put the next character of font |auxiliar| in
box |\MTbox|.  Box |\MTbox| size is that specified and it is not
affected by the character dimensions.

\macro\MTendchar;= finishing the figure definition.  During the {\it first
pass\/} it writes the {\MF} macro |endchar;| in file |auxiliar.mf|.  During
the {\it second pass}, at this point box |\MTbox| contains the complete
figure, including labels.  Note that something like |\box\MTbox| is
required to typeset the figure.

\macro\MTlabel*(s)cc"Text";= adding a label to the current figure.
The parameter between quotes, |Text| in the example, is the label
content; it will be put inside an |\hbox| and therefore could be
anything that {\TeX} allows inside an |\hbox|.  The label will be
located in {\MF} point |z.s|, being |s| the parameter between
parenthesis.  The reference point is specified by the optional
parameter between the point and the label, |cc| in the example, as
follows:  it is composed of two letters, the first one could be |t|
meaning top, |c| meaning center, |g| meaning ground, or |b| meaning
bottom, and the second letter could be |l| meaning left, |c| meaning
center or |r| meaning right, so for example |tl| means that the
label reference point is its top left corner.  Reference point
default value is |cc|, that is, its center.
The asterisk preceding the point is also optional, but if it is used
then it deletes the area of the figure already drawn that it is under
the label.  This macro should be used between macro |\MTbeginchar|
and macro |\MTendchar|.  During the {\it first pass\/} it writes in
file |auxiliar.mf| the {\MF} macros by which the |MF| program writes
in its log file, |auxiliar.log|, the label reference point location,
calculates the four label sides returning them as |y.s.t| the top
side, |y.s.b| the bottom side, |x.s.l| the left side and |x.s.r| the
right side and deletes, if it is required, the figure area already
drawn that it is under the rectangle occupied by the label.  During
the {\it second pass\/} it adds the label to the box |\MTbox| in the
place that reads from file |auxiliar.log| making no modification to
box |\MTbox| dimensions, even if the label is typeset outside the
box.

\macro\MPlabel*(s)cc"Text";= works as |\MTlabel|, but it uses {\MP}
labels, so while the printed result is (nearly) the same,
the labels are included in the figures and can be transformed,
but |g| as first letter does not work.

\smallskip

There are three ways, |\MT:|, |\MTcode| and |\MTline|, for passing
information to \MF, that is, to write general text in
file |auxiliar.mf|.  This happens only during the first pass, during
the second pass these macros make nothing.

\macro\MT:= writes in file |auxiliar.mf| everything till the end of
line.  It writes verbatim except character |\| that keeps its
|\catcode| equal to |0|.  Spaces are not ignored after macros.  The
sequence |\\| writes a single |\| in file |auxiliar.mf|.

\macro\MTcode= writes in file |auxiliar.mf| everything till it finds
a line equal, including |\catcode|s, to the current value of
|\MTendmark|, by default till it finds a blank line, that is,
|\def\MTendmark{}|.  It writes verbatim except character |\| that
keeps its |\catcode| equal to |0|.  The sequence |\\| writes a
single |\| in file |auxiliar.mf|.

\macro\MTline{text}= writes in file |auxiliar.mf| the parameter
value, |text| in the example.  It does not change the |\catcode|s,
so it does not perform verbatim writing.  But all |plain| special
characters can be written prefixing them by an escape character |\|.
The |plain| special characters are (not including the first colon
nor the final point):  |\{}$&#^_~%|.  For example, |\#| will result
in |#|.

To define {\TeX} macros that write in file |auxiliar.mf|, |\MTline|
should be used instead of |\MT:| or |\MTcode| because these two use
the end of line in a special way that is not usually available when
{\TeX} is reading a macro.  {\TeX} dimensions can be included using
any of the three writing macros, and for example, |\the\hsize| will
be expanded to {\tt \the\hsize} and written to {\MF} file
|auxiliar.mf|.  Note that character |\| keeps its escape |\catcode|
in all three writing macros.  In case of macro |\MTline| also
braces, |{}|, keep their |\catcode|s and therefore macros with
parameters can be used normally.


\vfill\break

\vglue1pc
\centerline{\biglogo META\bigrm\TeX\ Examples}
\bigskip

\beginsection Diagram

First a typical example of \MTeX\ usage, showing the \TeX\ file formats and
programs and its relations.  The figure width is exactly |\hsize|, but what
is more important is that the same code will adapt itself to any value of
the line size.  In fact, of course, not to any width but to any width
between, lets say, $\rm8\,cm$ and $\rm25\,cm$.

\MTline{input delay.mf;}

%%%%%%%%%%%%%%%%%%%%%%%%%%%%%%%%%%%%%%%%%%%%%%%%%%%%%%%%%%%%%%%%%%%%%%
\tofile{example1.tex}% EXAMPLE1.TEX

\dimen0=\hsize \divide\dimen0 by 10
\dimen2=\dimen0 \divide\dimen2 by 3 \multiply\dimen2 by 4

\MTbeginchar(\the\dimen0,\the\dimen2,0pt);
\MT: thickpen;
\MT: z1 = (0,0); z2 = (0,h); z3 = (w,h); z4 = (w,0);
\MT: z5 = (1/3w,1/8h); z6 = (1/2w,0); z7 = (2/3w,-1/16h);
\MT: draw z1 -- z2 -- z3 -- z4;
\MT: draw z1 .. z5 ..z6 .. z7 .. z4;
\MT: thinpen;
\MT: for i = 11 upto 20: y[i] = 1/8h + 7/80h*(i-10); endfor
\MT: for i = 12,13,15,16,17,19,20:
\MT:  draw (1/8w,y[i]) -- (7/8w,y[i]);
\MT: endfor;
\MT: for i = 11,14,18:
\MT:  draw (1/4w,y[i]) -- (7/8w,y[i]);
\MT: endfor;
\MTendchar;

\setbox0=\box\MTbox
\dimen0=\hsize %\multiply\dimen0 by 2 \divide\dimen0 by 3
\divide\dimen0 by 2

\ifdim\hsize<10cm \let\figfont=\smallsf \def\driver{drv}
 \else \let\figfont=\sf \def\driver{Driver}\fi

\MTbeginchar(\the\hsize,\the\dimen0,12pt);
\MT: save u, v; u = w/10; v = h/10;
\MT: thickpen;
\MT: x1l = x2l = x4l = x7l = 2u;
\MT: y1t = h - v; z1lbl = z1;
\MT: rectangle(1)(u,v);
\MTlabel(1lbl)"\figfont\TeX";
\MT: y2 = y3 = y1 - 2v; z2lbl = z2; z3lbl = z3; x3 = x2 + 2u;
\MT: ellipse(2)(u,v); rectangle(3)(u,v);
\MTlabel(2lbl)"\figfont\TeX"; \MTlabel(3lbl)"\figfont TFM";
\MT: y4 = y5 = y6 = y2 - 2v; x5 = x3 + 2u; x6 = x5 + 2u;
\MT: z4lbl = z4; z5lbl = z5; z6lbl = z6;
\MT: rectangle(4)(u,v); ellipse(5)(u,v); rectangle(6)(u,v);
\MTlabel(4lbl)"\figfont DVI"; \MTlabel(5lbl)"\figfont MF";
\MTlabel(6lbl)"\figfont MF";
\MT: y7 = y8 = y4 - 2v; x8 = x3; z7lbl = z7; z8lbl = z8;
\MT: ellipse(7)(u,v); rectangle(8)(u,v);
\MTlabel(7lbl)"\figfont\driver"; \MTlabel(8lbl)"\figfont GF";
\MT: x9 = x7; y9t = y7t - 2v;
\MTlabel(9)"\box0";
\MT: x31r = x2l - u; y31 = y2; z31lbl = z31; rectangle(31)(u,v);
\MTlabel(31lbl)"\figfont FMT";
\MT: x32 = x5; y32b = y5t + v; z32lbl = z32; rectangle(32)(u,v);
\MTlabel(32lbl)"\figfont BAS";
\MT: thinpen; join_radius:=15pt;
\MT: z11o = (x1,h); z11d = (x1,y1t); arrow(11);
\MT: z12o = (x1,y1b); z12d = (x2,y2t); arrow(12);
\MT: z13o = (x3l,y3); z13d = (x2r,y2); arrow(13);
\MT: z14o = (x2,y2b); z14d = (x4,y4t); arrow(14);
\MT: z15o = (w,y6); z15d = (x6r,y6); arrow(15);
\MT: z16o = (x6l,y6); z16d = (x5r,y5); arrow(16);
\MT: z17o = (x5l,y5); z17d = (x3,y3b); z17m = (x3,y5);
\MT: arroww(17);
\MT: z18o = (x5l,y5); z18d = (x8,y8t); z18m = (x8,y5);
\MT: arroww(18);
\MT: z19o = (x4,y4b); z19d = (x7,y7t); arrow(19);
\MT: z20o = (x8l,y8); z20d = (x7r,y7); arrow(20);
\MT: z21o = (x7,y7b); z21d = (x9,y9t); arrow(21);
\MT: z41o = (x31r,y31); z41d = (x2l,y2); arrow(41);
\MT: z42o = (x32,y32b); z42d = (x5,y5t); arrow(42);
\MT: x0 = w/2; y0b = -d;
\MTlabel(0)"\rm \TeX\ formats";
\MTendchar;

\centerline{\box\MTbox}
.

\input example1

\vskip2cm plus 1cm minus 1cm

\divide\hsize2
Now, we reduce |\hsize| to half its normal value just to see the
effect on the figure.

\input example1

\bigbreak

\multiply\hsize2

%%%%%%%%%%%%%%%%%%%

\input shadow

\def\beginshadowpar{\setbox0=\vbox\bgroup \kern0.1pt
 \advance\parindent by -3pt \leftskip=3pt\rightskip=3pt }
\def\endshadowpar{\par\kern3pt\egroup\framebox0\shadowbox0\box0 }

\beginsection Shadowing

\beginshadowpar\noindent
Both {\TeX} and {\MF} are ill fitted to get shadows.  In {\TeX} one
straight forwarding technique would be a double use of |\leaders|, but in
practice this results in huge |dvi| files.  In {\MF} drawing lots of tiny
points easily exceeds the capacity of the program.  The solution is to
coordinate the work of both programs.
\endshadowpar

The little files |shadow.tex| and |frame.tex| do the trick.  After
|\framebox|$\,x$, |\box|$\,x$ contains what was before and,
additionally, a frame, without changing the box dimensions.  After
|\shadowbox|$\,x$, |\box|$\,x$ contains what was before but
shadowed, not framed, without changing the box dimensions.  For
example,

\halign{#\hfil\cr
 |\setbox0=\hbox{framing}\framebox0\box0| results
 \setbox0=\hbox{framing}\framebox0\box0,\cr
 |\setbox0=\hbox{shadowing}\shadowbox0\box0| results
 \setbox0=\hbox{shadowing}\shadowbox0\box0, and\cr
 |\setbox0=\hbox{framing and shadowing}\framebox0\shadowbox0\box0|\cr
 results
 \setbox0=\hbox{framing and shadowing}\framebox0\shadowbox0\box0.\cr
 }

File |frame.tex| is

\beginlisting
\listing{frame.tex}
\endlisting

File |shadow.tex| is

\beginlisting
\listing{shadow.tex}
\endlisting


\beginsection Keys

After the following {\MTeX} macros:

%%%%%%%%%%%%%%%%%%%%%%%%%%%%%%%%%%%%%%%%%%%%%%%%%%%%%%%%%%%%%%%%%%%%%%
\tofile{example2.tex}% EXAMPLE.TEX

\MTcode
def keybox =
 pickup pencircle scaled 0.8pt;
 x1 = x3 = 1pt; x2 = x4 = w - 1pt; x5 = 0; x6 = w;
 y1 = y2 = -d; y3 = y4 = h; y5 = y6 = (h - d)/2;
 draw z1 -- z2 .. z6{up} .. z4 -- z3 .. z5{down} .. cycle;
enddef;

\catcode`\@=11
\def\defkey#1#2{\setbox0=\hbox{\sf#2}\count@=\MTn@
 \dimen0=\wd0\advance\dimen0 by 2pt
 \dimen2=\ht0\advance\dimen2 by 1pt
 \dimen4=\dp0\advance\dimen4 by 1pt
 \MTbeginchar(\the\dimen0,\the\dimen2,\the\dimen4);%
  \MTline{ keybox;}%
 \MTendchar;%
% \edef#1{\ifx\MTP\undefined
%  \rlap{\kern1pt\sf#2}{\MTfont\char\the\count@}\else
%  \ifx\MTA\undefined
%   \special{psfile=auxiliar.\the\count@}\kern1pt{\sf#2}\kern1pt\else
%   \ifMTf@le\noexpand\MPtoPDF{auxiliar.\the\count@}\fi
%   \kern1pt{\sf#2}\kern1pt
% \fi\fi}}
 \edef#1{\ifMTmf 
  \rlap{\kern1pt\sf#2}{\MTfont\char\the\count@}%
 \else           
  \ifcase\pdfoutput \special{psfile=auxiliar.\number\count@}%
  \else \ifMTf@le\noexpand\MPtoPDF{auxiliar.\number\count@}\fi
  \fi 
  \kern1pt{\sf#2}\kern1pt
 \fi}}
\catcode`\@=12

\def\makekey#1{\expandafter\defkey\csname#1\endcsname{#1}}
.
%%%%%%%%%%%%%%%%%%%%%%%%%%%%%%%%%%%%%%%%%%%%%%%%%%%%%%%%%%%

\beginlisting
\listing{example2}
\endlisting
\input example2

Then, it is enough to declare |\makekey{Alt}|\makekey{Alt} to
typeset {\Alt} just writing |\Alt|.  It is also possible to
declare |\defkey\escape{\tt\char92}|\defkey\escape{\tt\char92} 
and then |\escape| results in \escape.


\beginsection Labels reference points

The next figures show how labels are located, using $f$ as label.
For example, coding |\MTlabel(5)tl"$f$";| will locate
the |tl| corner of the bounding box of $f$ in point |z5|.
Remember that |gl|, |gc| and |gr| do not work with |\MPlabel|.

\newdimen\wid \dimen\wid=3cm

\MTcode
defaultfont:="cmtt10";
bboxmargin := 0pt; labeloffset := 2bp;
picture fp; fp = btex $f$ etex scaled 9;
numeric wd; wd := \the\dimen\wid;
numeric t; t = (wd - xpart lrcorner fp) / 2;

\MTline{}

\MTbeginchar(\the\dimen\wid,2.8cm,1.2cm);
\MT: draw fp shifted (t,0) withcolor 0.6white;
\MT: picture pic; pic := currentpicture;
\MT: path bb; bb := bbox pic;
\MT: draw bb withcolor 0.6white;
\MT: draw llcorner pic -- urcorner pic withcolor 0.6white;
\MT: draw ulcorner pic -- lrcorner pic withcolor 0.6white;
\MT: dotlabel.llft("bl", llcorner pic);
\MT: dotlabel.lrt("br",  lrcorner pic);
\MT: dotlabel.urt("tr", urcorner pic);
\MT: dotlabel.ulft("tl",  ulcorner pic);
\MT: dotlabel.rt("cc", center pic);
\MT: dotlabel.bot("bc", 1/2[llcorner pic,lrcorner pic]);
\MT: dotlabel.lft("cl", 1/2[ulcorner pic, llcorner pic]);
\MT: dotlabel.top("tc", 1/2[urcorner pic,ulcorner pic]);
\MT: dotlabel.rt("cr",  1/2[urcorner pic, lrcorner pic]);
\MT: dotlabel.llft("gl", (xpart llcorner pic,0));
\MT: dotlabel.lrt("gc", (xpart center pic,0));
\MT: dotlabel.lrt("gr", (xpart lrcorner pic,0));
\MT: draw (0,0) -- (w,0) ; %-- (w,h) -- (0,h) -- cycle;
%\MT: draw (0,0) -- (0,-d) -- (w,-d) -- (w,0);
\MTendchar;
\setbox0=\box\MTbox

\MTbeginchar(\the\dimen\wid,2.8cm,1.2cm);
\MT: draw fp shifted (t,0) withcolor 0.6white;
\MT: picture pic; pic := currentpicture;
\MT: path bb; bb := bbox pic;
\MT: draw bb withcolor 0.6white;
\MT: draw llcorner pic -- urcorner pic withcolor 0.6white;
\MT: draw ulcorner pic -- lrcorner pic withcolor 0.6white;
\MT: label.bot("w", 1/2[llcorner pic,lrcorner pic]);
\MT: label.top("w", 1/2[urcorner pic,ulcorner pic]);
\MT: label.rt("h", (xpart urcorner pic, 1/2 (ypart urcorner pic) ));
\MT: label.lft("h", (xpart ulcorner pic, 1/2 (ypart ulcorner pic) ));
\MT: label.lft("d", (xpart llcorner pic, 1/2[ypart llcorner pic, 0]));
\MT: label.rt("d", (xpart lrcorner pic,  1/2[ypart lrcorner pic, 0]));
\MT: draw (0,0) -- (w,0) ; %-- (w,h) -- (0,h) -- cycle;
%\MT: draw (0,0) -- (0,-d) -- (w,-d) -- (w,0);
\MTendchar;
\setbox2=\box\MTbox

$$\box2\qquad\box0$$



\beginsection Baroque tables

Baroque periods are the result of new technical achievements providing
unexplored possibilities and hence the urgent need to experiment them,
frequently further away from what discretion recommends. This explains the
software baroquism we live in, and increases the value of \MTeX, because it
provides means to easily draw baroque tables. I am not a baroque man, so my
baroque table example is not baroque but, and this is the point, it is not
built with straight lines.

\MTcode
def tablebox =
 pickup pencircle scaled 0.8pt;
 x0 = 5pt; y0 = 5pt;
 draw (0,h) ... (w/2,h+y0) ... (w,h) ... (w+x0,(h-d)/2) ... (w,-d) ...
      (w/2,-d-y0) ... (0,-d) ... (-x0,(h-d)/2) ... cycle;
enddef;

\def\baroquetable{\setbox0=\vbox\bgroup
 \halign\bgroup##\hfil&&\quad##\hfil\cr}
\def\endbaroquetable{\crcr\egroup\egroup
  \MTbeginchar(\the\wd0,\the\ht0,\the\dp0);%
  \MTline{ tablebox;}%
 \MTendchar;%
 $$\rlap{\box\MTbox}\box0$$}

\baroquetable
 This is not a straight table\cr
 but it's only an example\cr
 and therefore not so ample\cr
 of what's {\MTeX} able
\endbaroquetable

\beginsection PostScript

Taking advantage of {\MP} it is possible to get a PostScript
version of the file.  Just include the innocent assignment
|\let\MTP/| before the statement |\input metatex|, execute
 |mpost &mpost auxiliar.mf| instead of {\MF} and, after {\TeX} second
pass, execute |dvips| (I did not test other |dvi| to PostScript
drivers).

And thanks to PDF{\TeX} and Con{\TeX}t files |supp-mis.tex| and
|supp-pdf.tex|\footnote*{In fact, I use file |mptopdf.tex| instead. In
this file I put what {\MTeX} need from the Con{\TeX}t files.} it is also
possible to get an Acrobat (|pdf|) version. Just execute twice PDF{\TeX}
instead of {\TeX} and once {\MP} instead of {\MF}.

\bigbreak\bigskip
\rightline{Ram\'on Casares\quad
 \leavevmode
  \ifcase\pdfoutput
   \special{pdf: bann << /Type /Annot
    /Subtype /Link /Border [0 0 0] /A << /Type /Action
    /S /URI /URI (mailto:papa@ramoncasares.com) >> >>}%
  \else
   \pdfstartlink attr{/Border [0 0 0]}
    user{/Subtype /Link /A << /Type /Action
    /S /URI /URI (mailto:papa@ramoncasares.com) >>}%
  \fi
  ({\tt papa@ramoncasares.com})%
  \ifcase\pdfoutput \special{pdf: eann}%
  \else \pdfendlink
  \fi}

\vfill\break

\raggedbottom

\vglue1pc
\centerline{\biglogo META\bigrm\TeX\ Code}
\bigskip

\def\verbatimoptions{\tentt \parindent=0pt \count255=0
 \everypar{\advance\count255 1 \llap{\sevenrm\the\count255\quad}}}

\listing{metatex.tex}

\bye
