% firma.tex (RMCG20090909)

\input doc
\cm12fonts
\rmcglayout


\title0 Firma

Para hacer la firma automáticamante, se utiliza "firma.mf",
que define una macro capaz de generar 256 firmas diferentes.
\file{/home/papa/texmf/fonts/source/firma.mf}

Para usarla en un documento, se hace:
\begincode
\input metatex
\MTline{input firma;}
\MTbeginchar(4cm,2cm,2mm);\MTline{firma(69);}\MTendchar;
\centerline{\box\MTbox}
\endcode

El resultado es:

\input metatex
\MTline{input firma;}
\MTbeginchar(4cm,2cm,2mm);\MTline{firma(69);}\MTendchar;
\centerline{\box\MTbox}

A mayor tamaño:

\MTbeginchar(8cm,4cm,4mm);
\MT: draw (0,0)--(w,0)--(w,h)--(0,h)--cycle;
\MT: draw (0,-d)--(w,-d);
\MT: draw (w/2,-d)..(w/2,h);
\MTline{firma(69);}
\MTendchar;
\centerline{\box\MTbox}

\bigskip

Se puede definir una macro "\signature" que elija una firma distinta
cada día. Tal vez algún día la pase a "doc.tex", pero, de momento,
la dejo aquí y la uso abajo.

\newcount\signo
\def\signature{\ifx\MT\undefined \input metatex \fi
 \edef\daynumber{\todayiso}\count255=\daynumber
 \divide\count255by256 \multiply\count255by256
 \signo=\daynumber \advance\signo-\count255
 \message{Today signature number is \the\signo.}
 \MTline{input firma;}%
 \MTbeginchar(4cm,2cm,2mm);\MTline{firma(\the\signo);}\MTendchar;%
 \box\MTbox
}

En A Coruña, a \fecha:\par\centerline{\signature}

\bye

{\bigskip\catcode`\~=12
\listing{/home/papa/texmf/fonts/source/firma.mf}
\bigbreak}






\bye



\title9 Tables

"Tables.tex" permite la realización de tablas de párrafos.  La
tabla debe comenzar con la orden "\begintable" y termina con la
orden "\endtable".  Una tabla típica puede ser:

\tofile{auxiliar.tex}...
\input tables
\boxit{\begintable\sevenrm\baselineskip=9pt
 \rightskip=10pt \leftskip=10pt
 \let\celltopborder=\thinhrule
 \let\cellbottomborder=\thinhrule
 \let\cellleftborder=\thinvrule
 \let\cellrightborder=\thinvrule
 \celldef{5.25truecm}{\let\cellleftborder=\thickvrule
  \let\cellrightborder=\doblevrule\raggedright
  \parindent=0pt\parskip=\smallskipamount}
 \celldef{5truecm}{\raggedright} % No valen horizontales, \noindent
 \celldef{5truecm}{\let\cellrightborder=\thickvrule\raggedright}
\rowdef
\ruled\thickhrule\row
\cell\vfill
1. Esta es la columna que sirve para los títulos.
\botstrut\hrule\topstrut
2. Creo que sí.
\vfill
\cell
La inferencia que se obtiene al considerar conjuntamente los dos
axiomas anteriores es trivial y constituye, junto con las dos
premisas, una demostración matemática de la irreconciliabilidad del
amor con la jurisprudencia.
\cell
\begincenter
Quede ahí la demostración matemática, y ahondemos en el asunto.
\endcenter
\row
\ruled\doblehrule\row
\cell
Esta es la columna que sirve para los títulos.
\cell  La inferencia que se obtiene al considerar conjuntamente
los dos
axiomas anteriores es trivial y constituye, junto con las dos premisas,
una demostración matemática de la irreconciliabilidad del amor con la
jurisprudencia.
\cell \item{$\bullet$} Título
\endgraf
Parece obvio que los contratos legales tienen interés debido a que las
partes firmantes de los tales contratos no tienen una absoluta confianza
la una en la otra.  La legalidad supone que, siendo menester, podría
hacerse uso de la fuerza de la policía para obligar, a la otra parte, a
cumplir con lo firmado.
\row
\cell Esta segunda suposición puede admitirse porque, aunque la
experiencia parece probar que no existe una correlación directa entre
vivir siempre juntos y amar mejor, sin embargo es algo frecuente que los
amantes que viven separados tengan, como un objetivo prioritario, vivir
siempre juntos. Debe influir en esto el hecho cierto de que el acto
sexual, que está comprobado que es placentero, hayan de realizarlo ambos
amantes necesariamente juntos.
\endgraf
Admitamos, con estas consideraciones, los dos supuestos y
analicemos la situación en la que queda la pareja.
\cell Y ya hemos visto cuando debe elegirse una o la otra
forma.
\cell\row
\ruled\thickhrule\row
\endtable}
...

\bigskip
\def\verbatimoptions{\baselineskip=12pt\catcode`\^=7\tentt}
\listing{auxiliar.tex}

Y resulta lo que se muestra en la página siguiente.
\newpage

\input auxiliar

\bigbreak\noindent Otra tabla sencilla:\smallskip
\begintable
 \parindent=0pt\parskip=\smallskipamount\raggedright
 \rightskip=5pt \leftskip=5pt
 \let\celltopborder=\thinhrule
 \let\cellbottomborder=\thinhrule
 \let\cellleftborder=\thinvrule
 \let\cellrightborder=\thinvrule
 \celldef{5cm}{\let\cellleftborder=\thickvrule
  \let\cellrightborder=\doblevrule}
 \celldef{5cm}{\let\cellrightborder=\thickvrule
  \let\cellleftborder=\relax}
\rowdef
\ruled\thickhrule\row
\cell Uno\cell Dos\row
\cell Tres\cell Cuatro\row
\ruled\thickhrule\row
\endtable


\bye
