% T@X.TEX (RMCG19930827)

\input doc
\files \rmcglayout \lm12fonts \colors

\def\* #1 {\smallbreak
 \hangindent20pt\noindent{\tt #1}\quad\ignorespaces}

\font\manual=logo10 scaled1200
\def\MF{\hbox{\manual META}\-\hbox{\manual FONT}}
\def\MT{\hbox{\manual META}\TeX}
\def\MP{\hbox{\manual META}\-\hbox{\manual POST}}
\def\Debian{{\sf Debian}}


\title0 Mi sistema TeX

\title1 General

En este documento se presentan las peculiaridades de mi sistema {\TeX}.
{\TeX} es el sistema tipográfico para ordenador
desarrollado por \[Knuth], profesor de la Universidad de Stanford.
El documento se divide en cinco partes:
\beginpoints
\item esta presentación general,
\item una segunda parte sobre ejecutables ({\latin scripts} y binarios),
\item otra tercera sobre formatos,
\item la cuarta sobre macros, y
\item la última sobre tipos (fuentes, o {\latin fonts}).
\endpoints

Actualmente uso la distribución {\TeX} Live
empaquetada por el sistema operativo {\Debian} Gnu/Linux
en su versión 6.0, llamada Squeeze.
No instalo completo "texlive" porque es enorme,
y no necesito La{\TeX}, ya que uso mi propio formato "spplain",
que es una adaptación del formato "plain" al castellano.
En concreto he instalado los siguientes paquetes:
\beginpoints
\item "texlive-core", incluyendo su recomendado "lmodern",
\item "texlive-metapost", pero no su recomendado "feynmf", y
\item "texlive-fonts-recommended", pero no su recomendado "tipa".
\endpoints

Utilizo la organización de directorios propuesta por
{\Debian} y {\latin {\TeX} Directory Structure},
así que las peculiaridades de mi {\TeX} se encuentran en
el directorio "~/texmf", cuyo contenido es,
pues, el objeto del presente documento.

Uso Git para contralar las versiones de este directorio "~/texmf",
así que, si se dispone de un ordenador conectado a la red,
con el sistema operativo {\Debian} Squeeze, y están instalados
los paquetes "git" y "gcc" (incluyendo el recomendado "libc6-dev"),
entonces se puede obtener una copia haciendo la siguiente maniobra:
\begincode
cd
git clone git://github.com/ramoncasares/texmf.git
texmf/exe/sh/install.sh
\endcode
El comando "cd", sin argumentos, es justo lo mismo que "cd ~".
El comando "git clone" crea un directorio "~/texmf",
y copia "spplain" en él.
El comando "install.sh" se encarga de dejarlo todo en orden,
y listo para ser usado.

En "~/texmf/doc" está la documentación, en formato fuente.
Para crear la versión pdf de un documento {\TeX}
llamado "filename.tex" que se encuentre en el directorio de trabajo,
hay que escribir "wpdf filename" en el terminal.
De este modo se puede reconstruir toda la documentación.


\title1 Ejecutables

Los ejecutables de {\TeX} Live son accesibles desde cualquier
directorio, ya que {\Debian} se preocupa de que lo estén.
Pero no hablaré aquí de esos binarios, porque
ya se explican en la documentación de {\TeX} Live.

Mis ejecutables para {\TeX}, {\latin scripts}, fuentes y binarios,
se encuentran en el directorio "~/texmf/exe",
ordenados en subdirectorios:
los {\latin scripts} en "~/texmf/exe/sh",
los fuentes según su lenguaje,
 así que los escritos en C se encuentran en "~/texmf/exe/c",
y los binarios en "~/texmf/exe/bin".

\title2 Scripts

\* install.sh compila los binarios C,
construye los formatos {\TeX} binarios,
crea un directorio "~/bin", si es que no hay ya uno,
y lo llena con enlaces simbólicos a los ejecutables
(scripts and binarios),
y actualiza la base de datos de {\TeX} Live.
\file{~/texmf/exe/sh/installtex.sh}

\* wmake.sh produce mis formatos locales de {\TeX}.
Para {\MP} no se necesita, y ya no necesito acomodar los de {\MF}.
\file{~/texmf/exe/sh/wmake.sh}

\* wlsr.sh actualiza las bases de datos.
Es simplemente un sinónimo del "mktexlsr" de {\TeX} Live:
\file{~/texmf/exe/sh/wlsr.sh}

\* clrtex.sh que borra los ficheros auxiliares.
\file{~/texmf/exe/sh/clrtex.sh}

\* wtex.sh toma un fichero {\TeX} y crea el correspondiente "dvi",
usando el formato "spplain". Está especialmente diseñado para
ser usado con {\MT} e "index.tex".
Es decir, ejecuta {\MF} si cambian las figuras, y
hace tantas pasadas como sea menester para obtener
las referencias correctas. Es como sigue:
\file{~/texmf/exe/sh/wtex.sh}

\* wps.sh que es similar a "tex.sh", pero que ejecuta "dvips"
para entregar un fichero "ps" con las fonts "cm" de Donald Knuth
incluidas parcialmente en formato "t1". Además, usa {\MP} en vez
de {\MF} para generar las figuras.
\file{~/texmf/exe/sh/wps.sh}

\* wpdf.sh que es como "tex.sh" pero que usa "pdftex" por lo que
produce ficheros "pdf" (acrobat). También usa {\MP} para generar
las figuras.
\file{~/texmf/exe/sh/wpdf.sh}

\title2 Binarios

\* index.c es un pequeño programa C que utilizan los tres anteriores.
Uso:
\begincode
index ndx < ind > abc
\endcode
Toma las referencias de un fichero "ind",
las entradas de un fichero "ndx",
y produce un fichero "abc".
\file{~/texmf/exe/c/index.c}

\* firstindex.c sirve para componer el primer fichero "ndx" de
"index.c", a partir de un fichero "ind".
Al fichero "ind" hay que eliminarle la notación "^^" con "readtex.c",
después ordenarlo alfabéticamente con "texsort",
y finalmente "firstindex.c" hace un boceto de "ndx" creando
unas {\latin patterns} sencillas que ignoran si la primera
letra es mayúscula o minúscula, y añade plurales según una regla
fácil, pero que no siempre acierta.
Por eso, para realizar un buen índice hay que revisarlo completamente.
\file{~/texmf/exe/c/firstindex.c}

\* texsort.c es el programa que ordena alfabéticamente las líneas.
\file{~/texmf/exe/c/texsort.c}

\* readtex.c es un programa que convierte la notación "^^" de {\TeX}.
\file{~/texmf/exe/c/readtex.c}

\* writetex.c es el programa que hace lo inverso que "readtex.c".
\file{~/texmf/exe/c/writetex.c}


\title1 Formatos

He definido un formato específico de {\TeX} para trabajar en castellano.
Se construye sobre el formato "plain". Usa tres ficheros.

\* spplain.tex que hace la maniobra que permite construir el
formato sobre "plain".
\file{~/texmf/tex/spplain/base/spplain.tex}

\* spdflain.tex que es una variante de "spplain" para {Pdf\TeX}.
\file{~/texmf/tex/spplain/base/spdflain.tex}

\* sphyphen.tex que contiene el núcleo del formato.
Define las letras acentuadas y otros caracteres específicos del
castellano indistintamente según las codificaciones
ASCII ("plain" {\TeX}), ISO-8859-1, o UTF-8.
Divide las palabras usando las reglas del castellano.
Puede usar fonts codificadas como en
"plain", OT1, o según Cork, T1, o ANSI, "texnansi".
Excepto cuando usa "\plaindefaults", pone
el "catcode" de "^^c2" (Acircumflex) a 13 ({\latin active}), y
el "catcode" de "^^c3" (Atilde) a 0 ({\latin escape}).
\file{~/texmf/tex/spplain/hyphen/sphyphen.tex}

No necesito formatos locales de {\MF} ni de {\MP}.


\title1 Macros

\title2 TeX

Los ficheros de este apartado contienen macros de utilidad. Para
utilizarlas es necesario importar estas macros, para lo que se utilizará
el comando {\TeX} "\input <filename>". Se encuentran todos ellos en
"~/texmf/tex/spplain/rmcg"

\* explain.tex contiene definiciones varias de utilidad general.
\file{~/texmf/tex/spplain/rmcg/explain.tex}

\* index.tex contiene un sistema de generación de índices y referencias
adaptado a "pdf".
\file{~/texmf/tex/spplain/rmcg/index.tex}

\* fonts.tex contiene algunas macros que facilitan el uso de las
fonts que utilizan la codificación "plain", ahora llamada "7t".
Ha sido adaptado de "pdcfsel.tex" (v 3.5, 1995/03/28) by P.~Damian
Cugley. Uso: Se basa en los conceptos de familia y estilo.
Una familia es un conjunto de fonts relacionadas,
como "\cmfonts" que prepara las cosas para que las macros de estilo:
"\rm", "\it", "\sl", "\bf", "\sl" y "\tt",
usen la font correspondiente a dicho estilo dentro de la familia.
Al importar este fichero se definen las familias:
"\cmfonts" (10 pt plain),
"\viiifonts" (8 pt plain),
"\xiifonts" (12 pt plain),
"\xiv", "\xvii", "\xxifonts" (14, 17 y 21 pt plain),
"\psfonts" (10 pt PostScript),
"\xiipsfonts" (12 pt PostScript),
"\xivps", "\xviips", "\xxipsfonts" (14, 17 y 21 pt PostScript),
"\zerofonts" (17 pt, para los títulos),
"\onefonts" (14.4 pt, para los títulos) y
"\twofonts" (12 pt, para los títulos).
Las macros "\cmtitles", "\viiititles", "\xiititles" y "\xiipstitles"
definen las macros "\fontzero", "\fontone" y "\fonttwo"
para los títulos, de tal manera que
"\xxfonts" congenia con "\xxtitles".
\file{~/texmf/tex/spplain/rmcg/fonts.tex}

\* exfonts.tex incluye las fuentes menos usadas
("\ccfonts", "\pnfonts", "\lmfonts" y "\lmtitles")
y macros para la conversión de códigos.
Para el cambio de codificación se proporcionan las macros
"\dccoding" y "\cmcoding",
que presuponen que los ficheros de entrada usan para los caracteres
castellanos la codificación según la tabla ANSI (Win) y que éstos están
activos, como hace el formato "spplain".
%\file{~/texmf/tex/spplain/rmcg/exfonts.tex}

\* metatex.tex (\MT) permite incluir definiciones {\MF} en
ficheros {\TeX}. Se explica en más detalle en su propio documento.
\file{~/texmf/tex/spplain/rmcg/metatex.tex}

\* doc.tex permite escribir documentos.  Llama a los ficheros
"explain.tex", "fonts.tex", "index.tex" y, opcionalmente, "metatex.tex".
También se explica, a nivel de usuario, en su propio documento.
\file{~/texmf/tex/spplain/rmcg/doc.tex}

%\* marks.tex permite incluir marcas de modificación.  Si "\marks"
%está activo, entonces queda subrayado el texto desde "\new/" hasta
%"\wen/" y tacha el texto desde "\del/" hasta "\led/".  Si "\nomarks"
%está activo, y lo está por defecto al importar "marks", entonces
%ignora las marcas "\new/" y "\wen/" e ignora el texto desde "\del/"
%hasta "\led/".
%\file{~/texmf/tex/spplain/rmcg/marks.tex}

\* doubleco.tex permite tipografiar a dos columnas.
Uso:  una vez importado, la orden
"\doublecolumns" comienza la zona a dos columnas y la orden
"\singlecolumn" vuelve a una columna.
\file{~/texmf/tex/spplain/rmcg/doubleco.tex}

%\* twocols.tex permite tipografiar a dos columnas. Sólo permite
%escribir a una única columna al principio de una página, por
%ejemplo para hacer los títulos, mientras que todo lo demás lo hace
%a dos columnas.  Uso:  una vez importado, la orden "\titles"
%comienza hoja y pasa a una columna, y la orden "\endtitles" pasa a
%dos columnas.
%\file{~/texmf/tex/spplain/rmcg/twocols.tex}

%\* frame.tex enmarca una "\box" sin modificar las dimensiones.
%También permite modificar el grosor de las reglas con
%"\rulethickness".  Está en "rmcg".  Uso: "\framebox0" enmarca
%"\box0".
%\file{~/texmf/tex/spplain/rmcg/frame.tex}

%\* shadow.tex sombrea una "\box" sin modificar sus dimensiones.
%Llama a "frame.tex" y, si es preciso, a "metatex.tex".
%Uso:  "\shadowbox255" sombrea, pero no enmarca, la "\box255".
%\file{~/texmf/tex/spplain/rmcg/shadow.tex}

\* crops.tex permite añadir marcas de alineación (en inglés
{\it crop marks\/}) e imprimir sólo aquellas páginas que aparecen en el
fichero "pages.tex", a razón de un número de página por línea. Si no
existe el fichero "pagex.tex", entonces se imprimen todas las páginas.
\file{~/texmf/tex/spplain/rmcg/crops.tex}

\* java.tex que sirve para escribir en `literate' Java.
Los ficheros java se imprimen con el comando
\verb|\javainput"filename.java"|
que imprime el fichero tal cual es ({\it verbatim\/}) pero detecta los
comentarios de documentación, los que empiezan por "/**", y a partir de
ese instante y hasta que encuentra el siguiente "*/" interpreta lo que
sigue como {\TeX} además de interpretar los comandos de documentación de
Java, como "@param".
Se explica, a nivel de usuario, en su propio documento.
\file{~/texmf/tex/spplain/rmcg/java.tex}

%\* a5.tex permite usar el formato de ``El problema aparente'' aumentado
% al %120\% y con marcas de enmarcado de hoja.
%\file{~/texmf/tex/spplain/rmcg/a5.tex}


\title2 METAFONT

\* delay.mf contiene varias formas de uso generalizado para
{\MF} o {\MP}.
\file{~/texmf/fonts/source/delay.mf}

\* circuit.mf contiene las definiciones de los signos utilizados al
dibujar circuitos lógicos y eléctricos.
\begincode
or_gate(suffix s)(expr size)
and_gate(suffix s)(expr size)
not_gate(suffix s)(expr size)
power_supply(suffix s)(expr size)
resistor(suffix s)(expr size)
\endcode

\* graphbas.mf contiene las definiciones utilizadas por "mfpic.tex".


\title2 METAPost

Uso {\MP} para que {\MT} produzca las figuras en formato "eps" de
manera que se produzca un fichero "ps" en vez de "dvi".


\title1 Fonts

En esta sección se enumeran las fonts que uso.
Son las que se obtienen de {\TeX} Live, o, más concretamente,
de los paquetes de {\TeX} Live que he instalado, además de las mías.
Dentro de estas últimas están las creadas o modificadas por mi,
y las simplemete copiadas y ubicadas.
Repasaré los tres tipos: copiadas, empaquetadas, y modificadas.

\title2 Copiadas

Son dos, que no he tomado de los paquetes en los que se encuetran,
una porque es un paquete grande del que sólo me interesa una parte
minúscula, y otra porque todavía no está empaquetada. Son:
SkakNew-DiagramT, que es parte del paquete "texlive-games" en Squeeze,
y CCicons, que vendrá en la próxima versión de {\Debian}.

El procedimiento para incluir una de ellas, usaré CCicons como ejemplo,
es como sigue:
\beginnumbers
\item Conseguir los ficheros "ccicons.tfm", "ccicons.pfb",
      y "ccicons.map", y ubicarlos en los subdirectorios adecuados
      de "~/texmf/fonts/".
\item Ejecutar "wlsr.sh". Se puede comprobar con "kpsewhich" que
      {\TeX} Live los encuentra.
\item Ejecutar "updmap --enable Map=ccicons.map", para que
      los pueda usar "pdftex" y "dvips".
      Se debe usar "MixedMap" en vez de "Map" si también disponemos
      del código {\MF}.
\endnumbers


\title2 Empaquetadas

Hay más, pero aquí listo las más útiles.

\title3 Computer Modern

Mis fuentes por defecto son las canónicas de {\TeX},
o sea, las de la familia Computer Modern diseñada por \[Knuth].
Las 75 fuentes "cm" canónicas son:
\begincode
cmb10   cmbsy10
cmbx5 cmbx6 cmbx7 cmbx8 cmbx9 cmbx10 cmbx12
cmbxsl10 cmbxti10   cmcsc10 cmdunh10 cmex10
cmff10   cmfi10   cmfib8   cminch   cmitt10
cmmi5 cmmi6 cmmi7 cmmi8 cmmi9 cmmi10 cmmi12  cmmib10
cmr5 cmr6 cmr7 cmr8 cmr9 cmr10 cmr12 cmr17
cmsl8 cmsl9 cmsl10 cmsl12  cmsltt10
cmss8 cmss9 cmss10 cmss12 cmss17  cmssbx10  cmssdc10
cmssi8 cmssi9 cmssi10 cmssi12 cmssi17  cmssq8 cmssqi8
cmsy5 cmsy6 cmsy7 cmsy8 cmsy9 cmsy10
cmtcsc10  cmtex8 cmtex9 cmtex10
cmti7 cmti8 cmti9 cmti10 cmti12
cmtt8 cmtt9 cmtt10 cmtt12   cmu10   cmvtt10
\endcode
Están disponibles en formato {\TeX}, "pk",
y PostScript T1, "pfb".
Y este último de dos maneras:
convertidas por Y\&Y para AMS,
y rehechas y extendidas con acentos para
el proyecto Latin Modern (son "lm" en vez de "cm").

\title3 AMS

Son las fuentes adicionales, mantenidas por la AMS. Son:
\begincode
cmmib5 cmmib6 cmmib7 cmmib8 cmmib9 cmex9
cmbsy5 cmbsy6 cmbsy7 cmbsy8 cmbsy9
msam10 msam9 msam8 msam7 msam6 msam5
msbm10 msbm9 msbm8 msbm7 msbm6 msbm5
eufm10 eufm9 eufm8 eufm7 eufm6 eufm5
\endcode
Existen en ambos formatos, "pk" y "pfb".

\title3 Logo

Son las fonts del logo de {\MF} y {\MP}:
"logo10", "logo9", "logo8", "logobf10" y "logosl10".
Existen en ambos formatos, "pk" y "pfb".


\title2 Modificadas

Se trata se las fuentes de fonts creadas por mi mismo, algunas son
meras variaciones sobre la base de las "cm".

\title3 Fuentes virtuales

Cuando no existían las fuentes Latin Modern,
usé "fontinst" para crear fuentes virtuales componiendo los
caracteres de las fuentes Computer Modern,
y así obtener los caracteres acentuados.
Ahora ya no las necesito,
pero apunto el procedimiento por si lo necesitara en un futuro.

Lo primero era usar "tftopl" para crear la versión legible
de "cmr10.tfm", que es la lista de propiedades "cmr10.pl".
Después había que ejecutar {\TeX} haciendo "\input fontinst.sty",
e "\installfont{ecr10}...", que usaba la lista de propiedades y
el fichero auxiliar "rmcg.mtx", para obtener "ecr10.vpl".
Por último, con "vptovf" se generaban, a partir de "ecr10.vpl",
los ficheros "ecr10.vf" y "ecr10.tfm".

También usé "fontinst" para crear fuentes con los números
tipo {\latin old digits}, "cmrj10".

\title3 Fuentes alteradas

\* slssbx10.mf es como "cmssbx10.mf" pero con inclinación.
\file{~/texmf/fonts/source/slssbx10.mf}

\* slssdc10.mf es como "cmssdc10.mf" pero con inclinación.

\* rtssq8.mf es "cmssq8.mf" pero girada $90^\circ$. Es:
\file{~/texmf/fonts/source/rtssq8.mf}

\title3 Fuentes nuevas

%\* logotel.mf contiene los logotipos de {\sf Telefónica},
%los redondos y elípticos y en tres tamaños cada uno (22pt, 33pt y 66pt).
%(sustituye a "tlogonew.mf"). Usa la siguiente tabla: redonda pequeña
%es `.', redonda mediana es `o', redondo grande es `O', elíptica
%pequeña es `/', elíptica mediana es `t' y elíptica grande es `T'.
%Además los caracteres 0 (tl), 1 (tr), 2 (bl) y 3 (br) dibujan las
%esquinas del cuadro redondeado de la contraportada de los documentos
%de {\sf Telefónica}.

\* manos.mf contiene las instrucciones para dibujar dos manos que
apuntan, una a la derecha y otra a la izquierda, y en tres tamaños
(24pt, 48pt y 96pt).  Usa la siguiente tabla:  derecha pequeña
`">"', derecha mediana `r', derecha grande `R', izquierda pequeña
`"<"', izquierda mediana `l' e izquierda grande `L'.

\* figuras.mf contiene figuras diversas, con la siguiente
disposición: el teléfono es la `t', la base de datos es la `d', el papel
impreso es la `p', la antena radiante es la `i', el coche es la `c'.

\* firma.mf contiene 256 firmas diferentes de RMCG.


%\title2 Otros
%
%Strunder permite generar automáticamente fonts virtuales tachadas y
%subrayadas con ayuda de Scheme (Lisp).
%
%Para que las fuentes canónicas de PostScript sean proporcionadas a la
%Times Roman, debe usarse la Helvética al 90\% y la Courier condensada
%al 85\% del ancho.


\title9 Listados

\def\ndxline#1#2#3#4#5{\noindent\pdfgoto{{\tt #1}}{#3}\quad\S#4 pg #5\par}
\input auxiliar.ind


\title9 Índice

\input auxiliar.toc


\bye
