% MTMP2PDF.TeX (RMCG19970919)

% MTMP2PDF.TeX converts from PostScript code (by METAPost) to PDF
% Adapted by Ram\'on Casares from ConTeXt by PRAGMA
% ConTeXt copyright={PRAGMA / Hans Hagen & Ton Otten}
% Under the same terms as ConTeXt
% http://www.pragma-ade.com/general/manuals/mreadme.pdf

% \catcode`\@ should be 11

\newcount\MPcounter
\newdimen\MPdimen

% POSTSCRIPT operators that METAPOST uses

\def \PScurveto          {curveto}
\def \PSlineto           {lineto}
\def \PSmoveto           {moveto}
\def \PSshowpage         {showpage}
\def \PSnewpath          {newpath}
\def \PSfshow            {fshow}
\def \PSclosepath        {closepath}
\def \PSfill             {fill}
\def \PSstroke           {stroke}
\def \PSclip             {clip}
\def \PSrlineto          {rlineto}
\def \PSsetlinejoin      {setlinejoin}
\def \PSsetlinecap       {setlinecap}
\def \PSsetmiterlimit    {setmiterlimit}
\def \PSsetgray          {setgray}
\def \PSsetrgbcolor      {setrgbcolor}
\def \PSsetdash          {setdash}
\def \PSgsave            {gsave}
\def \PSgrestore         {grestore}
\def \PStranslate        {translate}
\def \PSscale            {scale}
\def \PSconcat           {concat}
\def \PSdtransform       {dtransform}

\def \PSBoundingBox      {BoundingBox:}
\def \PSHiResBoundingBox {HiResBoundingBox:}
\def \PSExactBoundingBox {ExactBoundingBox:}
\def \PSPage             {Page:}

% A stack

\newcount\nofMPsegments
\newcount\nofMParguments

\def\@@MP{@@MP}

\def\setMPargument#1{\advance\nofMParguments by 1
 \expandafter\def
 \csname\@@MP\the\nofMPsegments\the\nofMParguments\endcsname%
 {\do#1}}

\def\gMPa#1{\csname\@@MP0#1\endcsname}
\def\gMPs#1{\csname\@@MP\the\nofMPsegments#1\endcsname}

\def\setMPkeyword#1
  {\expandafter\def\csname\@@MP\the\nofMPsegments0\endcsname{#1}%
   \advance\nofMPsegments by 1 \nofMParguments=0\relax}

\def\getMPkeyword#1{\csname\@@MP#10\endcsname}

\def\resetMPstack{\catcode`\%=13 % active
   \let\handleMPgraphic=\handleMPendgraphic
   \def\resetMPstack{\nofMParguments=0\relax}%
   \resetMPstack}

% Ignore, most of times, ( and ), but with care
% Take care, also of TeX specials

\def\octalMPcharacter#1#2#3{\char'#1#2#3\relax}

\bgroup
\catcode`\|=14 \catcode`\%=13 \catcode`\[=13 \catcode`\]=13
\catcode`\{=13 \catcode`\}=13 \catcode`B=1 \catcode`E=2
\gdef\ignoreMPspecialsB\def%BE\def[BE\def]BE\def{BE\def}BEE
\gdef\obeyMPspecialsB\def%B\char 37\relax E|
 \def[B\char 91\relax E\def]B\char 93\relax E|
 \def{B\char123\relax E\def}B\char125\relax EE
\gdef\setMPspecialsB\catcode`\%=13
 \catcode`\[=13 \catcode`\]=13 \catcode`\{=13 \catcode`\}=13
 \catcode`\$=11 \catcode`\_=11 \catcode`\#=11 \catcode`\^=11
 \catcode`\&=11 \catcode`\|=11 \catcode`\~=11
 \def\(B\char40\relax E\def\)B\char41\relax E\def\\B\char92\relax E|
 \def\0B\octalMPcharacter0E\def\1B\octalMPcharacter1E|
 \def\2B\octalMPcharacter2E\def\3B\octalMPcharacter3E|
 \def\4B\octalMPcharacter4E\def\5B\octalMPcharacter5E|
 \def\6B\octalMPcharacter6E\def\7B\octalMPcharacter7E|
 \def\8B\octalMPcharacter8E\def\9B\octalMPcharacter9EE
\egroup

% We use the comment symbol as a sort of trigger:

\bgroup \catcode`\%=13
\gdef\startMPscanning{\let%=\startMPconversion}
\egroup

\def\startMPconversion{\catcode`\%=9 \ignoreMPspecials \handleMPsequence}

% The main loop.
% Most arguments are numbers that can be recognized by their \lccode.

\def\dohandleMPsequence#1#2 {\ifnum\lccode`#1=0 \setMPargument{#1#2}%
 \else \edef\somestring{#1#2}\ifx\somestring\PSmoveto
   \edef\lastMPmoveX{\gMPa1}\edef\lastMPmoveY{\gMPa2}%
   \pdfliteral{\gMPa1 \gMPa2 m}\resetMPstack
  \else\ifx\somestring\PSnewpath \let\handleMPsequence=\handleMPpath
  \else\ifx\somestring\PSgsave \pdfliteral{q}\resetMPstack
  \else\ifx\somestring\PSgrestore \pdfliteral{Q}\resetMPstack
  \else\ifx\somestring\PSdtransform % == setlinewidth
   \let\handleMPsequence=\handleMPdtransform
  \else\ifx\somestring\PSconcat
    \pdfliteral{\gMPa1 \gMPa2 \gMPa3 \gMPa4 \gMPa5 \gMPa6 cm}\resetMPstack
  \else\ifx\somestring\PSsetrgbcolor
    \pdfliteral{\gMPa1 \gMPa2 \gMPa3 rg \gMPa1 \gMPa2 \gMPa3 RG}%
    \resetMPstack
  \else\ifx\somestring\PSsetgray \pdfliteral{\gMPa1 g \gMPa1 G}\resetMPstack
  \else\ifx\somestring\PStranslate
   \pdfliteral{1 0 0 1 \gMPa1 \gMPa2 cm}\resetMPstack
  \else\ifx\somestring\PSsetdash \handleMPsetdash \resetMPstack
  \else\ifx\somestring\PSsetlinejoin \pdfliteral{\gMPa1 j}\resetMPstack
  \else\ifx\somestring\PSsetmiterlimit \pdfliteral{\gMPa1 M}\resetMPstack
  \else\ifx\somestring\PSfshow \handleMPfshow \resetMPstack
  \else\ifx\somestring\PSsetlinecap \pdfliteral{\gMPa1 J}\resetMPstack
  \else\ifx\somestring\PSrlineto
   \pdfliteral{\lastMPmoveX\space \lastMPmoveY\space l S}\resetMPstack
  \else\ifx\somestring\PSscale
   \pdfliteral{\gMPa1 0 0 \gMPa2 0 0 cm}\resetMPstack
  \else \handleMPgraphic{#1#2}%
 \fi\fi\fi\fi\fi\fi\fi\fi \fi\fi\fi\fi\fi\fi\fi\fi \fi
 \handleMPsequence}

% Graphics

\def\handleMPendgraphic#1{\ifx\somestring\PSshowpage
 \let\handleMPsequence=\finishMPgraphic \else
 \setMPargument{#1}\fi}

\def\handleMPbegingraphic#1{\ifx\somestring\PSBoundingBox
  \let\handleMPsequence=\handleMPboundingbox
 \else\ifx\somestring\PSHiResBoundingBox
  \let\handleMPsequence=\handleMPboundingbox
 \else\ifx\somestring\PSExactBoundingBox
  \let\handleMPsequence=\handleMPboundingbox
 \else\ifx\somestring\PSPage
   \let\handleMPsequence=\handleMPpage
 \else \setMPargument{#1}\fi\fi\fi\fi}

\let\handleMPgraphic=\handleMPbegingraphic

% There are three kind of bounding boxes:
%    BoundingBox: llx lly ucx ucy
%    HiResBoundingBox: llx lly ucx ucy
%    ExactBoundingBox: llx lly ucx ucy
% The dimensions are saved for later use.

\def\handleMPboundingbox #1 #2 #3 #4 {\nofMParguments=0
 \let\handleMPsequence=\dohandleMPsequence \handleMPsequence}

% The page comment is a signal that stackbuilding can be started

\def\handleMPpage #1 #2 {\nofMParguments=0
 \let\handleMPsequence=\dohandleMPsequence \handleMPsequence}

% METAPOST: 0 0 rlineto

\def\lastMPmoveX{0}\def\lastMPmoveY{0}

% Text

{\catcode`\.=12 \catcode`\p=12 \catcode`\t=12
 \gdef\WITHOUTPT#1pt{#1}}
\def\withoutpt#1{\expandafter\WITHOUTPT#1}
\def\ScaledPointsToBigPoints#1#2{\MPdimen=#1sp\relax
 \MPdimen=.996264\MPdimen \edef#2{\withoutpt{\the\MPdimen}}}

\def\handleMPfshow{\setbox0=\hbox{\obeyMPspecials
  \edef\size{\gMPa{\the\nofMParguments} }%
  \advance\nofMParguments by -1
  \font\temp=\gMPa{\the\nofMParguments} at \size bp
  \advance\nofMParguments by -1 \temp
  \ifnum\nofMParguments=1 \def\do(##1){##1}\gMPa1%
  \else \MPcounter=1
   \def\do(##1{##1}\gMPa{\the\MPcounter}\space\def\do{}%
   \loop \advance\MPcounter by 1
    \ifnum\MPcounter<\nofMParguments \gMPa{\the\MPcounter}\space\repeat
   \def\do##1){##1}\gMPa{\the\MPcounter}%
  \fi
  \unskip}%
 \dimen0=\lastMPmoveY bp \advance\dimen0 by \ht0
 \ScaledPointsToBigPoints{\number\dimen0}\lastMPmoveY
 \pdfliteral{n q 1 0 0 1 \lastMPmoveX\space\lastMPmoveY\space cm}%
 \dimen0=\ht0 \advance\dimen0 by \dp0 \box0 \vskip-\dimen0
 \pdfliteral{Q}}


% Take care of dashes

\def\handleMPsetdash{\bgroup \def\somestring{[}\MPcounter=1
   \loop \ifnum\MPcounter<\nofMParguments
     \edef\somestring{\somestring\space\gMPa{\the\MPcounter}}%
     \advance\MPcounter by 1 \repeat
   \edef\somestring{\somestring]\gMPa{\the\MPcounter} d}%
   \pdfliteral{\somestring}\egroup}

% setlinewidth

\def\handleMPdtransform{\ifdim\gMPa1pt>0pt
 \pdfliteral{\gMPa1 w}%
 \def\next##1 ##2 ##3 ##4 ##5 ##6 {\handleMPsequence}\else
 \pdfliteral{\gMPa2 w}%
 \def\next##1 ##2 ##3 ##4 {\handleMPsequence}\fi
 \let\handleMPsequence=\dohandleMPsequence \resetMPstack
 \next}

% Paths, concat

\def\handleMPpath{\chardef\finiMPpath=0
 \let\closeMPpath=\relax \let\flushMPpath=\flushnormalMPpath
 \resetMPstack \nofMPsegments=1
 \let\handleMPsequence=\dohandleMPpath \dohandleMPpath}

\def\flushnormalMPpath{\MPcounter=\nofMPsegments \nofMPsegments=1
 \loop \expandafter\ifcase\getMPkeyword{\the\nofMPsegments}\relax
   \pdfliteral{\gMPs1 \gMPs2 l}\or
   \pdfliteral{\gMPs1 \gMPs2 \gMPs3 \gMPs4 \gMPs5 \gMPs6 c}\or
   \pdfliteral{\lastMPmoveX\space \lastMPmoveY\space l S}\or
   \edef\lastMPmoveX{\gMPs1}\edef\lastMPmoveY{\gMPs2}%
   \pdfliteral{\lastMPmoveX\space \lastMPmoveY\space m}\fi
  \advance\nofMPsegments by 1\relax
  \ifnum\nofMPsegments<\MPcounter \repeat}

\def\flushconcatMPpath{\MPcounter=\nofMPsegments \nofMPsegments=1
 \loop \expandafter\ifcase\getMPkeyword{\the\nofMPsegments}\relax
   \doMPconcat{\gMPs1}\a{\gMPs2}\b \pdfliteral{\a\space \b\space l}%
  \or
   \doMPconcat{\gMPs1}\a{\gMPs2}\b
   \doMPconcat{\gMPs3}\c{\gMPs4}\d
   \doMPconcat{\gMPs5}\e{\gMPs6}\f
   \pdfliteral{\a\space \b\space \c\space \d\space \e\space \f\space c}%
  \or
   \bgroup \noMPtranslate \doMPconcat\lastMPmoveX\a\lastMPmoveY\b
    \pdfliteral{\a\space \b\space l S}\egroup
  \or
   \edef\lastMPmoveX{\gMPs1}\edef\lastMPmoveY{\gMPs2}%
   \doMPconcat\lastMPmoveX\a\lastMPmoveY\b
   \pdfliteral{\a\space \b\space m}%
  \fi
  \advance\nofMPsegments by 1\relax
  \ifnum\nofMPsegments<\MPcounter \repeat}

% The transformation of the coordinates

\def\MPconcatfactor{256}

\def\doMPreducedimen#1 {\count0=\MPconcatfactor
   \advance\dimen#1 \ifdim\dimen#1>0pt .5\else -.5\fi\count0
   \divide\dimen#1 \count0\relax}

\def\doMPexpanddimen#1 {\multiply\dimen#1 \MPconcatfactor\relax}

\def\presetMPconcat%
  {\dimen 0=\gMPs1 pt \doMPreducedimen 0    % r_x
   \dimen 2=\gMPs2 pt \doMPreducedimen 2    % s_x
   \dimen 4=\gMPs3 pt \doMPreducedimen 4    % s_y
   \dimen 6=\gMPs4 pt \doMPreducedimen 6    % r_y
   \dimen 8=\gMPs5 pt \doMPreducedimen 8    % t_x
   \dimen10=\gMPs6 pt \doMPreducedimen10 }  % t_y

\def\presetMPscale%
  {\dimen 0=\gMPs1 pt \doMPreducedimen 0
   \dimen 2=0pt
   \dimen 4=0pt
   \dimen 6=\gMPs2 pt \doMPreducedimen 6
   \dimen 8=0pt
   \dimen10=0pt }

\def\noMPtranslate% use this one grouped
  {\dimen 8=0pt                             % t_x
   \dimen10=0pt }                           % t_y

\def\doMPconcat#1#2#3#4%
  {\dimen12=#1 pt \doMPreducedimen12        % p_x
   \dimen14=#3 pt \doMPreducedimen14        % p_y
   %
   \dimen16  \dimen 0
   \multiply \dimen16  \dimen 6
   \dimen20  \dimen 2
   \multiply \dimen20  \dimen 4
   \advance  \dimen16 -\dimen20
   %
   \dimen18  \dimen12
   \multiply \dimen18  \dimen 6
   \dimen20  \dimen14
   \multiply \dimen20  \dimen 4
   \advance  \dimen18 -\dimen20
   \dimen20  \dimen 4
   \multiply \dimen20  \dimen10
   \advance  \dimen18  \dimen20
   \dimen20  \dimen 6
   \multiply \dimen20  \dimen 8
   \advance  \dimen18 -\dimen20
   %
   \multiply \dimen12 -\dimen 2
   \multiply \dimen14  \dimen 0
   \advance  \dimen12  \dimen14
   \dimen20  \dimen 2
   \multiply \dimen20  \dimen 8
   \advance  \dimen12  \dimen20
   \dimen20  \dimen 0
   \multiply \dimen20  \dimen10
   \advance  \dimen12 -\dimen20
   %
   \doMPreducedimen16
   \divide   \dimen18  \dimen16 \doMPexpanddimen18
   \divide   \dimen12  \dimen16 \doMPexpanddimen12
   %
   \edef#2{\withoutpt{\the\dimen18}}%       % p_x^\prime
   \edef#4{\withoutpt{\the\dimen12}}}       % p_y^\prime

\def\processMPpath{\flushMPpath \closeMPpath
   \pdfliteral{\ifcase\finiMPpath W n\or S\or f\or B\fi}%
   \let\handleMPsequence=\dohandleMPsequence
   \resetMPstack \nofMPsegments=0 \handleMPsequence}

\def\handleMPpathconcat{\presetMPconcat
 \pdfliteral{\gMPs1 \gMPs2 \gMPs3 \gMPs4 \gMPs5 \gMPs6 cm}%
 \resetMPstack}

\def\handleMPpathscale{\presetMPscale
   \pdfliteral{\gMPs1 0 0 \gMPs2 0 0 cm}\resetMPstack}

\def\dohandleMPpath#1#2 {\ifnum\lccode`#1=0 \setMPargument{#1#2}%
 \else \def\somestring{#1#2}\ifx\somestring\PSlineto \setMPkeyword0
 \else\ifx\somestring\PScurveto \setMPkeyword1
 \else\ifx\somestring\PSrlineto \setMPkeyword2
 \else\ifx\somestring\PSmoveto \setMPkeyword3
 \else\ifx\somestring\PSclip \let\handleMPsequence=\processMPpath
 \else\ifx\somestring\PSgsave \chardef\finiMPpath=3
 \else\ifx\somestring\PSgrestore
 \else\ifx\somestring\PSfill
  \ifnum\finiMPpath=0
   \chardef\finiMPpath=2 \let\handleMPsequence=\processMPpath \fi
 \else\ifx\somestring\PSstroke
  \ifnum\finiMPpath=0 \chardef\finiMPpath=1 \fi
  \let\handleMPsequence=\processMPpath
 \else\ifx\somestring\PSclosepath \def\closeMPpath{\pdfliteral{h}}%
 \else\ifx\somestring\PSconcat
  \let\flushMPpath=\flushconcatMPpath \handleMPpathconcat
 \else\ifx\somestring\PSscale
  \let\flushMPpath\flushconcatMPpath \handleMPpathscale
 \fi\fi\fi\fi\fi\fi\fi\fi\fi\fi\fi\fi\fi
 \handleMPsequence}

% The main conversion command

\def\MPtoPDF#1{\bgroup
 \setMPspecials\catcode`\^^M=10\startMPscanning\def\do{}%
 \setbox0=\vbox\bgroup
  \parskip=0pt \leftskip=0pt \parindent=0pt \everypar{}\offinterlineskip
  \pdfliteral{q}\let\handleMPsequence=\dohandleMPsequence
  \input #1 \relax}

\def\finishMPgraphic{\pdfliteral{Q}\egroup\box0\egroup}

\endinput
