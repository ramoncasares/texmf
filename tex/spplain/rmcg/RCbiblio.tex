% RCbiblio.tex (RMCG20170126)

\hyphenation{pro-p-o-si-tions Ent-schei-dungs-prob-lem}

\reference Abelson \& Sussman (1985)
Harold Abelson, and Gerald Sussman, with Julie Sussman,
\book{Structure and In\-terpretation of Computer Programs};
The MIT Press, Cambridge MA, 1985,
\ISBN 978-0-262-01077-1.

\reference ASCII (1963)
``American Standard Code for Information Interchange'';
ASA X3.4-1963,
American Standards Association, New York, NY, 1963.

\reference Barendregt (1985)
Henk P.\ Barendregt,
\book{The Lambda Calculus, its Syntax and Semantics};
Revised Edition, Studies in Logic and the Foundations of Mathematics,
Vol.\ 103, North-Holland Publishing Co, Amsterdam, 1985,
\ISBN 0-444-87508-5.

\reference Bell (1964)
John Bell,
``On the Einstein Podolsky Rosen Paradox'';
in \periodical{Physics},
vol.\ 1, pp.\ 195--200, 1964,
\DOIx{10.1142/9789812386540\_0002}{https://doi.org/10.1142/9789812386540_0002}.

\reference Berwick \& Chomsky (2016)
Robert C.\ Berwick, and Noam Chomsky,
\book{Why Only Us: Language and Evolution};
The MIT Press, Cambridge MA, 2016,
\ISBN 978-0-262-03424-1.

\reference Bickerton (1990)
Derek Bickerton,
\book{Language \& Species};
The University of Chicago Press, Chicago, 1990,
\ISBN 978-0-226-04611-2.

\reference Bickerton (2009)
Derek Bickerton,
\book{Adam's Tongue: How Humans Made Language, How Language Made Humans};
Hill and Wang, New York, 2009,
\ISBN 978-0-8090-1647-1.

\reference Bickerton (2014)
Derek Bickerton,
\book{More than Nature Needs: Language, Mind, and Evolution};
Harvard University Press, Cambridge MA, 2014,
\ISBN 978-0-674-72490-7.

\reference Bohm (1952)
David Bohm,
``A Suggested Interpretation of the Quantum Theory
in Terms of `Hidden' Variables''; two parts
in \periodical{Physical Review},
vol.\ 85, pp.\ 166--179 and 180--193, January 1952,
\DOI{10.1103/PhysRev.85.166} and\\
\DOI{10.1103/PhysRev.85.180}.

\reference Casares (2016a)
Ram\'on Casares,
``Proof of Church's Thesis'';
\URL{\tt arXiv:1209.5036}<http://arxiv.org/abs/1209.5036>.

\reference Casares (2016b)
Ram\'on Casares,
``Problem Theory'';
\URL{\tt arXiv:arXiv:1412.1044}<http://arxiv.org/abs/arXiv:arXiv:1412.1044>.

\reference Casares (2016c)
Ram\'on Casares,
``Syntax Evolution: Problems and Recursion'';\\
\URL{\tt arXiv:1508.03040}<http://arxiv.org/abs/arXiv:1508.03040>.

\reference Chomsky (1959)
Noam Chomsky,
``On Certain Formal Properties of Grammars'';\\
in \periodical{Information and Control},
vol.\ 2, no.\ 2, pp.\ 137--167, June 1959,\\
\DOI{10.1016/S0019-9958(59)90362-6}.

\reference Chomsky (1965)
Noam Chomsky,
\book{Aspects of the Theory of Syntax};
The MIT Press, Cambridge MA, 1965,
\ISBN 978-0-262-53007-1.

\reference Chomsky (1986)
Noam Chomsky,
\book{Knowledge of Language: Its Nature, Origin, and Use},
Convergence series;
Praeger, New York, 1986,
\ISBN 978-0-275-91761-6.

\reference Chomsky (1988)
Noam Chomsky,
\book{Language and Problems of Knowledge: The Managua Lectures},
Current Studies in Linguistics, 16;
The MIT Press,
Cambridge, Massachu\-setts, 1988,
\ISBN 978-0-262-53070-5.

\reference Chomsky (2000)
Noam Chomsky,
\book{New Horizons in the Study of Language and Mind};
Cambridge University Press, Cambridge, 2000,
\ISBN 978-0-521-65822-5.

\reference Chomsky (2005a)
Noam Chomsky,
``Some Simple Evo-Devo Theses:
 How True Might They Be for Language?'';
Paper of the talk given to the
Morris Symposium on the Evolution of Language,
held at Stony Brook University, New York,
in October 15, 2005;
{\sc url:}
\URL<https://linguistics.stonybrook.edu/events/morris/05/program>.

\reference Chomsky (2005)
Noam Chomsky,
``Three Factors in Language Design'';
in \periodical{Linguistic Inquiry},
vol.\ 36, no.\ 1, pp.\ 1--22, Winter 2005,
\DOI{10.1162/0024389052993655}.

\reference Chomsky (2006)
Noam Chomsky,
\book{Language and Mind}, Third Edition;
Cambridge University Press, Cambridge, 2006,
\ISBN 978-0-521-67493-5.

\reference Chomsky (2007)
Noam Chomsky,
``Of Minds and Language'';
in \periodical{Biolinguistics},
vol.\ 1, pp.\ 9--27, 2007,\\
{\sc url:}
\URL<http://www.biolinguistics.eu/index.php/biolinguistics/article/view/19>.

% \reference Chomsky (2010)
% Noam Chomsky,
% ``Some simple evo devo theses:
%   How true might they be for language?'';
% in \book{The Evolution of Human Language: Biolinguistic Perspectives},
% R.~Larson, V.~D\'eprez, H.~Yamakido (editors),
% pp.\ 45--62,
% Cambridge University Press, Cambridge, 2010,
% \ISBN 978-0-521-73625-1.

\reference Church (1935)
Alonzo Church,
``An Unsolvable Problem of Elementary Number Theory'';
in \periodical{American Journal of Mathematics},
vol.\ 58, no.\ 2, pp.\ 345--363, April 1936,
\DOI{10.2307/2371045}.
Presented to the American Mathematical Society,
April 19, 1935.

\reference Curry \& Feys (1958)
Haskell B.\ Curry, and
Robert Feys, with
William  Craig,
\book{Combinatory Logic}, Vol.\ I;
North-Holland, Amsterdam, 1958,
\ISBN 978-0-7204-2207-8.

\reference Davies \& Brown (1986)
Paul Davies and Julian R. Brown (editors),
\book{The Ghost in the Atom:
 A Discussion of the Mysteries of Quantum Physics};
Cambridge University Press, Cambridge, 1986,
\ISBN 0-521-45728-9.

\reference Davis (1965)
Martin Davis (editor),
\book{The Undecidable: Basic Papers on Undecidable Propositions,
Unsolvable Problems and Computable Functions};
Dover, Mineola, New York, 2004,
\ISBN 978-0-486-43228-1.
Corrected republication of the same title
by Raven, Hewlett, New York, 1965.

\reference Davis (1982)
Martin Davis,
``Why G\"odel Didn't Have Church's Thesis'';
in \periodical{Information and Control},
vol.\ 54, pp.\ 3--24, 1982,
\DOI{10.1016/s0019-9958(82)91226-8}.

\reference DeLong (1970)
Howard DeLong,
\book{A Profile of Mathematical Logic};
Dover, Mineola, New York, 1998,
\ISBN 0-486-43475-3.

\reference Deutsch (1985)
David Deutsch,
``Quantum theory, the Church-Turing principle and
  the universal quantum computer'';
in \periodical{Proceedings of the Royal Society A},
 vol.\ 400, pp.\ 97--117, 1985,
\DOI{10.1098/rspa.1985.0070}.

\reference Everett (2008)
Daniel L.\ Everett,
\book{Don't Sleep, There Are Snakes:
Life and Language in the Amazonian Jungle};
Vintage, New York, 2008,
\ISBN 978-0-307-38612-0.

\reference  Fitch, Hauser, and Chomsky (2005)
Tecumseh Fitch, Marc Hauser, and Noam Chomsky,
``The Evolution of the Language Faculty:
Clarifications and Implications'';
in \periodical{Cognition},
vol.\ 97, pp.\ 179--210, 2005,
\DOI{10.1016/j.cognition.2005.02.005}.
% Received 5 November 2004; accepted 15 February 2005.

\reference Fortran (1956)
International Business Machines Corporation,
``The FORTRAN Automatic Coding System for the IBM 704 EDPM'';
International Business Machines Corporation, New York,
Programmer's Reference Manual, October 15, 1956.

\reference Fortran (1966)
American National Standards Institute,
``FORTRAN: USAS X3.9-1966'';
United States of America Standards Institute, New York, NY, 1966.

\reference Fortran (1977)
International Organization for Standardization,\\
``FORTRAN: ISO 1539:1980'';
\URL<https://www.iso.org/standard/6127.html>.

\reference Fortran (1990)
International Organization for Standardization,\\
``Fortran: ISO/IEC 1539:1991'';
\URL<https://www.iso.org/standard/17366.html>.

\reference Friedman \& Felleisen (1987)
Daniel Friedman, and Matthias Felleisen,
\book{The Little LISPer}, Trade Edition;
The MIT Press, Cambridge, Massachusetts, 1987,
\ISBN 978-0-262-56038-2.

\reference Gandy (1980)
Robin Gandy,
``Church's Thesis and Principles for Mechanisms'';
\DOI{10.1016/s0049-237x(08)71257-6}.
In \book{The Kleene Symposium}
(editors: J.\ Barwise, H.J.\ Keisler \& K.\ Kunen),
Volume 101 of
Studies in Logic and the Foundations of Mathematics;
North-Holland, Amsterdam, 1980, pp.~123--148;
\ISBN 0-444-85345-6.

\reference Ginsburg (2016)
Jason Ginsburg,
``Modeling of problems of projection:
 A non-counter\-cyclic approach'';
in \periodical{Glossa},
vol.\ 1, no.\ 1, art.\ 7, pp.\ 1--46, 2016,
\DOI{10.5334/gjgl.22}.

\reference G\"odel (1930)
Kurt G\"odel,
``\"Uber formal unentscheidbare S\"atze der Principia Mathematica
 und verwandter Systeme I'';
in \periodical{Monatshefte f\"ur Mathematik und Physik},
vol.\ 38, pp.\ 173--198, 1931,
\DOI{10.1007/BF01700692}.
Received November 17, 1930.
English translation in \cite Davis (1965).

\reference G\"odel (1946)
Kurt G\"odel,
``Remarks before the Priceton Bicentennial Conference on
  Problems in Mathematics'';
in \cite Davis (1965), pp.\ 84--88.

\reference Gold (1967)
Mark Gold,
``Language Identification in the Limit'';\\
in \periodical{Information and Control},
Volume 10, Issue 5, pp.~447--474, May 1967,\\
\DOI{10.1016/S0019-9958(67)91165-5}.

\reference Hamblin (1973)
Charles Hamblin,
``Questions in Montague English'';
in \periodical{Foundations of Language},
vol.\ 10, pp.\ 41--53, 1973.

\reference Hauser, Chomsky, and Fitch (2002)
Marc Hauser, Noam Chomsky, and Tecumseh Fitch,
``The Language Faculty:
 Who Has It, What Is It, and How Did It Evolved?'';
in \periodical{Science} 298, pp.\ 1569--1579, 2002,
\DOI{10.1126/science.298.5598.1569}.

\reference Hilbert (1900)
David Hilbert.
``Mathematical Problems'';
Lecture delivered before the international
congress of mathematicians at Paris in 1900.
In \periodical{Bulletin of the American Mathematical Society},
vol.\ 8, no.\ 10, pp.\ 437--479, July 1902.
Translated by Dr.\ Mary Winston Newson.
\DOI{10.1090/S0002-9904-1902-00923-3}.

\reference Jackendoff (2011)
Ray Jackendoff,
``What Is the Human Language Faculty? Two Views'';
in \periodical{Language},
vol.\ 87, no.\ 3, pp.\ 586--624, September 2011,
\DOI{10.1353/lan.2011.0063}.

\reference Kenneally (2007)
Christine Kenneally,
\book{The First Word: The Search for the Origins of Language};
Penguin Books, New York, 2008,
\ISBN 978-0-14-311374-4.

\reference Kleene (1935)
Stephen Kleene,
``General Recursive Functions of Natural Numbers'';
in \periodical{Mathematische Annalen},
vol.\ 112, no.\ 1, pp.\ 727--742, December 1936,\\
\DOI{10.1007/BF01565439}.
Presented to the American Mathematical Society,
September 1935.

\reference Kleene (1936)
Stephen Kleene,
``$\lambda$-Definability and Recursiveness'';
in \periodical{Duke Mathematical Journal},
vol.\ 2, pp.\ 340--353, 1936,
\DOI{10.1215/s0012-7094-36-00227-2}.

\reference Kleene (1943)
Stephen Kleene,
``Recursive Predicates and Quantifiers'';
in \periodical{Transactions of the American Mathematical Society},
vol.\ 53, no.\ 1, pp.\ 41--73, January 1943,
\DOI{10.2307/1990131}.

\reference Kleene (1952)
Stephen Kleene,
\book{Introduction to Meta-Mathematics};
Ishi Press, New York, 2009,
\ISBN 978-0-923891-57-2.
Reprint of the same title by
North-Holland, Amsterdam, 1952.

\reference Krifka (2011)
Manfred Krifka,
``Questions'',
\DOI{10.1515/9783110255072.1742};\\
in \book{Semantics:
 An International Handbook of Natural Language Meaning},
 Volume 2, HSK 33.2,
 K.\ von Heusinger, C.\ Maienborn, P.\ Portner (editors),
pp.\ 1742--1785,
De Gruyter Mouton, Berlin/Boston, 2011,
\ISBN 978-3-11-018523-2.

\reference Marr (1982)
David Marr,
\book{Vision: A Computational Investigation into
 the Human Representation and Processing of Visual Information};
W.H.\ Freeman and Company, San Francisco, CA, 1982,
\ISBN 0-7167-1284-9.

\reference McCarthy (1960)
John McCarthy,
``Recursive Functions of Symbolic Expressions
and Their Computation by Machine, Part I'';
in \periodical{Communications of the ACM},
vol.\ 3, no.\ 4, pp.\ 184--195, April 1960,
\DOI{10.1145/367177.367199}.

\reference McCarthy et al. (1962)
John McCarthy, Paul Abrahams, Daniel Edwards,
Timothy Hart, and Michael Levin,
 \book{LISP 1.5 Programmer's Manual};
The MIT Press,
Cambridge, Massachusetts, 1962,
\ISBN 978-0-262-13011-0.

\reference McCulloch \& Pitts (1943)
Warren McCulloch and Walter Pitts,
``A Logical Calculus of the Ideas Immanent in Nervous Activity'';
in \periodical{Bulletin of Mathematical Biophysics},
vol.\ 5, no.\ 4, pp.\ 115--133, 1943,
\DOI{10.1007/BF02478259}.

\reference Miller (1956)
 George Miller,
``The Magical Number Seven, Plus or Minus Two:
Some Limits on Our Capacity for Processing Information'';
in \periodical{Psychological Review},
vol.\ 63, No.\ 2, pp.\ 81--97, 1956,
\DOI{10.1037/h0043158}.

\reference Nelson (1987)
R.\ J.\ Nelson,
``Church's Thesis and Cognitive Science'';
in \periodical{Notre Dame Journal of Formal Logic},
vol.\ 28, no.\ 4, pp.\ 581--614, October 1987,\\
\DOI{10.1305/ndjfl/1093637649}.

\reference Olszewsky (2005)
Adam Olszewski,
``Church's thesis as an empirical hypothesis'';
in \periodical{Annales UMCS Informatica},
vol.\ AI 3, pp.\ 119--130, 2005,
\DOI{10.17951/ai.2005.3.1.119-130},
\URL<https://journals.umcs.pl/ai/article/view/3013>.

\reference Pavlov (1927)
Ivan Pavlov,
\book{Conditioned Reflexes:
 An Investigation of the Physiological Activity
 of the Cerebral Cortex},
translated by G.V.\ Anrep;
Oxford University Press, London, 1927.
Reprinted by Dover, Minneola, NY, 1960, 2003;
\ISBN 0-486-43093-6.

\reference Penrose (1989)
Roger Penrose, \negthinspace
\book{The Emperor's New Mind: \negthinspace
Concerning Computers, Minds, and the Laws of Physics};
Oxford University Press, Oxford, 1989,
\ISBN 0-19-851973-7.

\reference Pierce (1867)
Charles Sanders Peirce,
``On a New List of Categories'';
in \periodical{Proceedings of the American Academy
of Arts and Sciences},
vol.\ 7, pp.\ 287--298, 1868,
\DOI{10.2307/20179567}.
Presented May 14, 1867.

\reference Pinker \& Jackendoff (2004)
Steven Pinker, and Ray Jackendoff,
``The Faculty of Language: What’s Special About It?'';
in \periodical{Cognition},
vol.\ 95, pp.\ 201--236, 2005,
\DOI{10.1016/j.cognition.2004.08.004}.
Received 16 January 2004; accepted 31 August 2004.

\reference Post (1936)
Emil Post,
``Finite Combinatory Processes --- Formulation 1'';
in \periodical{The Journal of Symbolic Logic},
Volume\ 1, Number\ 3, pp.~103--105, September 1936.
Received October 7, 1936.
\DOI{10.2307/2269031}.

\reference Post (1944)
Emil L.\ Post,
``Recursively Enumerable Sets of Positive Integers
  and their Decision Problems'';
in \periodical{Bulletin of the American Mathematical Society},
vol.\ 50, no.~5, pp.\ 284--316, 1944,
\DOI{10.1090/s0002-9904-1944-08111-1}.

\reference Progovac (2016)
Ljiljana Progovac,
``A Gradualist
Scenario for Language Evolution:
Precise Linguistic Reconstruction of
Early Human (and Neandertal) Grammars'';
in \periodical{Frontiers in Psychology},
vol.\ 7, art.\ 1714, 2016,
\DOI{10.3389/fpsyg.2016.01714}.

\reference Rosser (1936)
J.\ Barkley Rosser,
``Extensions of Some Theorems of G\"odel and Church'';
in \periodical{Journal of Symbolic Logic},
vol.\ 1, pp.\ 87--91, 1936,
\DOI{10.2307/2269028}.

\reference Searle (1980)
John Searle,
``Minds, Brains and Programs'';
in \periodical{Behavioral and Brain Sciences},
vol.\ 3, no.\ 3, pp.\ 417--457,
\DOI{10.1017/S0140525X00005756}.

\reference Shieber (1986)
Stuart Shieber,
``An Introduction to Unification-Based Approaches to Grammar'';
Microtome Publishing, Brookline, Massachusetts, 2003.
Reissue of the same title by
CSLI Publications, Stanford, California, 1986.\\
{\sc url:} \URL<http://nrs.harvard.edu/urn-3:HUL.InstRepos:11576719>.

\reference Simon \& Newell (1971)
Herbert Simon and Allen Newell,
``Human Problem Solving: The State of the Theory in 1970'';
in \periodical{American Psychologist},
 vol.\ 26, no.\ 2, pp.\ 145--159, February 1971,
\DOI{10.1037/h0030806}.

\reference Stabler (2014)
Edward Stabler,
``Recursion in Grammar and Performance'',\\
\DOIx{10.1007/978-3-319-05086-7\_8}%
{http://dx.doi.org/10.1007/978-3-319-05086-7_8};
in \book{Recursion: Complexity in Cognition},
Volume 43 of the series `Studies in Theoretical Psycholinguistics',
Tom Roeper, Margaret Speas (editors),
pp.\ 159--177,
Springer, Cham, Switzerland, 2014,
\ISBN 978-3-319-05085-0.

\reference Turing (1936)
Alan Turing,
``On Computable Numbers,
 with an Application to the Entscheidungsproblem'';
in \periodical{Proceedings of the London Mathematical Society},
vol.\ s2-42, no.\ 1, pp.\ 230--265, 1937,
\DOI{10.1112/plms/s2-42.1.230}.
Received 28 May, 1936. Read 12 November, 1936.

\reference Turing (1937)
Alan Turing,
``Computability and $\lambda$-Definability''; in
\periodical{The Journal of Symbolic Logic},
vol.\ 2, no.\ 4, pp.\ 153--163, December 1937,
\DOI{10.2307/2268280}.

\reference Vygotsky (1934)
Lev Vygotsky,
\book{Thought and Language},
newly revised and edited by Alex Kozulin;
The MIT Press, Cambridge, Massachusetts, 1986,
\ISBN 978-0-262-72010-6.

\reference Watumull et al. (2014)
Jeffrey Watumull, Marc Hauser, Ian Roberts \& Norbert Hornstein,
``On Recursion''; in
\periodical{Frontiers in Psychology},
vol.\ 4, article 1017, pp.\ 1--7, 2014,
\DOI{10.3389/fpsyg.2013.01017}.

\reference Zermelo (1908)
Ernst Zermelo,
``Untersuchungen \"uber die Grundlagen der Mengenlehre I'',
\DOIx{10.1007/978-3-540-79384-7\_6}%
{http://dx.doi.org/10.1007/978-3-540-79384-7_6};
in \book{Ernst Zermelo Collected Works} Volume I,
H.-D.\ Ebbinghaus, A.\ Kanamori (editors),
pp.\ 160--229,
Springer-Verlag, Berlin Heidelberg, 2010,
\ISBN 978-3-540-79383-0.

\endinput
