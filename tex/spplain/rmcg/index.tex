% index.tex (RMCG19931103)

\catcode`\@=11

\catcode`\^^02=12 \catcode`\^^03=12 \catcode`\^^04=12 \catcode`\^^05=12

% Check files auxiliar.aux, auxiliar.ind

\newread\infile

\newif\ifauxf
\openin\infile=auxiliar.aux
\ifeof\infile \auxffalse \else \auxftrue \fi
\closein\infile

\newif\ifindf
\openin\infile=auxiliar.ind
\ifeof\infile \indffalse \else \indftrue \fi
\closein\infile

% pdftex

\ifx\pdfoutput\undefined \csname newcount\endcsname\pdfoutput \fi

% \color{Red} ... \endcolor

\def\cmykWhite{0 0 0 0} \def\cmykDarkWhite{0 0 0 .3}
\def\cmykGray{0 0 0 .5} \def\cmykDarkGray{0 0 0 .7}
\def\cmykBlack{0 0 0 1} \let\cmykDarkBlack=\cmykBlack
\def\cmykYellow{0 0 1 0} \def\cmykDarkYellow{0 0 1 .7}
\def\cmykMagenta{0 1 0 0} \def\cmykDarkMagenta{0 1 0 .7}
\def\cmykCyan{1 0 0 0} \def\cmykDarkCyan{1 0 0 .7}
\def\cmykRed{0 1 1 0} \def\cmykDarkRed{0 1 1 .7}
\def\cmykGreen{1 0 1 0} \def\cmykDarkGreen{1 0 1 .7}
\def\cmykBlue{1 1 0 0} \def\cmykDarkBlue{1 1 0 .7}

\newcount\col@rstack
\def\pdf@color#1{\pdfcolorstack\col@rstack push {\csname cmyk#1\endcsname k}}
 \def\pdf@darkcolor#1{\pdfcolorstack\col@rstack push {\csname cmykDark#1\endcsname k}}
 \def\pdf@endcolor{\pdfcolorstack\col@rstack pop}
\def\dvips@color#1{\special{color push cmyk \csname cmyk#1\endcsname}}
 \def\dvips@darkcolor#1{\special{color push cmyk \csname cmykDark#1\endcsname}}
 \def\dvips@endcolor{\special{color pop}}
\def\dvipdf@color#1{\special{pdf: bc [\csname cmyk#1\endcsname]}}
 \def\dvipdf@darkcolor#1{\special{pdf: bc [\csname cmykDark#1\endcsname]}}
 \def\dvipdf@endcolor{\special{pdf: ec}}

\ifcase\pdfoutput
 \ifx\dvips\undefined
  \def\colors{\let\color=\dvipdf@color\let\endcolor=\dvipdf@endcolor}
  \def\darkcolors{\let\color=\dvipdf@darkcolor\let\endcolor=\dvipdf@endcolor}
 \else
  \def\colors{\let\color=\dvips@color\let\endcolor=\dvips@endcolor}
  \def\darkcolors{\let\color=\dvips@darkcolor\let\endcolor=\dvips@endcolor}
 \fi
\else
 \col@rstack=\pdfcolorstackinit{\pdfliteral{0 0 0 1 k}}
 \def\colors{\let\color=\pdf@color\let\endcolor=\pdf@endcolor}
 \def\darkcolors{\let\color=\pdf@darkcolor\let\endcolor=\pdf@endcolor}
\fi

\def\nocolor{\def\color##1{\relax}\def\endcolor{\relax}}
\nocolor % default value

% type of links

\def\numlink#1{num \number#1}
\def\namelink#1{name {(\number#1)}}
\ifx\xmpinclude\undefined \let\typelink=\numlink \else \let\typelink=\namelink \fi

% \goto{Text in red}{123}

{\catcode`\#=12 \xdef\hash{#}}

\def\pdf@goto#1#2{\leavevmode
 \pdfstartlink attr{/F 4 /Border [0 0 0]} goto \typelink{#2}%
 \color{Red}#1\endcolor\pdfendlink\relax}
\def\dvips@goto#1#2{\leavevmode
 \special{html:<a href=\string"\hash\typelink{#2}\string">}%
 \color{Red}#1\endcolor\special{html:</a>}\relax}
\def\dvipdf@goto#1#2{\leavevmode
 \special{pdf: bann << /Type /Annot /Subtype /Link /Border [0 0 0]
   /A << /S /GoTo /D (\typelink{#2}) >> >>}\color{Red}#1\endcolor
 \special{pdf: eann}\relax}
\def\nolinks@goto#1#2{\leavevmode
 \closeout-1\color{Red}#1\endcolor\closeout-1\relax}
%\closeout-1 is just a whatsit, i.e. an extension as pdf commands are

% \dest

\newcount\d@stno

\def\pdf@dest{\global\advance\d@stno1
 \ifvmode \pdfdest \typelink{\d@stno} fitbh\else
  \vbox to0pt{\vss\pdfdest \typelink{\d@stno} fitbh\kern1pc}\fi}
\def\dvips@dest{\global\advance\d@stno1
 \ifvmode \special{html:<a name=\string"\typelink{\d@stno}\string">}
          \special{html:</a>}\else
  \vbox to0pt{\vss\special{html:<a name=\string"\typelink{\d@stno}\string">}
                  \special{html:</a>}\kern1pc}\fi}
\def\dvipdf@dest{\global\advance\d@stno1
 \ifvmode
  \special{pdf: dest (\typelink{\d@stno}) [@thispage /FitBH @ypos]}\else
  \vbox to0pt{\vss
   \special{pdf: dest (\typelink{\d@stno}) [@thispage /FitBH @ypos]}\kern1pc}\fi}
\def\nolinks@dest{\global\advance\d@stno1
 \ifvmode \closeout-1\else \vbox to0pt{\vss\closeout-1\kern1pc}\fi}

\ifcase\pdfoutput
 %\def\pdfcode#1\pdfendcode{\relax\closeout-1\relax}
 \ifx\dvips\undefined
  \def\links{\let\goto=\dvipdf@goto\let\dest=\dvipdf@dest}
 \else
  \def\links{\let\goto=\dvips@goto\let\dest=\dvips@dest}
 \fi
\else
 %\let\pdfcode=\relax \let\pdfendcode=\relax
 \def\links{\let\goto=\pdf@goto\let\dest=\pdf@dest}
\fi

\def\nolinks{\let\goto=\nolinks@goto\let\dest=\noliks@dest}
\links % default value

% \footnote

\def\footnoteoptions{} % \eightpt, for example
% adapted from [363]; parameter #2 (the text) is read later
\def\footnote#1{\let\@sf\empty
  \ifhmode\edef\@sf{\spacefactor\the\spacefactor}\/\fi
  \global\advance\d@stno1 \goto{#1}{\number\d@stno}\global\advance\d@stno-1% RMCG
  \@sf\vfootnote{#1}}
\def\vfootnote#1{\insert\footins\bgroup
  \interlinepenalty\interfootnotelinepenalty
  \splittopskip\ht\strutbox % top baseline for broken footnotes
  \splitmaxdepth\dp\strutbox \floatingpenalty\@MM
  \leftskip\z@skip \rightskip\z@skip \spaceskip\z@skip \xspaceskip\z@skip
  \footnoteoptions \textindent{#1}\footstrut\dest\futurelet\next\fo@t} % RMCG

% \pdf@ut calculates the number of branches in \@utno

\newtoks\pdft@k \pdft@k{}
\def\pdft@toc#1{\global\pdft@k=\expandafter{\the\pdft@k#1}}
\newcount\@utno \newcount\@uts
\newcount\@utss \newcount\@utt
\def\c@rcdr#1#2,{\edef\car{#1}\edef\cdr{#2}\ifx\cdr\empty\edef\cdr{0}\fi}
\def\pdf@ut{\expandafter\c@rcdr\the\pdft@k,\@uts=\car
 \global\pdft@k=\expandafter{\cdr}\toks0=\expandafter{\cdr}%
 \@utno=0 \@utss=\car \advance\@utss1
 \loop \expandafter\c@rcdr\the\toks0,\@utt=\car\toks0=\expandafter{\cdr}%
  \ifnum\@uts<\@utt \ifnum\@utss=\@utt\advance\@utno1\fi \repeat}

% Indexing

\def\ref#1{??} \def\refsc#1{??} \def\refpg#1{??}
% \toc{level}{text} \pdftocline{text}{\d@stno}
\let\toc=\gobbletwo \let\pdftocline=\gobbletwo
% \lbl{mnemonic}{replacement} \ndx{text}{type}
\let\lbl=\gobbletwo \let\ndx=\gobbletwo

% \pdfout{level}{title}{secid}
\def\pdfout#1#2#3{\if*#3#2\else#3 #2\fi}
\def\pdfoutA#1#2#3{#3 #2}
\def\pdfoutB#1#2#3{#2}

\def\pdf@outline#1#2#3#4#5{\ifnum#1<9
 \def\4{#4}\ifx\4\empty \let\pdfout=\pdfoutB \else \let\pdfout=\pdfoutA \fi
  {\stringate\doaccents\dosymbols\def\S{^^a7}\stringaccents \pdf@ut
   \pdfoutline goto \typelink{#3} count \number\@utno
               {\pdfout{#1}{#2}{#4}}}\fi}
\def\dvips@outline#1#2#3#4#5{\relax}
\def\dvipdf@outline#1#2#3#4#5{\ifnum#1<9
 \def\4{#4}\ifx\4\empty \let\pdfout=\pdfoutB \else \let\pdfout=\pdfoutA \fi
 {\stringate\doaccents\dosymbols\def\S{^^a7}\stringaccents \pdf@ut
  \special{pdf: out #1 << /Title (\pdfout{#1}{#2}{#4})
   /Dest (\typelink{#3}) >>}}\fi}

\def\fourdigits#1{\ifnum#1<0 \errmessage{min is 0}\else
            \ifnum#1>9999 \errmessage{max is 9999}\else
 \ifnum#1<1000 0\fi \ifnum#1<100 0\fi \ifnum#1<10 0\fi\fi\fi \number#1}
\ifx\secid\undefined \let\secid\empty\fi

\def\save#1{{\def\\{\string\\}\def~{\string~}\def\S{\string\S}\stringaccents
  \stringate\dohigh \deactivate\dohigh \utf\utf@Ch \let^^c2\empty
  \edef\next{\write\auxf{#1\string{\fourdigits\d@stno\string}%
    \string{\secid\string}\string{\noexpand\folio\string}}}\next}}

\newwrite\auxf \newwrite\tocf \newwrite\ndxf

\def\files{\begingroup \stringate\dohigh\stringaccents \let^^c2\empty
 \def~{\string~}\def\\{\string\\}\def\S{\string\S}\def\&{\string\&}% \catcode`\@=11
 \def\ftoc##1##2##3##4##5{\ifnum##1<9\pdft@toc{##1}\fi
  \immediate\write\tocf{\string\tocline
   \string{##1\string}\string{##2\string}%
   \string{##3\string}\string{##4\string}\string{##5\string}}}
 \def\fndx##1##2##3##4##5{\immediate\write\ndxf{\string\ndxline
  \string{##1\string}\string{##2\string}%
  \string{##3\string}\string{##4\string}\string{##5\string}}}
 \def\flbl##1##2##3##4##5{\setbox0=\hbox{\csname ^^03##1\endcsname}%
  \ifdim\wd0=0pt
   \expandafter\gdef\csname ^^02##1\endcsname{##2}% replacement
   \expandafter\gdef\csname ^^03##1\endcsname{##3}% \d@stno
   \expandafter\gdef\csname ^^04##1\endcsname{##4}% \secid
   \expandafter\gdef\csname ^^05##1\endcsname{##5}% \folio
  \else\errmessage{label ##1 redefined}\fi}
 \ifauxf
  \immediate\openout\tocf=auxiliar.toc
  \immediate\openout\ndxf=auxiliar.ind
  \input auxiliar.aux
  \immediate\closeout\tocf \immediate\closeout\ndxf
  \pdft@toc0
  \ifcase\pdfoutput
   \ifx\dvips\undefined \let\tocline=\dvipdf@outline
   \else \let\tocline=\dvips@outline\fi
  \else \let\tocline=\pdf@outline \fi
  \input auxiliar.toc
 \fi
 \endgroup
 \openout\auxf=auxiliar.aux
 \def\ref{\r@f^^02}\def\refsc{\r@f^^04}\def\refpg{\r@f^^05}%
 \def\toc##1##2{\save{\string\ftoc\string{##1\string}\string{##2\string}}}%
 \def\lbl##1##2{\save{\string\flbl\string{##1\string}\string{##2\string}}}%
 \def\ndx##1##2{\save{\string\fndx\string{##1\string}\string{##2\string}}}%
}
% Note that \vbox{}someword or someword\vbox{} does not hyphenate someword.

\def\r@f#1#2{{\stringate\dohigh\stringaccents\def\&{\string\&}\utf\utf@Ch \let^^c2\empty
  \xdef\n@@{\string"#2\string" }\xdef\@@n{\csname ^^03#2\endcsname}%
  \expandafter\global\expandafter\let\expandafter\@n@\csname #1#2\endcsname}%
 \setbox0=\hbox{\@n@}\ifdim\wd0=0pt ?\ifauxf\errmessage{label \n@@ undefined}\fi
  \else\goto{\@n@}{\@@n}\fi}

\def\label#1#2{\dest\lbl{#1}{#2}}
\def\index#1#2{\dest\ndx{#1}{#2}}

\catcode`\@=12
