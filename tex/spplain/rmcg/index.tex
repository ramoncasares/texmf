% index.tex (RMCG19931103)

\catcode`\@=11

% Check files auxiliar.aux, auxiliar.ind

\newread\infile

\newif\ifauxf
\openin\infile=auxiliar.aux
\ifeof\infile \auxffalse \else \auxftrue \fi
\closein\infile

\newif\ifindf
\openin\infile=auxiliar.ind
\ifeof\infile \indffalse \else \indftrue \fi
\closein\infile

% PDFTeX

\ifx\pdfoutput\undefined \csname newcount\endcsname\pdfoutput \fi

\def\nocolor{\let\Green=\relax \let\Blue=\relax \let\Cyan=\relax
             \let\Red=\relax \let\Magenta=\relax \let\Yellow=\relax
             \let\Black=\relax \let\Gray=\relax \let\White=\relax}

\ifcase\pdfoutput
 \def\pdfcode#1\pdfendcode{\relax\closeout-1\relax}
 %\closeout-1 is just a whatsit, i.e. an extension as pdf commands are
 \let\colors=\nocolor
\else
 \let\pdfcode=\relax \let\pdfendcode=\relax
 \def\pdfGreen{\pdfliteral{1 0 1 0 k}}
 \def\pdfBlue{\pdfliteral{1 1 0 0 k}}
 \def\pdfCyan{\pdfliteral{1 0 0 0 k}}
 \def\pdfRed{\pdfliteral{0 1 1 0 k}}
 \def\pdfMagenta{\pdfliteral{0 1 0 0 k}}
 \def\pdfYellow{\pdfliteral{0 0 1 0 k}}
 \def\pdfBlack{\pdfliteral{0 0 0 1 k}}
 \def\pdfGray{\pdfliteral{0 0 0 0.50 k}}
 \def\pdfWhite{\pdfliteral{0 0 0 0 k}}
 \def\colors{\let\Green=\pdfGreen \let\Blue=\pdfBlue \let\Cyan=\pdfCyan
             \let\Red=\pdfRed\let\Magenta=\pdfMagenta\let\Yellow=\pdfYellow
             \let\Black=\pdfBlack \let\Gray=\pdfGray \let\White=\pdfWhite}
\fi

\nocolor % default value

% \pdfgoto{Text in red}{123}
\def\pdfgoto#1#2{\leavevmode\pdfcode
 \pdfstartlink attr{/F 4 /Border [0 0 0]} goto num #2\pdfendcode
 \Red#1\Black \pdfcode\pdfendlink\pdfendcode}

\newcount\d@stno

\def\pdflabel{\pdfcode \global\advance\d@stno1
 \ifvmode \pdfdest num \number\d@stno fitbh\else
  \vbox to0pt{\vss\pdfdest num \number\d@stno fitbh\kern1pc}\fi
 \pdfendcode}

% Footnotes for PDFTeX

\def\footnoteoptions{} % \eightpt, for example
% adapted from [363]; parameter #2 (the text) is read later
\def\footnote#1{\let\@sf\empty
  \ifhmode\edef\@sf{\spacefactor\the\spacefactor}\/\fi
  \global\advance\d@stno1 \pdfgoto{#1}{\number\d@stno}% RMCG
  \@sf\vfootnote{#1}}
\def\vfootnote#1{\insert\footins\bgroup
  \interlinepenalty\interfootnotelinepenalty
  \splittopskip\ht\strutbox % top baseline for broken footnotes
  \splitmaxdepth\dp\strutbox \floatingpenalty\@MM
  \leftskip\z@skip \rightskip\z@skip \spaceskip\z@skip \xspaceskip\z@skip
  \pdfcode\pdfdest num \number\d@stno fitbh\pdfendcode \footnoteoptions% RMCG
  \textindent{#1}\footstrut\futurelet\next\fo@t}

% \pdf@ut calculates the number of branches in \@utno

\newtoks\pdft@k \pdft@k{}
\def\pdft@toc#1{\global\pdft@k=\expandafter{\the\pdft@k#1}}
\newcount\@utno \newcount\@uts
\newcount\@utss \newcount\@utt
\def\c@rcdr#1#2,{\edef\car{#1}\edef\cdr{#2}\ifx\cdr\empty\edef\cdr{0}\fi}
\def\pdf@ut{\expandafter\c@rcdr\the\pdft@k,\@uts=\car
 \global\pdft@k=\expandafter{\cdr}\toks0=\expandafter{\cdr}%
 \@utno=0 \@utss=\car \advance\@utss1
 \loop \expandafter\c@rcdr\the\toks0,\@utt=\car\toks0=\expandafter{\cdr}%
  \ifnum\@uts<\@utt \ifnum\@utss=\@utt\advance\@utno1\fi \repeat}

% Indexing

\def\ref#1{??} \def\refsc#1{??} \def\refpg#1{??}
% \toc{level}{text} \pdftocline{text}{\d@stno}
\let\toc=\gobbletwo \let\pdftocline=\gobbletwo
% \lbl{mnemonic}{replacement} \ndx{text}{type}
\let\lbl=\gobbletwo \let\ndx=\gobbletwo

% \pdfout{level}{title}{secid}
\def\pdfout#1#2#3{\if*#3#2\else #3 #2\fi}

\def\fourdigits#1{\ifnum#1<0 \errmessage{min is 0}\else
            \ifnum#1>9999 \errmessage{max is 9999}\else
 \ifnum#1<1000 0\fi \ifnum#1<100 0\fi \ifnum#1<10 0\fi\fi\fi \number#1}
\ifx\secid\undefined \let\secid\empty\fi

\def\save#1{{\def\\{\string\\}\def~{\string~}%
  \stringaccents \deactivate\dohigh \utf\utf@Ch
  \edef\next{\write\auxf{#1\string{\fourdigits\d@stno\string}%
    \string{\secid\string}\string{\noexpand\folio\string}}}\next}}

\newwrite\auxf \newwrite\tocf \newwrite\ndxf

\def\files{\begingroup \stringate\dohigh\stringaccents
 \def~{\string~}\def\\{\string\\}% \catcode`\@=11
 \def\ftoc##1##2##3##4##5{\ifnum##1<9\pdft@toc{##1}\fi
  \immediate\write\tocf{\string\tocline
   \string{##1\string}\string{##2\string}%
   \string{##3\string}\string{##4\string}\string{##5\string}}}
 \def\fndx##1##2##3##4##5{\immediate\write\ndxf{\string\ndxline
  \string{##1\string}\string{##2\string}%
  \string{##3\string}\string{##4\string}\string{##5\string}}}
 \def\flbl##1##2##3##4##5{\setbox0=\hbox{\csname ^^03##1\endcsname}%
  \ifdim\wd0=0pt
   \expandafter\gdef\csname ^^02##1\endcsname{##2}% replacement
   \expandafter\gdef\csname ^^03##1\endcsname{##3}% \d@stno
   \expandafter\gdef\csname ^^04##1\endcsname{##4}% \secid
   \expandafter\gdef\csname ^^05##1\endcsname{##5}% \folio
  \else\errmessage{label ##1 redefined}\fi}
 \ifauxf
  \immediate\openout\tocf=auxiliar.toc
  \immediate\openout\ndxf=auxiliar.ind
  \input auxiliar.aux
  \immediate\closeout\tocf \immediate\closeout\ndxf
  \pdft@toc0\pdfcode
  \def\tocline##1##2##3##4##5{\ifnum##1<9
   {\stringate\doaccents\dosymbols \stringaccents \pdf@ut
   \pdfoutline goto num ##3 count \number\@utno
               {\pdfout{##1}{##2}{##4}}}\fi}%
  \input auxiliar.toc
  \pdfendcode
 \fi
 \endgroup
 \openout\auxf=auxiliar.aux
 \def\ref{\r@f^^02}\def\refsc{\r@f^^04}\def\refpg{\r@f^^05}%
 \def\toc##1##2{\save{\string\ftoc\string{##1\string}\string{##2\string}}}%
 \def\lbl##1##2{\save{\string\flbl\string{##1\string}\string{##2\string}}}%
 \def\ndx##1##2{\save{\string\fndx\string{##1\string}\string{##2\string}}}%
}
% Note that \vbox{}someword or someword\vbox{} does not hyphenate someword.

\def\r@f#1#2{{\stringate\dohigh\stringaccents
  \xdef\n@@{\string"#2\string" }\xdef\@@n{\csname ^^03#2\endcsname}%
  \expandafter\global\expandafter\let\expandafter\@n@\csname #1#2\endcsname}%
 \setbox0=\hbox{\@n@}\ifdim\wd0=0pt ?\ifauxf\errmessage{label \n@@ undefined}\fi
  \else\pdfgoto{\@n@}{\@@n}\fi}

\def\label#1#2{\pdflabel\lbl{#1}{#2}}
\def\index#1#2{\pdflabel\ndx{#1}{#2}}

\catcode`\@=12
