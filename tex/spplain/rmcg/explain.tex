% EXPLAIN.TEX (RMCG19931021)

\catcode`\@=11

\def\uncatcodespecials{\deactivate\dospecials}
\def\uncatcodeall{\deactivate\dospecials\dohigh}

% \listing#1 prints file #1 contents verbatim
% \verb<char><text><char> writes <text> verbatim

\def\listing#1{\ifvmode\vskip\parskip\else\par\fi\begingroup
 \def~{\string~}\setupverbatim \input#1 \endgroup\condpar}
\long\def\condpar#1{\ifx#1\endgraf\else\vskip-\parskip\noindent#1\fi}
\def\verb{\begingroup\setupverbatim\d@verbatim}
\def\setupverbatim{\tt\parskip0pt plus0.1pt minus0.1pt
 \def\par{\leavevmode\endgraf}\catcode`\`=13
 \uccode`\~=`\ \uppercase{\let~=\ }\uccode`\~=`\`
 \uppercase{\def~{\relax\lq}}\uccode`\~=0
 \obeylines\uncatcodespecials\obeyspaces \verbatimoptions}
\def\d@verbatim#1{\def\next##1#1{##1\endgroup}\next}
\def\verbatimoptions{}
{\catcode`\`=13 \gdef`{\lq}} % [228], [254]
% From ``The \TeX book'', pg 380--382
%  For line numbering do:
%    \def\verbatimoptions{\parindent=60pt 
%     \everypar{\llap{\sevenrm\the\inputlineno\quad}}}
%  For an escape character do:
%    \def\verbatimoptions{\catcode`\\=0 }
%  For minimal markup do:
%    \catcode`\"=13 \def"{\verb"}

% \plisting#1
% #2
% #3
% prints file #1 contents verbatim 
% from the first line equal to #2
% to the next line equal to #3
\def\plisting#1{\par \begingroup
 \def~{\string~}\def\1{#1}\beforeplisting
 \def\d@next{\expandafter\pa@ss\input #1 }\setupverbatim\getm@rks}
{\obeylines\gdef\getm@rks^^M#1^^M#2^^M{%
  \gdef\beginm@rk{#1}\gdef\endm@rk{#2}\d@next}}
{\obeylines\gdef\pa@ss#1^^M{\def\next{#1}\ifx\next\beginm@rk %
  #1\par \let\next\pr@nt \else \let\next\pa@ss\fi \next}}
\def\endpr@nt{\afterplisting\endgroup\endinput}  
{\obeylines\gdef\pr@nt#1^^M{#1\par \def\next{#1}\ifx\next\endm@rk %
  \let\next\endpr@nt \else \let\next\pr@nt \fi \next}}
\let\beforeplisting=\relax \let\afterplisting=\relax

{\obeylines\gdef\getm@rk#1^^M{\def\endm@rk{#1}\d@next}}

% \comment[end mark]
%    ignores until the next [end mark] line

\def\comment{\begingroup
 \uncatcodeall\obeylines\let\d@next=\ign@re\getm@rk}
{\obeylines\gdef\ign@re#1
 {\def\next{#1}\ifx\next\endm@rk \let\next\endgroup \else %
  \let\next\ign@re\fi \next}}

% \filed filename \anyCS Title of Section
%    goes to the top of file filename, ignoring everything till a line
%    \anyCS Title of Section
%    and typesets from the next line to the next
%    \anyCS or \endanyCS command, 
%    and then returns to the current file.
\def\filed#1 {\begingroup 
 \def\d@next{\expandafter\ign@re\input #1 }%
 \begingroup\uncatcodeall\get@}
\catcode`\ =12%
\def\get@#1#2 {\endgroup\catcode`#1=0\def\end@{end#2}%
\expandafter\def\csname#2\endcsname##1\par{\endinput\endgroup}%
\expandafter\def\csname\end@\endcsname##1\par{\endinput\endgroup}%
\begingroup\uncatcodeall\obeylines\getm@rk#1#2 }
\catcode`\ =10%

% \tofile{<filename>}[eof mark]
%    writes verbatim to <filename> till the next [eof mark] line

\newwrite\auxfile
\def\tofile#1{\immediate\openout\auxfile=#1 \begingroup
 \uncatcodeall\obeyspaces\obeylines\let\d@next\copytom@rk\getm@rk}
{\obeylines\gdef\copytom@rk#1
 {\def\next{#1}\ifx\next\endm@rk %
   \def\next{\endgroup\immediate\closeout\auxfile}\else %
  \immediate\write\auxfile{\next}\let\next\copytom@rk\fi\next}}

% \fverb\write\file<char><text><char> or \wverb\file<char><text><char>
%    writes verbatim <text> in \openout'ed file
% \fverb\immediate\write\file<char><text><char> or
%    \iverb\file<char><text><char> idem immediatelly

\def\fverb#1\write{\ifx#1\immediate \let\next=\iverb
 \else \let\next=\wverb \fi \next}
\def\iverb#1{\begingroup\uncatcodeall\obeyspaces\obeylines\d@iverb#1}
\def\d@iverb#1#2{\def\next##1#2{\immediate\write#1{##1}\endgroup}\next}
\def\wverb#1{\begingroup\uncatcodeall\obeyspaces\obeylines\d@wverb#1}
\def\d@wverb#1#2{\def\next##1#2{\write#1{##1}\endgroup}\next}
% To include material in toc do:
%    \wverb\auxf"\iverb\tocf|material in toc|"

% \todayiso prints date format YYYYMMDD

\def\twodigits#1{\ifnum #1<10 0\fi \number#1}
\def\todayiso{\number\year \twodigits\month \twodigits\day}
\def\todayISO{\number\year.\twodigits\month.\twodigits\day}
% From ``The METAFONTbook'', pg 337

% \fecha prints date in Spanish

\def\fecha{\number\day\ de \ifcase\month\or
 enero\or febrero\or marzo\or abril\or mayo\or junio\or
 julio\or agosto\or septiembre\or octubre\or noviembre\or
 diciembre\fi\ de \number\year}

% \frameit#1 makes a frame around box #1
% \boxit#1 makes a frame 3pt around #1 (``The \TeX book'' pg 331)

\def\frameit#1{\vbox{\hrule\hbox{\vrule\vbox{#1}\vrule}\hrule}}
\def\boxit#1{\vbox{\hrule\hbox{\vrule\kern3pt
 \vbox{\kern3pt#1\kern3pt}\kern3pt\vrule}\hrule}}

% \Folio is like \folio but writing uppercase roman numerals

\def\Folio{\ifnum\pageno<0
 \uppercase\expandafter{\romannumeral-\pageno}\else\number\pageno\fi}

% Paragrafs

\def\subitem#1*#2*{\par \setbox0=\hbox{#2\enspace}% 
 \ifdim\parindent>0pt \dimen0=\parindent \else \dimen0=20pt \fi
 \ifdim\dimen0>\wd0 \setbox0=\hbox to\dimen0{\hfil#2\enspace}\fi
 \hangindent=#1\dimen0 
 {\parindent=\hangindent\advance\parindent-\dimen0\indent}% 
 \box0\ignorespaces}

\def\bulletedpar{\subitem*$\bullet$*}
\def\subbulletedpar{\subitem2*$\circ$*}
\def\titledpar#1:{\subitem*\bf#1\enspace*}

\def\indented#1par{\par \dimen0=20pt \ifdim\parindent>0pt 
 \hangindent=#1\parindent \else \hangindent=#1\dimen0 \fi
 \hangafter0{\divide\parskip4\noindent}\ignorespaces}
\def\unindentedpar{\par\hang\noindent}

\def\begincenter{\par\begingroup \parindent=0pt
 \advance\leftskip 0pt plus 1fill \advance\rightskip 0pt plus 1fill
 \def\\{\unskip\break\ignorespaces}}
\def\endcenter{\par\endgroup}

\def\beginnarrow{\par\begingroup
 \ifdim\parindent>0pt \dimen0=\parindent \else \dimen0=20pt \fi
 \advance\leftskip\dimen0 \advance\rightskip\dimen0 }
\def\endnarrow{\par\endgroup}

\def\beginlefted{\par\begingroup
 \ifdim\parindent>0pt \dimen0=\parindent \else \dimen0=20pt \fi
 \advance\leftskip\dimen0 }
\def\endlefted{\par\endgroup}

% Gobblers

\def\gobble#1{}
\def\gobbletwo#1#2{}

% Other macros

\def\newpage{\ifvmode\null\vfil\break\else\vadjust{\null\vfil\break}\fi}
\def\newline{\ifhmode\null\hfil\break\else\null\fi}

\hsize=15.92cm \vsize=24.62cm % DIN A4
\def\m@g{\mag=\count@
 \hsize=15.92truecm \vsize=24.62truecm \dimen\footins=12truecm}

\catcode`\@=12
