% RCstyle.tex (20141204)

\catcode`\^^c2=13
\ifx\files\undefined \input index \fi

\catcode`\@=11
\let\utf=\relax
\let\utf@Ch=\relax
\let\stringate=\relax
\let\dohigh=\relax
\let\doaccents=\relax
\let\dosymbols=\relax
\let\deactivate=\relax
\def\stringaccents{\def\'{\string\'}\def\~{\string\~}%
 \def\"{\string\"}\def\`{\string\`}\def\^{\string\^}}
\catcode`\@=12

\files

% Fonts

\ifx\loadfont\undefined \input fonts \fi
\xiifonts \xiititles \rm
\font\sf=cmss12
\font\sfit=cmssi12

\catcode`\@=11

\font\xiibb=msbm10 at12pt
\font\ixbb=msbm9
\font\vibb=msbm6
\newfam\bbfam % fam 8
\textfont\bbfam=\xiibb \scriptfont\bbfam=\ixbb
\scriptscriptfont\bbfam=\vibb
\def\bb{\fam\bbfam\xiibb}
%\mathchardef\beth="0869
\mathchardef\subsetneq="3828

\font\xiifrak=eufm10 at12pt
\font\ixfrak=eufm9
\newfam\frakfam \textfont\frakfam=\xiifrak \scriptfont\frakfam=\ixfrak
\def\frak{\fam\frakfam\xiifrak}

\mathcode`?="603F % ? works as punctuation
\def\ifmath$#1${\relax\ifmmode #1\else$#1$\fi}
\def\QED{\ifmath$\diamond$}

% Layout

\ifcase\pdfoutput
 \special{papersize=210mm,297mm}
\else
 \pdfcompresslevel=9
 \pdfdecimaldigits=4
 \pdfhorigin=1truein
 \pdfvorigin=1truein
 \pdfpageheight=297truemm
 \pdfpagewidth=210truemm
\fi
\hsize=15.92truecm \vsize=24.62truecm % DIN A4
\advance\vsize -30pt

\def\twodigits#1{\ifnum #1<10 0\fi \number#1}
\def\todayiso{\number\year \twodigits\month \twodigits\day}
\let\version=\todayiso
\def\Folio{\ifnum\pageno<0
 \uppercase\expandafter{\romannumeral-\pageno}\else\number\pageno\fi}

\headline={\vrule height10.2pt depth4.2pt width0pt
 {\xiitt www.ramoncasares.com}\quad{\xiirm\version}\hfil
 \quad{\xiibf\jobname\quad\Folio}}%
\def\makeheadline{\vbox to 30pt{\colorblack\line{\the\headline}%
  \kern 1pt \hrule height 1pt\vfil\endcolor}\nointerlineskip}
\nopagenumbers

% Sectioning

\newcount\secno
\newcount\ssecno
\newcount\thno
\newcount\parno
\let\presec=\empty
\def\presec{\S} % comment out for an empty one

\parskip=0pt plus 1pt
\newdimen\oldparindent \oldparindent=20pt \parindent=0pt
\def\hang{\hangindent\oldparindent}

\def\numberedpars{\global\advance\parno1 %\dest
 \noindent\hbox to\oldparindent{{\xiiscriptsy\char123
  \xiiscriptrm\number\parno}\hfil$\cdot$\hfil}\ignorespaces}

\def\begincenter{\par\begingroup \parindent=0pt
 \advance\leftskip 0pt plus 1fill \advance\rightskip 0pt plus 1fill
 \def\\{\unskip\break\ignorespaces}}
\def\endcenter{\par\endgroup}

\outer\def\section#1{\vskip0pt plus 90pt\penalty-500\vskip0pt plus-90pt
 \everypar{}\advance\secno1
 \ssecno=0 \thno=0 \parno=0
 \goodbreak\vskip 2pc plus 1pc
 \def\secid{\presec\the\secno}
 \noindent{\fonttwo\secid\ #1\dest\toc1{#1}\lbl{#1}{\secid}}\par
 \everypar{\numberedpars}}
\outer\def\xsection#1{\vskip0pt plus 90pt\penalty-500\vskip0pt plus-90pt
 \everypar{}\parno=0
 \goodbreak\vskip 2pc plus 1pc
 \def\secid{}
 \noindent{\fonttwo#1\dest\toc1{#1}\lbl{#1}{}}\par
 \everypar{\numberedpars}}

%\outer\def\annex#1{\def\presec##1##2{#1}} % no longer used

\outer\def\subsection#1{\vskip0pt plus 60pt\penalty-250\vskip0pt plus-60pt
 \everypar{}\advance\ssecno1 \par
 \ifnum\parno=0 \vskip 0.5pc plus 3pt minus 3pt\else\vskip 1pc plus 6pt minus 6pt\fi
 \thno=0 \parno=0
 \def\secid{\presec\the\secno.\the\ssecno}%
 \noindent{\fontthree\secid\ #1\dest\toc2{#1}\lbl{#1}{\secid}}\par
 \everypar{\numberedpars}}
\outer\def\clause#1{\vskip0pt plus 30pt\penalty-150\vskip0pt plus-30pt
 \everypar{}\medskip\hang\noindent
 \advance\thno1 \def\secid{\presec\the\secno.\the\ssecno.\the\thno}%
 {\sc\secid\ #1}\quad\ignorespaces}
\outer\def\comment#1{\everypar{}\par
 \hang\noindent{\sc #1}\quad\ignorespaces}

\def\note#1{\ifvmode
 \vtop to0pt{\vss\rlap{\color{White}\xiiscriptrm #1\endcolor}\kern-2pt\null}%
\else
 \vadjust{\vtop to0pt{\vss
 \rlap{\color{White}\xiiscriptrm #1\endcolor}\kern-2pt\null}}\fi}
%\def\note#1{\relax} % comment to write labels % uncomment to hide labels
\def\label#1{\note{#1}\dest\lbl{#1}{\secid}\ignorespaces}

% Openings

\def\title#1{\def\titledoc{#1}}
\def\author#1{\def\authordoc{#1}}
\def\contact#1{\def\contactdoc{#1}}
\def\keywords#1{\def\keywordsdoc{#1}}
\def\subject#1{\def\subjectdoc{#1}}

\def\beginnote{\insert\footins\bgroup\strut\colorblack}
\def\endnote{\endcolor\egroup}

\def\maketitle{\par
 \begingroup
 \def\'##1{\if##1o\string^^f3\else ##1\fi}% hack \'o -> f3
 \def\\{ }%
 \ifcase\pdfoutput
  \ifx\dvips\undefined
   \special{pdf: docinfo <<
    /Title (\titledoc) /Author (\authordoc)
    /Keywords (\keywordsdoc) /Subject (\subjectdoc) >>}%
  \else
   \special{ps: [
    /Title (\titledoc) /Author (\authordoc)
    /Keywords (\keywordsdoc) /Subject (\subjectdoc)
    /DOCINFO pdfmark }%
  \fi
 \else
  \pdfinfo{/Title (\titledoc) /Author (\authordoc)
    /Keywords (\keywordsdoc) /Subject (\subjectdoc)}\fi
 \endgroup
 \null\vskip 3pc plus 4pc minus 1pc
 \def\secid{}\dest\toc0{\titledoc}\lbl{\titledoc}{}
 \begincenter \fontzero\titledoc \endcenter
 \vskip 1pc plus 1pc
 \centerline{\fontone\authordoc}
 \ifx\contactdoc\undefined\else
  \vskip 6pt minus 3pt
  \centerline{\rm\contactdoc}\fi
 \vskip 2pc plus 1pc minus 1pc\relax}

\def\beginabstract{\begingroup \parindent=55pt\narrower \sl\parindent=20pt\noindent}
\def\endabstract{\par \ifx\keywordsdoc\undefined\else \sfit\rightskip=55pt plus3em
 \smallskip\setbox0=\hbox{Keywords:\quad}\hangindent\wd0
 \noindent\box0\keywordsdoc\par\fi \smallskip\endgroup}

% References

%% \cite Post (1944) -> Post (1944)
%% \cite* Post (1944) -> Post 1944
%% \cite[free \1's text \2] Post (1944) -> free Post's text 1944

\def\citenone #1 (#2){\ref{#1#2}}
\def\citetext [#1] #2 (#3){\refx{{\def\1{#2}\def\2{#3}#1}}{#2#3}}
\def\citestar * #1 (#2){\refx{#1\ #2}{#1#2}}

\def\cite{\futurelet\nexttoken\citexx}
\def\citexx{\if *\nexttoken \let\next=\citestar \else
 \if [\nexttoken \let\next=\citetext \else
 \let\next=\citenone \fi\fi\next}

\def\chreference #1 (#2){\everypar{}\par
 \vskip0pt plus 2\baselineskip\penalty-43
 \vskip0pt plus-2\baselineskip
 \noindent\hangindent20pt\relax
 #1\thinspace(#2)\dest\lbl{#1#2}{#1\ (#2)}}

\def\brreference #1 (#2){\everypar{}\par
 \vskip0pt plus 2\baselineskip\penalty-43
 \vskip0pt plus-2\baselineskip
 \noindent\hangindent20pt\relax
 [#1 #2]\dest\lbl{#1#2}{[#1 #2]}}

\let\reference=\chreference

\def\references{\vskip0pt plus 90pt\penalty-500\vskip0pt plus-90pt
 \everypar{}\parno=0
 \goodbreak\vskip 2pc plus 1pc
 \def\secid{}
 \noindent{\fonttwo References\dest\toc1{References}\lbl{References}{}}\par
 \begingroup
  \def\reference ##1 ({{\stringaccents\def\&{\string\&}\xdef\1{##1}}\next}
  \def\next ##1) ##2\par{
    \expandafter\gdef\csname \1##1\endcsname{##2}}
  \input RCbiblio
 \let\item=\bibitem \let\xitem=\xbibitem
 \everypar{}\smallskip \parskip=0pt \frenchspacing}
\def\endreferences{\par\endgroup}

\def\bibitem #1 (#2){\chreference #1 (#2):\enskip
 {\stringaccents\def\&{\string\&}\xdef\1{#1}}\csname \1#2\endcsname}
\def\xbibitem #1 (#2) = #3 (#4){\chreference #1 (#2):\enskip
 {\stringaccents\def\&{\string\&}\xdef\1{#3}}\csname \1#4\endcsname}

\def\book#1{{\it#1\/}}
\def\periodical#1{{\it#1\/}}
\def\ISBN{{\sc isbn: }}
\def\DOI#1{{\sc doi: }\URL{#1}<http://dx.doi.org/#1>}
\def\DOIx#1#2{{\sc doi: }\URL{#1}<#2>}

% Table of contents

\def\tocline#1#2#3#4#5{\par \ifcase#1
 \toclinezero{#2}{#3}{#4}{#5}\or
 \toclineone{#2}{#3}{#4}{#5}\or
 \toclinetwo{#2}{#3}{#4}{#5}\else
 \toclinethree{#2}{#3}{#4}{#5}\fi}

\def\gobblefour#1#2#3#4{\relax}
\def\leaderfill{\leaders\hbox to\baselineskip{\hss.\hss}\hfill}
\def\toclinezero#1#2#3#4{\centerline{\def\\{ }\goto{\fonttwo #1}{#2}}\bigskip}
\def\toclineone#1#2#3#4{\def\3{#3}%
 \line{\fontthree \ifx\3\empty\else #3 \fi#1\leaderfill \goto{#4}{#2}}}
\def\toclinetwo#1#2#3#4{\line{\qquad #3 #1\leaderfill \goto{#4}{#2}}}
\let\toclinethree=\gobblefour

% Differences

% The TeX Book Exercise 14.28
\def\strutdepth{\dp\strutbox}
\def\marginalstar{\strut\vadjust{\kern-\strutdepth\specialstar}}
\def\specialstar{\vtop to \strutdepth{
 \baselineskip\strutdepth
 \vss\llap{\truecolor{Red}\rm*\endcolor\quad}\null}}

\def\new/{\truecolor{Red}}
\def\wen/{\endcolor\ifhmode\marginalstar\fi}

\def\uncatcodespecials{\def\do##1{\catcode`##1=12 }\dospecials}
\def\del/{\begingroup\uncatcodespecials
 \ifvmode \let\next=\DELv \else \let\next=\DELh \fi\next}
{\catcode`\|=0 \catcode`\\=12
 |long|gdef|DELv#1\led/{|endgroup}
 |long|gdef|DELh#1\led/{|marginalstar|endgroup|ignorespaces}}
\def\led/{\errmessage{Error! Unnested led.}}

% Verbatim

% \verb<char><text><char> writes <text> verbatim

\def\uncatcodespecials{\def\do##1{\catcode`##1=12 }\dospecials}

\def\verb{\begingroup\setupverbatim\d@verbatim}
\def\setupverbatim{\tt\parskip0pt plus0.1pt minus0.1pt
 \everypar={}\def\par{\leavevmode\endgraf}\catcode`\`=13
 \uccode`\~=`\ \uppercase{\let~=\ }\uccode`\~=`\`
 \uppercase{\def~{\relax\lq}}\uccode`\~=0
 \obeylines\uncatcodespecials\obeyspaces \verbatimoptions}
\def\d@verbatim#1{\def\next##1#1{##1\endgroup}\next}
\def\verbatimoptions{}
{\catcode`\`=13 \gdef`{\lq}} % [228], [254]
% From ``The \TeX book'', pg 380--382
%  For line numbering do:
%    \def\verbatimoptions{\parindent=60pt
%     \everypar{\llap{\sevenrm\the\inputlineno\quad}}}
%  For an escape character do:
%    \def\verbatimoptions{\catcode`\\=0 }
%  For minimal markup do:
%    \catcode`\"=13 \def"{\verb"}


% Misc

\def\newline{\ifvmode\null\else\null\hfil\break\fi}
\let\\=\newline

\def\beginpoints{\begingroup\parskip=0pt\parindent=20pt\everypar{}}
\def\point{\item{$\circ$}}
\def\endpoints{\par\noindent\endgroup}

\def\needspace#1{\vskip0pt plus #1\penalty-250\vskip0pt plus -#1\relax}

\def\definition#1{{\sl #1\/}}
\def\latin#1{{\it #1\/}}

\def\URL{\leavevmode\begingroup\catcode`\#=12\catcode`\_=12\relax\@RL}
\def\@RL#1<#2>{\def\1{#1}\ifx\1\empty\def\2{#2}\else\def\2{#1}\fi
\ifcase\pdfoutput
 \ifx\dvips\undefined
  \special{pdf: bann << /Type /Annot /Subtype /Link /Border [0 0 0]
  /A << /Type /Action /S /URI /URI (#2) >> >>}{\color{Red}\2\endcolor}\special{pdf: eann}\else
  \special{html:<a href=\string"#2\string">}{\color{Red}\2\endcolor}\special{html:</a>}%
\fi
\else
 \pdfstartlink attr{/Border [0 0 0]}
  user{/Subtype /Link /A << /Type /Action
  /S /URI /URI (#2) >>}{\color{Red}\2\endcolor}\pdfendlink
\fi\endgroup\relax}

\catcode`\@=12
