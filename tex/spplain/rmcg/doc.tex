% doc.tex (RMCG19941017)

% Format: spplain

\ifx\comment\undefined \input explain \fi
\ifx\loadfont\undefined \input fonts \fi
\ifx\index\undefined \input index \fi

\catcode`\@=11

% Language

%\ifcase\language % 0 is English
% \def\PageSTR{Page}
%\or             % 1 is no-language
%\or             % 2 is Spanish
% \def\PageSTR{P\'agina}
%\else           % 3 is not defined
%\fi

% Characters

\def\ordinals#1{\def\tura##1{\leavevmode \raise0.85ex\hbox{#1\b{##1}}}}
 \font\tenss=cmss10 \font\eightss=cmss8
% \def\sf{\tenss\ordinals\eightss}

\def\ifmath$#1${\ifmmode #1\else$#1$\fi}

% Dagger is unicode 20-20,        UTF-8 E2-80-A0
% Double dagger is unicode 20-21, UTF-8 E2-80-A1
% Bullet is unicode 20-22,        UTF-8 E2-80-A2
% Tricolon is unicode 20-5D,      UTF-8 E2-81-9D
% EUR (€) is unicode 20-AC,       UTF-8 E2-82-AC
% Trade Mark is unicode 21-22,    UTF-8 E2-84-A2
\begingroup
\deactivate\dohigh
\expandafter\gdef\csname^^e2^^80^^a0\endcsname{\dag}
\expandafter\gdef\csname^^e2^^80^^a1\endcsname{\ddag}
\expandafter\gdef\csname^^e2^^80^^a2\endcsname{\ifmath$\bullet$}
\expandafter\gdef\csname^^e2^^81^^9d\endcsname{\ifmath$\vdots$}
\expandafter\gdef\csname^^e2^^82^^ac\endcsname{\EUR}
\expandafter\gdef\csname^^e2^^84^^a2\endcsname{\TradeMark}
\endgroup
\catcode`\^^e2=13
\def^^e2{\begingroup\edef\next{\string^^e2}\deactivate\dohigh\g@tnexttwo}
\def\g@tnexttwo#1#2{\csname \next#1#2\endcsname\endgroup}

% EUR (€) is A4 in ISO 8859-15
\catcode`\^^a4=13 \def^^a4{\EUR}

{\catcode`\^^a9=13 \gdef^^a9{\copyright}}
{\catcode`\^^ae=13 \gdef^^ae{\registered}}

\def\EUR{{\sc eur}}
\def\TradeMark{\leavevmode\raise0.5ex\hbox{\sc tm}}
\def\registered{{\ooalign{\hfil\raise.07ex\hbox{\sc r}\hfil\crcr
 \mathhexbox20D}}}

% Fonts

\newskip\ttglue

\def\cm#1fonts{\def\1{#1}\plainfont
 \ifx\1\empty\def\1{cm}\else\def\1{\romannumeral#1}\fi
 \csname\1fonts\endcsname \csname\1titles\endcsname
 \font\tentt=cmtt10 % usado por \tenttlistingoptions
 \font\euromr=feymr10 at #1pt
 \font\euromo=feymo10 at #1pt
 \font\eurobr=feybr10 at #1pt
 \font\eurobo=feybo10 at #1pt
 \def\EUR{\leavevmode\ifdim \fontdimen1\font>0pt
  \hbox{\euromo e}\else\hbox{\euromr e}\fi}%
 \def\EURbf{\leavevmode\ifdim \fontdimen1\font>0pt
  \hbox{\eurobo e}\else\hbox{\eurobr e}\fi}%
 \catcode`\^^a9=13 \catcode`\^^ae=13
}

\def\lm#1fonts{\ansifont \catcode`\^^e2=13
 \def\0{10}\def\1{#1}\ifx\1\0 \def\1{0}\fi
 \csname lm\romannumeral\1fonts\endcsname
 \csname lm\romannumeral\1titles\endcsname
 \font\tentt=texnansi-lmtt10 % usado por \tenttlistingoptions
 \def\EUR{\char1 }\def\TradeMark{\char153 }%
}

\def\ps#1fonts{\def\1{\romannumeral#1}\ansifont \catcode`\^^e2=13
 \csname ps\1fonts\endcsname \csname ps\1titles\endcsname
 \font\tentt=pcrr8r at 10pt % usado por \tenttlistingoptions
 \def\TradeMark{\char153 }%
 \font\euromr=feymr10 at #1pt
 \font\euromo=feymo10 at #1pt
 \font\eurobr=feybr10 at #1pt
 \font\eurobo=feybo10 at #1pt
 \def\EUR{\ifdim \fontdimen1\font>0pt
  \hbox{\euromo e}\else\hbox{\euromr e}\fi}%
 \def\EURbf{\ifdim \fontdimen1\font>0pt
  \hbox{\eurobo e}\else\hbox{\eurobr e}\fi}%
}

% Figures

\newcount\figno
\newcount\prefigpenalty \prefigpenalty=200
\newcount\postfigpenalty \postfigpenalty=200
\newskip\prefigskip \prefigskip=\bigskipamount
\newskip\postfigskip \postfigskip=\bigskipamount
\newskip\capfigskip \capfigskip=\smallskipamount

\catcode`\"=12 % should be redundant

\def\MTfigure(#1,#2,#3);{\catcode`\"=12
 \ifx\MT\undefined \input metatex
  \MTline{if unknown make_del: input delay; fi}\fi
 \MTline{}\MTbeginchar(#1,#2,#3);\MTline{ save u,v; thinpen;}}

\def\MTendfigure{\futurelet\n@xt\MTendfig@re}
\def\MTendfig@re{\MTendchar;%
 \penalty\prefigpenalty\vskip\prefigskip
 \centerline{\box\MTbox}%
 \if ;\n@xt \let\next=\MT@ndfig \else \let\next=\MT@ndf@g \fi \next}
\def\MT@ndf@g"#1";{\global\advance\figno1
 \nobreak \vskip\capfigskip
 \centerline{{\bf Figura \number\figno:} #1}\MT@ndfig;}
\def\MT@ndfig;{\catcode`\"=13
 \penalty\postfigpenalty\vskip\postfigskip}

% Verbatim

\def\otherquote{"} % Catcoded 12 (other) quote (")
\catcode`\"=13 \edef"{\noexpand\verb\otherquote} % Minimal markup
\expandafter\def\expandafter\dospecials\expandafter{\dospecials\do\"}
%\catcode`\"=13 \def"{\verb"} % Minimal markup
%\def\verbatimoptions{\def"{\string"}}

\def\file{\ifvmode\vskip\parskip\else\par\fi
 \begingroup \lineskip=3pt \ifdim\parindent<20pt \parindent=20pt \fi
 \catcode`\~=12 \catcode`\"=12 \catcode`\_=12 \catcode`\\=12 \@file}
\def\@file#1{\index{#1}1
 \hbox{\indent\vbox{\hrule\hbox{\strut\Blue\tt#1\Black}\hrule}}\nobreak
 \let\verbatimoptions=\tenttlistingoptions \setupverbatim \input#1
 \dimen0=\prevdepth \advance\dimen0by-3pt \prevdepth=\dimen0
 \everypar{}\noindent\endgroup \ignorespaces}
\def\tenttlistingoptions{\tentt
 \baselineskip=12pt plus0.1pt minus0.1pt
 \parindent=40pt \rightskip=0pt plus 3cm
 \everypar{\hangindent60pt \llap{\sevenrm\the\inputlineno\quad}}}

\def\pfile{\begingroup\let\beforeplisting=\tenttplisting
 \let\afterplisting=\endgroup\plisting}
\def\tenttplisting{\ifdim\lastskip<\smallskipamount
 \removelastskip \smallskip\fi \index{\1}1
 \let\verbatimoptions=\tenttlistingoptions
 \ifx\1\@pfile \else \xdef\@pfile{\1}%
  \hbox{\indent\vbox{\hrule\hbox{\strut\Blue\tt\1\Black}\hrule}}\nobreak
 \fi}

\def\numbercodeoptions{\tentt \parindent=40pt \rightskip=0pt plus 3cm
 \baselineskip=12pt plus0.1pt minus0.1pt \count255=0
 \everypar{\hangindent60pt\advance\count255 1
  \llap{\sevenrm\the\count255\quad}}}
\def\codeoptions{\parindent=20pt}
\def\begincode{\ifvmode \vskip\parskip \else \par\fi
  \begingroup\let\verbatimoptions\codeoptions\setupverbatim\d@code}
\catcode`/=0 \catcode`\\=12
 /def/d@code{/def/next##1##2\endcode{##2/everypar{}/noindent/endgroup
  /ignorespaces}/next}
/catcode`\=0 /catcode`\/=12

%\saveplain\| \def\|{\penalty100\smallskip\verb|\aftergroup\smallbreak}
%\saveplain\" \def\"{\verb"} % NO. It breaks dieresis, \"u.

% To mark blocks of text to be exported
%    To import use \filed filename.ext \block label
\def\beginblock #1 \par{\relax}
\def\endblock #1 \par{\relax}
\let\block=\beginblock

% Other

% Every point is a period
\count255=`A \loop\sfcode\count255=1000
 \ifnum\count255<`Z\advance\count255 1 \repeat
\def~{\nobreak\ \ignorespaces}

% Emphasis with automatic italic correction (\/).
% Use: {\em italic, but {\em roman}, text}.
\def\em{\ifdim \fontdimen1\font>0pt \rm
 \else \it \expandafter\aftergroup \fi \smartitc@r}
\def\smartitc@r{\ifhmode \expandafter\itpuncl@ok \fi}
\def\itpuncl@ok{\begingroup\futurelet\ITCt@mpa\itcort@st}
\def\itcort@st{\def\ITCt@mpb{\ITCt@mpa}%
 \ifcat\noexpand\ITCt@mpa,\setbox0=\hbox{\ITCt@mpb}%
  \ifdim\ht0<0.3ex \let\itc@rdo=\endgroup \fi\fi \itc@rdo}
\def\itc@rdo{\skip0=\lastskip \ifdim\skip0=0pt \/\else
 \unskip \/\hskip\skip0 \fi \endgroup}

\let\latin=\em
\def\person#1{{\sc#1}}
\def\[#1]{{\sc#1}}

\def\beginalign{$$\displayindent=\parindent \offinterlineskip
  %\abovedisplayskip=\smallskipamount \belowdisplayskip=\smallskipamount
  \def\rule{\noalign{\vskip0pt\hrule\vskip1pt}}\def\\{\cr}%
  \halign\bgroup\strut ##\hfil&&\quad##\hfil\cr}
\def\endalign{\crcr\egroup$$}

\def\raggedright{\rightskip=0pt plus 3em\relax}

\saveplain\* \def\*#1*{\subitem*#1*}
\let\\=\newline
\let\describe=\titledpar
\let\point=\bulletedpar
\let\subpoint=\subbulletedpar

\saveplain\> \def\>{\ifmmode\plain\>\else
 \ifdim\parindent>0pt \dimen0=\parindent \else \dimen0=20pt \fi
 \hbox{\kern\dimen0}\ignorespaces\fi}

% Default tabs
\settabs\+\indent&\indent&\indent&\indent&\indent&\indent&\cr

% Lists

\def\it@m#1{\par\noindent\llap{#1\enspace}\ignorespaces}

\newcount\l@st
\def\beginlist{\par \begingroup
 \ifnum\l@st=0\parskip=0pt \ifdim\parindent=0pt \parindent=20pt\fi \fi
 \global\advance\l@st1 \leftskip=\l@st\parindent}
\def\endlist{\par\global\advance\l@st-1\endgroup}

\def\itemi{\it@m{$\bullet$}}
\def\itemii{\it@m{$\circ$}}
\def\itemiii{\it@m{$\cdot$}}
\def\itemiv{\par\noindent}
\def\beginpoints{\beginlist
 \expandafter\let\expandafter\item\csname item\romannumeral\l@st\endcsname}
\let\endpoints=\endlist

\newcount\l@@t
\def\numbersstyle#1{\ifcase#1\or \def\next{\bf\number}\or
  \def\next{\number}\else \def\next{\romannumeral}\fi \next}
\def\beginnumbers{\beginlist\l@@t=0
 \def\item{\advance\l@@t1 \it@m{\numbersstyle{\number\l@st}\l@@t}}}
\let\endnumbers=\endlist

\def\begintitles{\beginlist
 \def\item##1{\par\noindent\kern-\parindent{\bf ##1}\enspace\ignorespaces}}
\let\endtitles=\endlist

% Lines

%%% Leave the blank line after the \let, or you will get an error!
\let\eol=^^M

\def\beginlines{\unskip\begingroup \let\par=\newline \obeylines}
\def\endlines{\endgroup\futurelet\next\endlin@s}
\def\endlin@s{\ifcat\eol\next \unpenalty\fi}

% Footnotes

\saveplain\footnote % was redefined for pdf in index.tex

\newcount\footnoteno
\def\numberedfootnote{\global\advance\footnoteno1
 \plain\footnote{$^{\number\footnoteno}$}}
\let\footnote=\numberedfootnote

\dimen\footins=0.5\vsize
%\def\footnoterule{\nobreak\kern-3pt
% \nobreak\hrule width 3cm \nobreak\kern2.6pt \nobreak}

% Sectioning

\newcount\s@zero \newcount\s@one \newcount\s@two
\newcount\s@three \newcount\s@four \newcount\s@five
\def\incsec#1{\ifnumbering\ifcase #1\advance\s@zero by 1
  \s@one=0 \s@two=0 \s@three=0 \s@four=0 \s@five=0 \or
 \advance\s@one by 1 \s@two=0 \s@three=0 \s@four=0 \s@five=0 \or
 \advance\s@two by 1 \s@three=0 \s@four=0 \s@five=0 \or
 \advance\s@three by 1 \s@four=0 \s@five=0 \or
 \advance\s@four by 1 \s@five=0 \or
 \advance\s@five by 1 \fi\fi}

\def\secid{\ifnum\s@one>0
 \ifappendix \ifcase\s@one X\or A\or E\or I\or O\or U\else X\fi
  \else \the\s@one \fi
 \ifnum\s@two>0 .\the\s@two \ifnum\s@three>0 .\the\s@three
 \ifnum\s@four>0 .\the\s@four \ifnum\s@five>0 .\the\s@five \fi\fi\fi\fi\fi}
\def\chapid{\ifnum\s@zero>0
 \uppercase\expandafter{\romannumeral\s@zero}~\fi \secid}

\newif\ifappendix
\def\appendix{\appendixtrue \s@one=0 }

% \title1 Section \par makes Section a level one title

\def\title#1 {\par \incsec{#1}\let\next=\titlesp
 \ifcase #1\let\next=\titlezero \or \let\next=\titleone \or
 \let\next=\titletwo \or \let\next=\titlethree
 \or \let\next=\titlefour \or \let\next=\titlefive \fi \next}

\newif\ifparra
\everypar=\expandafter{\the\everypar\global\parratrue}

\newif\ifnumbering \numberingtrue
\def\nonumbers{\numberingfalse}
\def\numbers{\numberingtrue}

\newif\iflabeled
\def\labeled#1\title{\def\thel@bel{#1}\labeledtrue\title}
\def\l@bel#1{\iflabeled \setbox0=\hbox{\thel@bel}%
 \ifdim\wd0=0pt\lbl{#1}{\secid}\else\lbl{\thel@bel}{\secid}\fi
 \global\let\thel@bel\null \global\labeledfalse \fi}
\def\regindex#1#2{\pdflabel\ifnumbering\toc#1{#2}\fi\l@bel{#2}}

\def\titlezero#1 \par{\vfill\break\vglue30pt
 \begincenter \parskip=0pt\baselineskip24pt
  \fontzero #1\pdflabel\toc0{#1}\endcenter
 \parrafalse\bigskip\bigskip}
\def\titleone#1 \par{\goodpage\vskip\baselineskip\vskip\parskip
 \titleline{\fontone \noindent \ifnumbering \secid \quad\fi}%
 {\baselineskip=17pt \fontone #1\regindex1{#1}}\parrafalse\nobreak}
\def\titletwo#1 \par{\vskip\parskip \ifparra\vskip\baselineskip\fi
 \titleline{\noindent\fonttwo \ifnumbering\secid\quad\fi}%
 {\baselineskip=15pt \fonttwo #1\regindex2{#1}}\parrafalse\nobreak}
\def\titlethree#1 \par{\vskip\parskip \ifparra\vskip\baselineskip\fi
 \titleline{\noindent\bf \ifnumbering\secid\quad\fi}%
 {\bf #1\regindex3{#1}}\parrafalse\nobreak}
\def\titlefour#1 \par{\par\noindent\regindex4{#1}%
 {\bf\ifnumbering\secid~\fi #1}\quad}
\def\titlefive#1 \par{\par\noindent\regindex5{#1}%
 {\sl\ifnumbering\secid~\fi #1}\quad}
\def\titlesp#1 \par{\vfill\break\vglue30pt
 \begincenter \parskip=0pt\baselineskip24pt
  \fontzero #1{\let\secid\empty\pdflabel\toc8{#1}}\endcenter
 \bigskip\bigskip}

\def\titleline#1#2{\setbox0\hbox{#1}\dimen0=\hsize \advance\dimen0 -\wd0
 \line{\box0\hss\vtop{\hsize=\dimen0 \raggedright\let\\=\ \noindent #2}}}
\def\goodpage{\vskip0pt plus.3\vsize\penalty-250\vskip0pt plus-.3\vsize}

% TOC

\def\tocline#1{\ifcase #1\let\next=\toclinezero \or
 \let\next=\toclineone \or \let\next=\toclinetwo \or
 \let\next=\toclinethree \or \let\next=\toclinefour \or
 \let\next=\toclinefive \else \let\next=\toclinesp \fi \next}

\def\tocln#1#2#3#4#5#6{\pdftocline{#1}{#2}\contentsline{#4}{#5}{#3}{#6}}

\def\toclinezero#1#2#3#4{\bigbreak\vskip1pc \pdftocline{#1}{#2}%
 \begincenter\parskip=0pt\baselineskip24pt\fontzero #1\endcenter\vglue1pc}
\def\toclineone#1#2#3#4{\bigbreak \def\3{#3}%
 \tocln{#3 #1}{#2}{\ifx\3\null\else\bf#3\quad\fi}{\bf#1}{#2}{#4}}
\def\toclinetwo#1#2#3#4{\medbreak \def\3{#3}%
 \tocln{#3 #1}{#2}{\quad\ifx\3\null\else #3\quad\fi}{#1}{#2}{#4}}
\def\toclinethree#1#2#3#4{\def\3{#3}%
 \tocln{#3 #1}{#2}{\qquad \ifx\3\null\else #3\quad\fi}{#1}{#2}{#4}}
\def\toclinefour#1#2#3#4{\def\3{#3}%
 \tocln{#3 #1}{#2}{\qquad\qquad\ifx\3\null\else#3\quad\fi}{#1}{#2}{#4}}
\def\toclinefive#1#2#3#4{\def\3{#3}%
 \tocln{#3 #1}{#2}%
 {\qquad\qquad\qquad \ifx\3\null\else #3\quad\fi}{#1}{#2}{#4}}
\def\toclinesp#1#2#3#4{\bigbreak \tocln{#1}{#2}{}{\bf#1}{#2}{#4}}

\def\contentsline#1#2#3#4{\setbox0\hbox{#3}\setbox2\hbox{#1}%
 \dimen0=\hsize \advance\dimen0 by -\wd0
 \multiply\dimen0 by 8 \divide\dimen0 by 10
 \ifdim\dimen0>\wd2 \line{\box0 {#1}\tocleaders \pdfgoto{#4}{#2}}\else
  \line{\box0 \vtop{\hsize=\dimen0 \raggedright \normalbaselines
   \let\\=\ \noindent #1\strut}\tocleaders \pdfgoto{#4}{#2}}\fi}

\def\tocleaders{\leaders\hbox to\baselineskip{\hss\bf.\hss}\hfil}

% Bibliographic refs

\newcount\refno
\def\bibitem#1{\par\global\advance\refno by 1
 \noindent\hang[\Green\number\refno\Black]\label{#1}{\number\refno}%
 \quad\ignorespaces}
\def\cite#1{~[\ref{#1}]}

% Layouts

\def\pdfsetpagesize{\pdfcode \pdfpageheight=\vsize
  \advance\pdfpageheight2in \advance\pdfpageheight2\voffset
 \pdfpagewidth=\hsize \advance\pdfpagewidth2in
  \advance\pdfpagewidth2\hoffset \pdfendcode}

\let\plainmakeheadline=\makeheadline

\def\plainlayout{\vfill\break
 \pdfsetpagesize\advance\vsize -2\baselineskip
 \headline={\hfil}\footline={\hss\tenrm\folio\hss}%
 \let\makeheadline=\plainmakeheadline}

\let\header=\relax

\def\rmcglayout{\vfill\break \cm12fonts \colors
 \parskip=\medskipamount \parindent=0pt
 \nopagenumbers \pdfsetpagesize \advance\vsize -30pt
 \headline={\rm \Author \quad \todayiso \quad \header
  \hfil \quad{\tt \jobname}\quad {\bf\Folio}}%
 \def\makeheadline{\vbox to 30pt{\line{\the\headline}%
  \kern 1pt \hrule height 1pt\vfil}\nointerlineskip}}

\def\Author{{\tt www.ramoncasares.com}}
\def\docinfo{{\tt\jobname\ \copyright\ \Author\ \todayiso}}
%\def\infodoc{{\tt\jobname\ \copyright\ \Author\ \todayiso}}
\let\infodoc=\docinfo

\def\URL#1<{\def\1{#1}\begingroup\setupverbatim\@RL}
\def\@RL#1>{\ifx\1\empty\def\2{{\tt#1}}\else\def\2{{\rm\1}}\fi
 \ifcase\pdfoutput \2\else
  \pdfstartlink attr{/Border [0 0 0]}
   user{/Subtype /Link /A << /Type /Action
    /S /URI /URI (#1) >>}{\Red\2\Black}\pdfendlink
 \fi\endgroup}

\newdimen\longindentation \longindentation=10truecm
\let\date=\fecha \let\address=\relax

\def\PILI{Pilar Porto Castelao}
\def\PILIemail{\URL{\tt pili@ramoncasares.com}<mailto:pili@ramoncasares.com>}
\def\RMCG{Ram\'on Casares Gallego}
\def\RMCGemail{\URL{\tt papa@ramoncasares.com}<mailto:papa@ramoncasares.com>}
\let\quien=\RMCG
\let\eaddress=\RMCGemail

\def\letterlayout{\vfill\break
 \headline={\hfil}\footline={\hfil}\let\makeheadline=\plainmakeheadline
 \lm12fonts \parindent=0pt \parskip=0pt
 \def\firma{\vskip3pc \leftskip=\longindentation \quien}%
 \def\Firma##1{\leftskip=\longindentation \vskip5pc
  \noindent\vbox{\let\\=\cr\halign{&\hfil####\hfil\cr##1\crcr}}}%
 \def\PD{\vskip0pt plus 1filll\leftskip=0pt\relax}%
 \let\\=\par
 \leftskip=\longindentation
  {\sf \ordinals\eightss \quien \\
  San Andr\'es, 112 (4\tura{o} piso)\\
  15003 A Coru\~na\\
  \vskip 3pt \hrule height 1pt \vskip3pt
  \strut Tfno: 981 229 531\\
  \strut Email: \eaddress\\ % <---
  \vskip 1pc}
   \date\\ % <---
  \vskip 1pc
  \leftskip=0pt
   \address\\ % <---
  \parskip=\medskipamount \rightskip=0pt \bigskip}

\def\booklayout{\vfill\break \cm10fonts
 \headline={\hfil}
 \footline={\tenrm\ifodd\pageno \docinfo\hfil\folio
             \else \folio\hfil\infodoc \fi\strut}
 \hsize=10.5cm \hoffset=-1in \advance\hoffset by 21.75mm
 \vsize=41\baselineskip \advance\vsize\topskip % 41 + 1 lines
 \voffset=-1in \advance\voffset by 16.76mm
 \pdfsetpagesize \advance\vsize -2\baselineskip % - 2 = 40 lines
 \parskip=0pt plus 0.0001pt minus 0.0001pt
 \parindent=20pt }

% pdfa xmp defaults
\ifx\xmpTitle\undefined \def\xmpTitle{\jobname}\fi
\ifx\xmpAuthor\undefined \def\xmpAuthor{Ramón Casares}\fi
\ifx\xmpOrg\undefined \def\xmpOrg{www.ramoncasares.com}\fi
\ifx\xmpCopyright\undefined \def\xmpCopyright{© Ramón Casares \the\year}

 % #1 <- by-sa, etc
\def\xmpcc#1{\def\xmpWebStatement{http://creativecommons.org/licenses/#1/4.0}%
 \def\xmpCertificate{http://creativecommons.org/licenses/#1/4.0/legalcode}%
 \def\xmpUsageTerms{\xmpCopyright, cc-#1}}

\catcode`\@=12

